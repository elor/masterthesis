\section{Stand der Forschung}
\label{present}

Molekulardynamische und kinetische Monte Carlo-Methoden wurden wiederholt zur Simulation von verschiedenen Aspekten von Gasphasenabscheidungen zu Rate gezogen.
Im Folgenden soll ein Überblick über die bisherigen Untersuchungen gegeben werden.
Referenz~\cite{dollet_multiscale_2004} gibt einen Überblick über weitere Multiskalen-Modelle für Gasphasenabscheidungen, von denen die meisten keine strukturellen Aussagen ermöglichen, sondern sich auf Aussagen auf der Reaktor-Skala beschränken.

\subsection{Anwendungen von KMC-Simulationen für die Gasphasenabscheidung}

\subsubsection{Dwivedi: gitterbasierte 2D-ALD}

\textsc{V. Dwivedi} und \textsc{A. Adomaitis} haben ein zweidimensionales KMC-Modell für \ce{Al2O3}-ALD entwickelt\cite{dwivedi_multiscale_2009,dwivedi_multiscale_2009-1,dwivedi_multiscale_2010}, um den Gasverbrauch und den GPC-Wert in Abhängigkeit der Partialdrücke der Precursorgase zu ermitteln und in eine Gasflusssimulation zur Kontrolle des Durchmessers von Mikroporen einzubinden.
Dazu werden die Plätze eines hexagonalen Gitters auf die Spezies Sauerstoff, Hydroxyl und Aluminium aufgeteilt, Zustandsübergänge in Abhängigkeit der 6 Nachbarzellen definiert und Übergangsraten ausgehend von den Energiebarrieren der entsprechenden chemischen Reaktionen ermittelt.
Aussagen über die Schichthöhe ergeben sich über die Festlegung des vertikalen Gitterabstandes anhand der Einheitszelle eines \ce{Al2O3}-Kristalles mit anschließender vertikaler Oberflächensuche.

Damit lassen sich zwar GPC-Werte und Precursorverbrauch in Annäherung bestimmen, doch ist durch die festlegung auf das zweidimensionale Simulationsgitter keine strukturelle Aussage möglich, obwohl passend gewählte Ereignisse die \ce{Al2O3}-Stöchiometrie erzwingen.
Eigene Untersuchungen des Dwivedi-Modelles resultierten in der Bildung von vertikalen, isolierten \ce{Al-O}-Ketten sowie in der Verstärkung von Kratern und konnten die präsentierten GPC-Werte nicht reproduzieren.

\subsubsection{Mazaleyrat: gitterbasierte 3D-ALD}

Ähnlich zum Dwivedi-Modell nutzt auch das Modell von \textsc{Mazaleyrat et al.}\cite{mazaleyrat_methodology_2005} ein Simulationsgitter, das aber über drei Dimensionen definiert ist, auf dem \ce{MgAl2O4}-Spinell-basiert und als Annäherung der eigentlichen Atompositionen verstanden wird.
Durch geschickte Indizierung des Gitters wird eine effiziente Methode geschaffen, \ce{Al2O3} in KMC-Simulationen aufzuwachsen, deren strikt kristalline Repräsentation die eigentlich amorphe Struktur nicht darstellen kann, obwohl das Ziel des Modelles in der Abscheidung von Dielektrika auf Silizium-Substraten liegt.
Von den Autoren wurde keine Betrachtung der Rauheit, des Besetzungsgrades der Gitterplätze oder des Wachstumsverhaltens jenseits des dritten Schrittes veröffentlicht.
%% {Anmerkung von Jörg: eigene Erfahrungen mit diesem Modell ergänzen}
%% {Ich habe keine eigenen Erfahrungen mit diesem Modell}

\subsubsection{Stamatakis: Oberflächen-Reaktionen mit Zacros}

Zwar handelt es sich beim Zacros-Modell von der Forschergruppe um \textsc{M. Stamatakis}\cite{stamatakis_graph-theoretical_2011,nielsen_parallel_2013,stamatakis_zacros_2014} um eine zweidimensionale Oberflächenbeschreibung für katalytische Reaktionen, doch liegt nahe, seinen effizienten graph-basierten Suchansatz für Ereignisse auch für Gasphasenabscheidungen nutzen zu wollen.
Die Besonderheit von Zacros liegt in einer vereinheitlichten Formulierung der lokalen Zustände und Zustandsübergänge als Teilgraphen des Simulationsgitters, welche effizient durch Algorithmen zum Teilgraphen-Vergleichs überprüft werden können.
Ergänzt durch quantenmechanische Simulationen für die Bestimmung der Reaktionsraten, ergibt sich ein wertvolles Werkzeug zur stochastischen Simulation der Reaktionskinetik von Oberflächenprozessen.
Aufgrund seiner Formulierung ist aber auch Zacros auf ein periodisches Simulationsgitter begrenzt, welches durch seine Beschreibung als Graph jedoch aus beliebig geformten Einheitszellen bestehen kann.

Die Kombination der in der vorliegenden Arbeit vorgestellten Ansätze mit denen von Zacros ist Gegenstand des derzeit in Beantragung befindlichen EU-Projektes ACCELERATE, einem Konsortium zur ALD-Simulation bestehend aus dem Fraunhofer ENAS, der Gruppe um Stamatakis und weiteren Partnern.\todo{Ref auf Accelerate? Es gibt eine ganze Menge von Accelerate-Projekten.}

\subsubsection{Clark: Off-Lattice-Simulationen von chemischen Gasphasenabscheidungen}
\todoline{Jörg}
\textsc{Clark et al.}\cite{clark_hybrid_1996} haben eine Off-Lattice-Methode zur Simulation von Diamantwachstum per CVD entwickelt, die auf der Kombination von KMC- mit Simulated Annealing per Metropolis-Monte-Carlo-Simulationen (MMC) basiert.
Dafür wird die lokale Nachbarschaft jedes KMC-Ereignisses, welches aus der Bindung eines Precursormoleküles an die Oberfläche besteht, durch MMC-Simu\-la\-tionen relaxiert, während hochfrequente Bewegungen von Wasserstoffatomen durch deren stochastische Verschiebung auf der Oberfläche beschrieben werden.
Die MMC-Simulationen nutzen für die energetische Beschreibung der Ereignisorte molekulardynamische Kraftfelder, doch werden zusätzliche Zwangsbedingungen eingeführt, die stabile Bindungen erzwingen.
Damit reduziert sich Simulationszeit, doch müssen mögliche Ereignisse durch in Vorarbeit erforscht werden, statt vom Modell dynamisch erkannt zu werden.

%% Die kombinierte Simulation unterteilt sich in die Betrachtung von hochfrequenten Gleichgewichts-Ereignissen per KMC-Methoden und sowie die Simulation von Adsorptionen und Reaktionen durch MMC-Relaxationen anhand von MD-Kraftfeldern innerhalb der Nachbarschaft.
%% In der ersten Phase des Zyklus' werden hauptsächlich Wasserstoffatome an der Oberfläche durch KMC-Ereignisse verschoben und entsprechend der erwarteten Oberflächenbedeckung ad- und desorbiert.
%% Anschließend werden Precursorfragmente in Form von Methylgruppen an ein Kohlenstoff-Atom an der Oberfläche gebunden und bei konstanter Bindungslänge per MMC-Simulation um dieses rotiert.
%% Zuletzt werden Bindungen zwischen Kohlenstoff-Atomen in der Nachbarschaft des Ereignisortes gebildet, indem sie in einer weiteren MMC-Simulation verschoben werden.

Dieses Modell ist dem im folgenden Abschnitt~\ref{parsivald} vorgestellten Parsivald-Modell in seiner Funktionsweise ähnlich, doch führt letzteres eine thermische Relaxation der Ereignisorte per Molekulardynamik durch.
Diese ist gegenüber MMC-Simulationen zwar rechenaufwendiger, erlaubt im Gegenzug aber die freie Bewegung der Teilchen und Moleküle einerseits, andererseits lässt sie reaktive Kraftfelder und den dafür notwendigen kontinuierlichen Ladungsaustausch zu.
%% Eine Erweiterung des Parsivald-Modelles zu MMC-Relaxierungen wäre aber in Zusammenhang mit der Einbindung von Elektronenstrukturrechnungen interessant.

\subsubsection{Biehl: Off-Lattice-KMC von verspannten Kristallen}
\todoline{Jörg}

In den Veröffentlichungen von \textsc{Biehl et al.}\cite{biehl_off-lattice_2005} wird ein KMC-MD-Hybrid zur zweidimensionalen Off-Lattice-Simulation von Verspannungen in heteroepitaxial gewachsenen Kristallen vorgestellt, der neben Adsorptionsereignissen auch Oberflächendiffusionen betrachtet.
Die Relaxation von Ereignisorten wird bei diesem Modell durch eine nicht näher benannte Energieminimierung der Position einzelner Atome durchgeführt.
Zusätzlich findet in großen zeitlichen Intervallen eine kostspielige globale Energieminimierung statt, die laut der Autoren notwendig ist, obwohl nur kleine Änderungen der Struktur durchgeführt werden.
Da die Anwendung in der Simulation von Kristallen liegt, wurden Lennard-Jones-Potentiale zur Darstellung von Kristallstrukturen gewählt, wodurch sich sehr stabile Zustände aufgrund von glatten Energielandschaften ergeben.
Damit ist nicht ersichtlich, ob sich diese Methode auch für komplizierte dreidimensionale Strukturen und Kraftfelder eignet.

\subsubsection{Fazit}

KMC-Simulationen werden zur atomistischen Simulation von Gasphasenabscheidungen genutzt, doch sind die Atompositionen häufig auf die Punkte eines Simulationsgitters begrenzt.
Damit können zwar große Simulationsräume effektiv betrachtet werden, doch werden realistische Beschreibungen der oftmals amorphen Materialien verhindert.

Vereinzelt existieren auch atomistische Off-Lattice-Ansätze für KMC-Simulationen, die für die Durchführung von Ereignissen häufig auf molekulardynamische Simulationen zurück greifen.
Diese nutzen meist N-Teilchen-Potentiale in vergleichsweise kleinen Simulationsräumen, wodurch sich amorphe Strukturen oder Oxide nur unzuverlässig beschreiben lassen, doch beschränken sich die untersuchten Arbeiten ohnehin auf kristallines Wachstum oder Verspannungen in Kristallen.

\subsection{Anwendung von MD-Simulationen für die Gasphasenabscheidung}
\todoline{Arbeiten von Hermann Wolf und Co. - welche? Konnte nur ein Paper finden}

Molekulardynamische Simulationen werden seit vielen Jahrzehnten zur Bestimmung struktureller und thermodynamischer Eigenschaften verschiedener Materialien eingesetzt, aber nur selten zur Beschreibung von Gasphasenabscheidungen.
Sie lassen sich aber zur Beschreibung von Teilaspekten nutzen, wofür meist durch die Simulation periodischer Strukturen Rückschlüsse auf bestimmte Eigenschaften gezogen werden.

\subsubsection{Gold}
\textsc{Chamati et al.}\cite{chamati_second-moment_2004} haben das thermische Verhalten von Gold-Bulksystemen untersucht, wo hingegen \textsc{Chui et al.}\cite{chui_molecular_2007}, \textsc{Liu et al.}\cite{liu_melting_2001} und \textsc{Shim et al.}\cite{shim_molecular_2003} strukturelle und thermodynamische Eigenschaften von Gold-Nanoclustern betrachtet haben.
Veröffentlichungen zu MD-Simulationen von Aspekten der Gold-PVD finden sich ebenfalls vereinzelt, doch simulieren die meisten den Einschlag von Edelgas-Ionen auf dem Target mit Energien von \SIrange{1}{400}{\kilo\electronvolt}\cite{insepov_molecular_1995,shapiro_simulation_1999} oder tragen komplette Cluster statt einzelner Atome auf\cite{inoue_molecular_2008}.

\subsubsection{Kupfer und Nickel}
Kupfer-Nickel-Multilagensysteme wurden unter anderem von \textsc{Foiles et al.}\cite{foiles_calculation_1985} simuliert, wobei der Einfluss der Sputterenergie der Atome auf die Rauheit der abgeschiedenen Oberflächen und Schichten hauptsächlich von \textsc{Zhou et al.}\cite{zhou_atomistic_1998} untersucht wurde.
Der durch unterschiedliche Bindungslängen entstehende Versatz zwischen den Kupfer- und Nickel-Kristallen wurde von \textsc{Rao et al.}\cite{rao_atomistic_2000} simuliert.

\subsubsection{Silizium}
Für Silizium finden sich thermodynamische Expansions-Simulationen von \textsc{Buda et al.}\cite{buda_thermal_1990} sowie Untersuchungen von \textsc{Insepov et al.}\cite{insepov_molecular_1995} zur Auswirkung von gerichteten Cluster-Einschlägen in Abhängigkeit der Einschlagsenergie.
\textsc{Song et al.}\cite{song_reactive_2012} haben außerdem die Oxidierung von Silizium-Nanopartikeln mit reaktiven Kraftfeldern simuliert.
%% {mehr! PVD und Oxidation / Hydroxylierung von Oberflächen} - Gibt's nicht

\subsubsection{Aluminiumoxid}
Für amorphe und kristalline \ce{Al2O3}-Bulksysteme finden sich Simulationsergebnisse von \textsc{Alvarez et al.}\cite{alvarez_computer_1995,alvarez_molecular_1992} und von \textsc{Gutierrez et al.}\cite{gutierrez_molecular_2002} sowie für \ce{Al2O3}-Oberflächen von \textsc{Adiga et al.}\cite{adiga_atomistic_2006}.
Sowohl Schmelzen als auch amorphes \ce{Al2O3} wurden unter anderem von \textsc{Gutierrez et al.}\cite{gutierrez_structural_2000} und \textsc{Vashishta et al.}\cite{vashishta_interaction_2008} simuliert.
\textsc{Russo et al.}\cite{russo_molecular_2011} haben erfolgreich Oberflächen-Reaktionen zwischen einem Aluminium-Cluster und Wassermolekülen unter Nutzung der ReaxFF-Formulierung simuliert, während \textsc{Puri et al.}\cite{puri_thermo-mechanical_2010} das Schmelz- und Diffusionsverhalten von \ce{Al2O3}-ummantelten \ce{Al}-Nanopartikeln untersuchten.
Diese Vielfalt an Simulationen zeigt sich jedoch nicht in der Zahl der veröffentlichten Potentialparametrisierungen, von denen nur eine kleine Zahl für \ce{Al2O3}-Simulationen zur Verfügung stehen.

\subsubsection{Fazit}
Die Untersuchungen von Gasphasenabscheidungsprozessen mittels Molekulardynamik beschränken sich aufgrund der langen Relaxationszeiten und des Rechenaufwandes unter Nutzung reaktiver Kraftfelder.
Wenige Untersuchungen simulieren deshalb mehr als \num{50000} Atome, weshalb sich besonders bei der Bestimmung struktureller Eigenschaften Finite Size-Effekte bemerkbar machen können.
Darüber hinaus werden einige der genutzten Potentialparametrisierungen nicht veröffentlicht, wodurch die entsprechenden Ergebnisse nicht reproduzierbar oder auf andere Probleme übertragbar sind.
Dem steht die Vielzahl von verschiedenen Parametrisierungen gegenüber, von denen jede einzelne nur auf ein spezielles Problem anwendbar ist.
Eine Suche nach EAM-Kraftfeldern für Kupferatome in der Potentialdatenbank des NIST\cite{becker_interatomic_2014} ergibt beispielsweise eine zweistellige Zahl an Kupfer-Parametrisierungen, die einzeln hinsichtlich eigener Problemstellungen überprüft werden müssen (Abschnitt~\ref{copperpvd}).
