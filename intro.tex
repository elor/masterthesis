\chapter{Einleitung}
\label{intro}

%% {Aktueller Stand der Technologien und damit verbundene Herausforderungen}
Gasphasenabscheidungen werden zur Herstellung konformer, dünner Schichten auf strukturierten und glatten Substraten genutzt, wobei besonders die Erzeugung ultradünner, hoch reiner, homogener Schichten eine Schlüsselstellung bei der Fertigung von Bauelementen in der Nanoelektronik einnimmt\cite{granneman_thin_1993}.
Sie sind beispielsweise für die Produktion dünner Dielektrika\cite{gordon_vapor_2001}, Interconnects\cite{waechtler_copper_2009} und Diffusionsbarrieren\cite{granneman_thin_1993,raaijmakers_low_1994} interessant, für die besonders präzise Abscheidungsmethoden notwendig sind.
%% {weitere Anwendungen von Gasphasenabscheidung}
Gasphasenabscheidungen werden auch für eine Vielzahl anderer Anwendungen verwendet, wie beispielsweise für die Produktion von passivierenden und protektiven Beschichtungen\cite{yun_passivation_2004,poodt_high-speed_2010,higashiwaki_algan/gan_2006}.
Weiterhin erlauben sie die Herstellung mehrlagiger Materialien mit besonderen mechanischen (erhöhte Festigkeit\cite{cammarata_nanoindentation_1990}), magnetischen (Riesenmagnetowiderstand\cite{peter_influence_2007,seyama_giant_1999,bird_giant_1995}, magnetischer Tunnelwiderstand\cite{sun_magnetic_2006}) und optischen Eigenschaften (Röntgenspiegel\cite{jankowski_subnanometer_1989}).
Doch auch größere Oberflächen werden per Gasphasenabscheidung mit Dünnschichten versehen, etwa zur Erhöhung der Witterungs- und Wärmebeständigkeit von Oberflächen\cite{mccurdy_successful_1999}.

%% {Kandidat zur Lösung von Herausforderungen}
Bei Gasphasenabscheidungen adsorbieren einzelne Atome oder Moleküle aus der Gasphase auf der Oberfläche und bilden so eine dünne Schicht, deren Dicke präzise im Sub-Nanometer-Bereich kontrolliert werden kann\cite{mattox_handbook_2010,pierson_handbook_1999}.
Mit der Physikalischen Gasphasenabscheidung (PVD, Physical Vapor Deposition), der Chemischen Gasphasenabscheidung (CVD, Chemical Vapor Deposition) und der Atomlagenabscheidung (ALD, Atomic Layer Deposition) stehen Methoden zur Verfügung, welche das Wachstum verschiedener Materialien mit unterschiedlichen Graden der Prozesskontrolle erlauben.

%% {Mein Gebiet}
Atomistische Simulationen können helfen, unter verschiedenen Bedingungen die strukturellen Eigenschaften der abgeschiedenen Schicht zu beschreiben, anhand derer auch Rückschlüsse auf elektronische\cite{aspnes_optical_1982,steudel_influence_2004} oder mechanische Eigenschaften\cite{chasiotis_mechanical_2003,cammarata_nanoindentation_1990} gezogen werden können.
Dafür ist eine Modellierung chemischer Reaktionen und amorpher Strukturen ebenso notwendig wie die Beschreibung von Simulationsräumen auf der Größenordnung kompletter Nano-Bauelemente, um ebenfalls Aussagen über das Wachstum auf strukturierten, polykristallinen oder heterogenen Substraten treffen zu können.
Zur Simulation von Abscheidungen auf kompletten Nano-Bauelementen sind große Simulationsräume im Bereich von \SI{200x200}{\nano\meter} mit bis zu \num{1e9} Atomen notwendig, die mit rein atomistischen Simulationen bisher aufgrund des hohen Rechenaufwandes nicht möglich sind\cite{plimpton_computational_1995}.
% oder keine strukturellen Aussagen ermöglichen.

%% {Möglichkeiten zur Optimierung}
Ansatzpunkte für die Simulation von Gasphasenabscheidungen bestehen in Kinetischen Monte Carlo-Methoden (KMC)\cite{voter_introduction_2007} einerseits und in der Molekulardynamik (MD)\cite{hoover_molecular_1986} andererseits.
KMC kann die Reaktionskinetik komplizierter Prozesse in großen Simulationsräumen zufriedenstellend beschreiben, ermöglicht allerdings keine detaillierte strukturelle Beschreibung der simulierten Materialien.
MD hingegen simuliert effizient atomistische Strukturen, lässt sich aber nur unzureichend zur Beschreibung von großen Simulationsräumen nutzen.
Mit der Einführung reaktiver Kraftfelder haben molekulardynamische Simulationen zudem die Möglichkeit gewonnen, chemische Reaktionen effizient zu simulieren, doch sind diese rechenaufwendiger als herkömmliche Kraftfelder und erlauben nur die Simulation weniger tausend Atome\cite{van_duin_reaxff:_2001}.
Eine mögliche Lösung dieser Probleme stellt eine Kombination der beiden Methoden dar, durch die atomistische Simulationen von vollständigen Beschichtungen kompletter Nano-Bauelemente ermöglicht werden.

%% {Anknüpfungspunkte und Fokus der Arbeit}

Das Ziel der vorliegenden Arbeit ist, ein bestehendes Multiskalen-Modell zur Simulation von Atomlagenabscheidungen per KMC und MD auf allgemeine Gasphasenabscheidungen zu erweitern und die Simulation von Gasphasenabscheidungen, insbesondere der physikalischen Gasphasenabscheidung, damit zu realisieren.
Dabei soll auch die Präparation neuer Abscheidungsprozesse sowie die Möglichkeit untersucht werden, die Oberflächen-Reaktionen bei chemischen Gasphasenabscheidungen durch reaktive Kraftfelder molekulardynamisch zu beschreiben.
Außerdem soll vorliegende Implementierung dieses Modelles zugunsten der schnellen Prozesspräparation um Konfigurationsdateien, zentrale Datenbanken der Substrate und Prozesskonfigurationen sowie um eine einfachere Kontrolle der Prozessparallelisierung ergänzt werden.

%% {Struktur der Arbeit}
\subsubsection{Struktur der Arbeit}

Im Anschluss an diese Einleitung wird zuerst in Kapitel~\ref{theory} ein Überblick über die Arten und Funktionsweisen von Gasphasenabscheidungen, sowie die Simulationsmethoden der Molekulardynamik und Kinetischen Monte Carlo-Methoden gegeben.

Danach stellt Kapitel~\ref{models} den aktuellen Stand der Simulationen von Gasphasenabscheidungen mit KMC und MD vor, präsentiert die Erweiterungen des kombinierten KMC/MD-Modells sowie eine Laufzeitanalyse dieses Modells und gibt einen Überblick über allgemeine molekulardynamische Methoden zur Präparation von Prozessen und Substraten sowie über die Analyse von Ergebnissen.

Der Hauptteil der Arbeit beschäftigt sich in Kapitel~\ref{results} mit den Ergebnissen der Simulation von Gasphasenabscheidungen und den notwendigen Voruntersuchungen der molekulardynamischen Potentialparametrisierungen.
Dafür wird zuerst eine Gold-PVD-Simulation als Beispielprozess für nichtreaktive Abscheidungen präpariert und damit das Wachstumsverhalten auf strukturierten Substraten sowie die Skalierbarkeit des kombinierten Modelles überprüft.
Anhand von Kupfer-PVD werden danach Unterschiede zwischen verschiedenen Parametersätzen des selben Kraftfeldes untersucht und die Bildung und Schließung von Oberflächendefekten beobachtet.
Die Simulation gemischter Systeme wird anhand der Multilagen-Abscheidung durch Kupfer-Nickel-PVD untersucht und die entstandenen Strukturen mit den Ergebnissen einer reinen MD-Simulation verglichen.
Anschließend wird die Präparation und Abscheidung amorpher Strukturen am Silizium-PVD-System untersucht, wobei auch Betrachtungen der Gasphasen-Reaktionen für Siliziumoxid-CVD durchgeführt werden.
Zuletzt wird die reaktive Hydroxylierung einer Aluminiumoxid-Oberfläche mit Wassermolekülen simuliert.

Zum Abschluss wird die Arbeit mit Kapitel~\ref{summary} zusammen gefasst, bevor zukünftige Fragestellungen und Anknüpfungspunkte für weitere Untersuchungen gegeben werden.

Im Anhang der Arbeit sind experimentelle Referenzwerte, weitergehende Informationen zu den verwendeten Datenstrukturen und dem Laufzeitverhalten von Parsivald sowie ergänzende Ergebnisse der simulierten Gasphasenabscheidungen zu finden.
