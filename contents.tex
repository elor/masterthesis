%%======================================================================
%%      Inhaltsverzeichnis
%%======================================================================
%\clearpage

%\cleardoubleemptypage

%\pdfbookmark{Inhaltsverzeichnis}{Inhaltsverzeichnis}
\tableofcontents
\todo[inline]{Überschriften überarbeiten}

%%======================================================================
%%      Abbildungsverzeichnis
%%======================================================================
%\cleardoublepage
\markboth{Abbildungsverzeichnis}{Abbildungsverzeichnis}
\listoffigures

%%======================================================================
%%      Tabellenverzeichnis
%%======================================================================
%\cleardoublepage
\markboth{Tabellenverzeichnis}{Tabellenverzeichnis}
\listoftables

%======================================================================
%  Literaturverzeichnis
%======================================================================
%\cleardoublepage
%\manualmark
%\markboth{Literaturverzeichnis}{Literaturverzeichnis}
%\bibliographystyle{unsrt}
%\bibliographystyle{own}
%\bibliography{literatur}

%%======================================================================
%%      Algorithmenverzeichnis
%%======================================================================
%% \renewcommand{\listalgorithmname}{Algorithmenverzeichnis}
%% \addcontentsline{toc}{chapter}{Algorithmenverzeichnis}
%% \listofalgorithms

%%======================================================================
%%      Abkuerzungsverzeichnis
%%======================================================================
%\cleardoublepage
\chapter*{Abkürzungsverzeichnis}
\addcontentsline{toc}{chapter}{Abkürzungsverzeichnis}
\markboth{Abkürzungsverzeichnis}{Abkürzungsverzeichnis}
\def\listacronymname{Abkürzungsverzeichnis}

\newcolumntype{Y}{>{\flushleft\arraybackslash}X}

%\begin{acronym}[SQL]
% \acro{CNTs}{Kohlenstoffnanoröhrchen (engl. Carbon Nanotubes, CNTs)}
% \acro{DFT}{Dichtefunktionaltheorie}
%\end{acronym}

\def\arraystretch{1.3}
\begin{tabular}{lll}
AFM       & Atomic Force Microscopy             & Rasterkraftmikroskopie                                     \\
ALD       & Atomic Layer Deposition             & Atomlagenabscheidung                                       \\
CG        & Conjugated Gradients                & Konjugierte Gradienten                                     \\
CFD       & Computational Fluid Dynamics        & Eine Methode zur Gasfluss/Simulation                       \\
CVD       & Chemical Vapor Deposition           & Chemische Gasphasenabscheidung                             \\
DFT       & Density Functional Theory           & Dichtefunktionaltheorie                                    \\
EAM       & Embedded Atom Method                & Eingebettete-Atom-Methode                                  \\
GPC       & Growth per Cycle                    & Schichtwachstum pro ALD-Zyklus                             \\
KMC       & Kinetic Monte Carlo                 & Kinetische Monte Carlo-Methoden                            \\
MMC       & Metropolis Monte Carlo              & Metropolis Monte Carlo-Methoden                            \\
MD        & Molecular Dynamics                  & Molekulardynamik                                           \\
MEAM      & Modified Embedded Atom Method       & Modifizierte Eingebettete-Atom-Methode                     \\
MOSFET    & Metal Oxide Field Effect Transistor & Metalloxid-Feldeffekt-Transistor                           \\
MPI       & Message Passing Interface           & Eine Kommunikationsschnittstelle                           \\
PVD       & Physical Vapor Deposition           & Physikalische Gasphasenabscheidung                         \\
RAM       & Random-Access Memory                & Arbeitsspeicher                                            \\
RDF       & Radial Distribution Function        & Radiale Verteilungsfunktion                                \\
RMS       & Root Mean Square                    & Standardabweichung                                         \\
ReaxFF    & Reactive Force Fields               & Reaktive Kraftfelder                                       \\
TMA       & Trimethylaluminum (\ce{Al(CH3)3})   & Trimethylaluminium (\ce{Al(CH3)3})                         \\
NPT       & \multicolumn{2}{l}{Großkanonisches Ensemble: Teilchenzahl, Druck und Temperatur sind konstant}   \\
NVE       & \multicolumn{2}{l}{Mikrokanonisches Ensemble: Teilchenzahl, Volumen und Energie sind konstant}   \\
NVT       & \multicolumn{2}{l}{Kanonisches Ensemble: Teilchenzahl, Volumen und Temperatur sind konstant}     \\
Parsivald & \multicolumn{2}{l}{Parallel Atomistic Reaction Simulator for Vapor and Atomic Layer Depositions} \\
\end{tabular}

\begin{comment}
  Liste der Abkürzungen, die nicht weiter erklärt werden:
  (Comment-Umgebung, damit sie vom Parser trotzdem erfasst werden)

NB        & Neighborhood                        & (atomare) Nachbarschaft                                     \\
CLI       & Command Line Interface              & Kommandozeile                                               \\
FEM       & Finite Element Method               & Finite-Elemente-Methoden                                    \\
  ENAS      & \multicolumn{2}{l}{Fraunhofer-Institut für Elektronische Nano-Systeme}                            \\
  LAMMPS    & \multicolumn{2}{l}{Large-scale Atomic/Molecular Massively Parallel Simulator (MD-Bibliothek)}     \\
  LMP
  GMR       & Giant Magnetoresistance             & Riesenmagnetowiderstand                                     \\
  TMR
  AMBER
  CHARMM
  EON
  LJ
  NIST
  NO
  OH
  GROMACS
  BIOVIA
  TU
  EU
  ACCELERATE
  MOSFETs
\end{comment}

\todo[inline]{Prüfen, ob die Abkürzungen im Text ordentlich eingeführt werden}

\chapter*{Symbolverzeichnis}
\addcontentsline{toc}{chapter}{Symbolverzeichnis}
\markboth{Abkürzungsverzeichnis}{Symbolverzeichnis}
\def\listacronymname{Symbolverzeichnis}

\begin{tabular}{ll}
$\alpha$         & Parameter der Alpha-Form / Schrittweite bei MD-Minimierungen \\
$\Delta t$       & Zeitschrittweite                                             \\
$E$              & Energie                                                      \\
$E_i$            & Ein KMC-Ereignis                                             \\
$F_\alpha$       & Einbettungsfunktion des EAM-Potentiales                      \\
$\kappa$         & Dielektrizitätskonstante eines Materiales                    \\
$k_B$            & Boltzmann-Konstante                                          \\
$m$              & Masse eines Teilchens                                        \\
$M$              & Virtuelle Masse eines Thermostates                           \\
$n_\text{cyc.}$  & Zahl der simulierten KMC-Zyklen                              \\
$r_\text{bond}$  & Bindungslänge                                                \\
$r_c$            & Cutoff-Radius eines Kraftfeldes                              \\
$r_d$            & Umkugel-Radius eines Delaunay-Simplexes                      \\
$r_s$            & Suchradius                                                   \\
$r_{ij}$         & Radialer Abstand zwischen Teilchen $i$ und $j$               \\
$r_i$            & Rate des Ereignisses $E_i$                                   \\
$R_i$            & Akkumulierte Rate des Ereignisses $E_i$                      \\
$R_N$            & Summe der Raten aller möglichen Ereignisse                   \\
$\sigma_z$       & RMS-Rauheit einer Oberfläche                                 \\
$\tau$           & Dämpfungszeit von Thermostaten                               \\
$t_\text{relax}$ & Relaxationszeit einer MD-Simulation                          \\
$u$,~$u'$        & $[0,1)$-gleichverteilte Zufallszahlen                        \\
$\vec F(X)$      & Kraft auf ein Teilchen                                       \\
$\vec p$         & Impulsvektor                                                 \\
$\vec v$         & Geschwindigkeitsvektor                                       \\
$V(X)$           & Potentielle Energie eines Teilchens                          \\
$X$              & Zustand eines Systemes                                       \\
$\mathbb{X}$     & Zustandsraum eines Systems                                   \\
%% $d$              & Anzahl der Dimensionen                                       \\
%% $T$              & Temperatur                                                   \\
%% $V$              & Volumen                                                      \\
%% $p$              & Druck                                                        \\
%% $\nabla$         & Nabla-Operator                                               \\
\end{tabular}

%\printglosstex(acr)

%%======================================================================
%%      Ende
%%======================================================================
