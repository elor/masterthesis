%%======================================================================
%%      Inhaltsverzeichnis
%%======================================================================
%\clearpage

%\cleardoubleemptypage

%\pdfbookmark{Inhaltsverzeichnis}{Inhaltsverzeichnis}
\tableofcontents

%%======================================================================
%%      Abbildungsverzeichnis
%%======================================================================
%\cleardoublepage
\markboth{Abbildungsverzeichnis}{Abbildungsverzeichnis}
\listoffigures
\todo[inline]{Beschreibungen der Abbildungen überarbeiten}

%%======================================================================
%%      Tabellenverzeichnis
%%======================================================================
%\cleardoublepage
\markboth{Tabellenverzeichnis}{Tabellenverzeichnis}
\listoftables
\todo[inline]{Beschreibungen der Tabellen überarbeiten}

%======================================================================
%  Literaturverzeichnis
%======================================================================
%\cleardoublepage
%\manualmark
%\markboth{Literaturverzeichnis}{Literaturverzeichnis}
%\bibliographystyle{unsrt}
%\bibliographystyle{own}
%\bibliography{literatur}

%%======================================================================
%%      Algorithmenverzeichnis
%%======================================================================
%% \renewcommand{\listalgorithmname}{Algorithmenverzeichnis}
%% \addcontentsline{toc}{chapter}{Algorithmenverzeichnis}
%% \listofalgorithms

%%======================================================================
%%      Abkuerzungsverzeichnis
%%======================================================================
%\cleardoublepage
\chapter*{Abkürzungsverzeichnis}
\addcontentsline{toc}{chapter}{Abkürzungsverzeichnis}
\markboth{Abkürzungsverzeichnis}{Abkürzungsverzeichnis}
\def\listacronymname{Abkürzungsverzeichnis}

\newcolumntype{Y}{>{\flushleft\arraybackslash}X}

%\begin{acronym}[SQL]
% \acro{CNTs}{Kohlenstoffnanoröhrchen (engl. Carbon Nanotubes, CNTs)}
% \acro{DFT}{Dichtefunktionaltheorie}
%\end{acronym}

\def\arraystretch{1.5}
\begin{tabularx}{\linewidth}{lll}

ALD    & Atomic Layer Deposition             & Atomlagenabscheidung                   \\
AFM    & Atomic Force Microscopy             & Rasterkraftmikroskopie                 \\
CG     & Conjugated Gradients                & Konjugierte Gradienten                 \\
CLI    & Command Line Interface              & Kommandozeile                          \\
CVD    & Chemical Vapor Deposition           & Chemische Gasphasenabscheidung         \\
DFT    & Density Functional Theory           & Dichtefunktionaltheorie                \\
EAM    & Embedded Atom Method                & Eingebettete-Atom-Methode              \\
FEM    & Finite Element Method               & Finite-Elemente-Methoden               \\
KMC    & Kinetic Monte Carlo                 & Kinetische Monte Carlo-Methoden        \\
MD     & Molecular Dynamics                  & Molekulardynamik                       \\
MEAM   & Modified Embedded Atom Method       & Modifizierte Eingebettete-Atom-Methode \\
MOSFET & Metal Oxide Field Effect Transistor & Metalloxid-Feldeffekt-Transistor       \\
MPI    & Message Passing Interface           & eine Kommunikationsschnittstelle       \\
PVD    & Physical Vapor Deposition           & Physikalische Gasphasenabscheidung     \\
RDF    & Radial Distribution Function        & Radiale Verteilungsfunktion            \\
ReaxFF & Reactive Force Fields               & Reaktive Kraftfelder                   \\
RMS    & Root Mean Square                    & Standardabweichung                     \\
TMA    & Trimethylaluminum (\ce{Al(CH3)3})   & Trimethylaluminium (\ce{Al(CH3)3})     \\

NPT        &   \multicolumn{2}{X}{Großkanonisches Ensemble: Teilchenzahl, Druck und Temperatur sind konstant}   \\
NVT        &   \multicolumn{2}{X}{Kanonisches Ensemble: Teilchenzahl, Volumen und Temperatur sind konstant}     \\
NVE        &   \multicolumn{2}{X}{Mikrokanonisches Ensemble: Teilchenzahl, Volumen und Energie sind konstant}   \\
Parsivald  &   \multicolumn{2}{X}{Parallel Atomistic Reaction Simulator for Vapor and Atomic Layer Depositions} \\

\end{tabularx}

\todo[inline]{Scan auf neue Abkürzungen}
\todo[inline]{Prüfen, ob die Abkürzungen im Text ordentlich eingeführt werden}

%\begin{longtable}{lll}
%
%MOSFET & Metall-Oxid-Halbleiter-Feldeffekttransistor & metal-oxid-semiconductor field-effect transistor \\
%
%\end{longtable}


\chapter*{Symbolverzeichnis}
\addcontentsline{toc}{chapter}{Symbolverzeichnis}
\markboth{Abkürzungsverzeichnis}{Symbolverzeichnis}
\def\listacronymname{Symbolverzeichnis}

\begin{tabularx}{\linewidth}{ll}
$E_\text{A}$       & Aktivierungsenergie            \\
$p_\text{partial}$ & Partialdruck                   \\
$\sigma_z$         & RMS-Rauheit einer Oberfläche   \\
$\tau$             & Dämpfungszeit des Thermostates \\
\end{tabularx}
\todo[inline]{die wichtigsten Symbole automatisch aus allen Formeln exportieren}

%\printglosstex(acr)

%%======================================================================
%%      Ende
%%======================================================================
\cleardoublepage
\pagenumbering{arabic} % Fäng erneut bei 1 an.
