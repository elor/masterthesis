\documentclass{scrreprt}

\usepackage{amsmath}
\usepackage{amsfonts}
\usepackage[range-phrase={\,bis\,},binary-units,retain-explicit-plus]{siunitx}

\begin{document}

\begin{equation}
k\approx 8
\end{equation}
\begin{equation}
k=3.9
\end{equation}
\begin{equation}
\gamma
\end{equation}
\begin{equation}
\alpha
\end{equation}
\begin{equation}
\alpha
\end{equation}
\begin{equation}
\alpha
\end{equation}
\begin{equation}
\alpha
\end{equation}
\begin{equation}
\alpha
\end{equation}
\begin{equation}
\gamma
\end{equation}
\begin{equation}
\alpha
\end{equation}
\begin{equation}
\gamma
\end{equation}
\begin{equation}
\alpha
\end{equation}
\begin{equation}
\alpha
\end{equation}
\begin{equation}
\alpha
\end{equation}
\begin{equation}
n
\end{equation}
\begin{equation}
k
\end{equation}
\begin{equation}
k
\end{equation}
\begin{equation}
k
\end{equation}
\begin{equation}
k+1
\end{equation}
\begin{equation}
\subset
\end{equation}
\begin{equation}
\Leftrightarrow
\end{equation}
\begin{equation}
\subset
\end{equation}
\begin{equation}
\Rightarrow
\end{equation}
\begin{equation}
k_d
\end{equation}
\begin{equation}
E_\text{A}
\end{equation}
\begin{equation}
p_\text{partial}
\end{equation}
\begin{equation}
\sigma_z
\end{equation}
\begin{equation}
\tau
\end{equation}
\begin{equation}
t_\text{relax}
\end{equation}
\begin{equation}
t=59
\end{equation}
\begin{equation}
t=65
\end{equation}
\begin{equation}
t=75
\end{equation}
\begin{equation}
t=85
\end{equation}
\begin{equation}
t=95
\end{equation}
\begin{equation}
t=100
\end{equation}
\begin{equation}
n
\end{equation}
\begin{equation}
1
\end{equation}
\begin{equation}
1
\end{equation}
\begin{equation}
1
\end{equation}
\begin{equation}
n
\end{equation}
\begin{equation}
n
\end{equation}
\begin{equation}
n
\end{equation}
\begin{equation}
n
\end{equation}
\begin{equation}
n\log{n}
\end{equation}
\begin{equation}
n
\end{equation}
\begin{equation}
n
\end{equation}
\begin{equation}
n
\end{equation}
\begin{equation}
n
\end{equation}
\begin{equation}
1
\end{equation}
\begin{equation}
n
\end{equation}
\begin{equation}
\frac{r_c^3}{s^3}n^2
\end{equation}
\begin{equation}
n
\end{equation}
\begin{equation}
1
\end{equation}
\begin{equation}
1
\end{equation}
\begin{equation}
1
\end{equation}
\begin{equation}
r_s^3
\end{equation}
\begin{equation}
r_s^3
\end{equation}
\begin{equation}
c
\end{equation}
\begin{equation}
n+c
\end{equation}
\begin{equation}
n\log{c}
\end{equation}
\begin{equation}
\log{c}
\end{equation}
\begin{equation}
\log{c}
\end{equation}
\begin{equation}
1
\end{equation}
\begin{equation}
r_s^3\log{c}
\end{equation}
\begin{equation}
r_s^3\log{c}
\end{equation}
\begin{equation}
\log{c}
\end{equation}
\begin{equation}
n+c^\frac{2}{3}
\end{equation}
\begin{equation}
n\log{n}
\end{equation}
\begin{equation}
\log{n}
\end{equation}
\begin{equation}
\log{n}
\end{equation}
\begin{equation}
\log{n}
\end{equation}
\begin{equation}
r_s^3\log{n}
\end{equation}
\begin{equation}
r_s^3\log{n}
\end{equation}
\begin{equation}
\log{n}
\end{equation}
\begin{equation}
n
\end{equation}
\begin{equation}
n\log{n}
\end{equation}
\begin{equation}
k\log{k}
\end{equation}
\begin{equation}
k\log{k}
\end{equation}
\begin{equation}
k\log{k}
\end{equation}
\begin{equation}
r_s^3+n^\frac{1}{3}
\end{equation}
\begin{equation}
r_s^3
\end{equation}
\begin{equation}
1
\end{equation}
\begin{equation}
nk
\end{equation}
\begin{equation}
n
\end{equation}
\begin{equation}
1
\end{equation}
\begin{equation}
1
\end{equation}
\begin{equation}
1
\end{equation}
\begin{equation}
n
\end{equation}
\begin{equation}
n
\end{equation}
\begin{equation}
n
\end{equation}
\begin{equation}
n
\end{equation}
\begin{equation}
n
\end{equation}
\begin{equation}
k
\end{equation}
\begin{equation}
b
\end{equation}
\begin{equation}
k_r
\end{equation}
\begin{equation}
d \leq r_c
\end{equation}
\begin{equation}
r_c
\end{equation}
\begin{equation}
r_s
\end{equation}
\begin{equation}
s
\end{equation}
\begin{equation}
\frac{\log{c}}{d\log{2}}
\end{equation}
\begin{equation}
atoms
\end{equation}
\begin{equation}
size[3]
\end{equation}
\begin{equation}
depth
\end{equation}
\begin{equation}
atoms, spacesize, depth
\end{equation}
\begin{equation}
cellsize[0] \gets spacesize[0]\cdot2^{-depth}
\end{equation}
\begin{equation}
cellsize[1] \gets spacesize[1]\cdot2^{-depth}
\end{equation}
\begin{equation}
cellsize[2] \gets spacesize[2]\cdot2^{-depth}
\end{equation}
\begin{equation}
root \gets
\end{equation}
\begin{equation}
depth
\end{equation}
\begin{equation}
atom
\end{equation}
\begin{equation}
atoms
\end{equation}
\begin{equation}
cellindex[0] \gets \lfloor atom.pos[0] / cellsize[0] \rfloor
\end{equation}
\begin{equation}
cellindex[1] \gets \lfloor atom.pos[1] / cellsize[1] \rfloor
\end{equation}
\begin{equation}
cellindex[2] \gets \lfloor atom.pos[2] / cellsize[2] \rfloor
\end{equation}
\begin{equation}
cell \gets
\end{equation}
\begin{equation}
root, cellindex
\end{equation}
\begin{equation}
root
\end{equation}
\begin{equation}
root
\end{equation}
\begin{equation}
i[3]
\end{equation}
\begin{equation}
allocate
\end{equation}
\begin{equation}
cell, id, allocate
\end{equation}
\begin{equation}
d \gets 
\end{equation}
\begin{equation}
d = 0
\end{equation}
\begin{equation}
cell.children
\end{equation}
\begin{equation}
cell.children \gets 
\end{equation}
\begin{equation}
childid \gets 
\end{equation}
\begin{equation}
2^{d-1}
\end{equation}
\begin{equation}
2\cdot
\end{equation}
\begin{equation}
2^{d-1}
\end{equation}
\begin{equation}
4\cdot
\end{equation}
\begin{equation}
2^{d-1}
\end{equation}
\begin{equation}
cell.children[childid], i, allocate
\end{equation}
\begin{equation}
=
\end{equation}
\begin{equation}
N
\end{equation}
\begin{equation}
points
\end{equation}
\begin{equation}
k
\end{equation}
\begin{equation}
points, dim\gets0
\end{equation}
\begin{equation}
n\gets
\end{equation}
\begin{equation}
n=0
\end{equation}
\begin{equation}
points, dim
\end{equation}
\begin{equation}
points
\end{equation}
\begin{equation}
dim
\end{equation}
\begin{equation}
root\gets{}points\left[\lfloor\frac{n}{2}\rfloor\right]
\end{equation}
\begin{equation}
dim\gets(dim+1)\mod{k}
\end{equation}
\begin{equation}
root.left \gets
\end{equation}
\begin{equation}
points\left[0:\lfloor\frac{n}{2}\rfloor-1\right], dim
\end{equation}
\begin{equation}
root.right \gets
\end{equation}
\begin{equation}
points\left[\lfloor\frac{n}{2}\rfloor+1:n-1\right], dim
\end{equation}
\begin{equation}
root
\end{equation}
\begin{equation}
k
\end{equation}
\begin{equation}
k
\end{equation}
\begin{equation}
k+1
\end{equation}
\begin{equation}
r_d
\end{equation}
\begin{equation}
\alpha
\end{equation}
\begin{equation}
\alpha
\end{equation}
\begin{equation}
\alpha \rightarrow \infty
\end{equation}
\begin{equation}
\alpha \rightarrow 0
\end{equation}
\begin{equation}
\alpha \approx r_\text{bond}
\end{equation}
\begin{equation}
_0
\end{equation}
\begin{equation}
_0 \in
\end{equation}
\begin{equation}
\neq \emptyset
\end{equation}
\begin{equation}
\in
\end{equation}
\begin{equation}
\setminus
\end{equation}
\begin{equation}
\in
\end{equation}
\begin{equation}
\cap
\end{equation}
\begin{equation}
\cap
\end{equation}
\begin{equation}
\setminus
\end{equation}
\begin{equation}
r_d > \alpha
\end{equation}
\begin{equation}
r_d < \alpha
\end{equation}
\begin{equation}
\sim t
\end{equation}
\begin{equation}
\sim t
\end{equation}
\begin{equation}
\sim n_\text{cyc.}
\end{equation}
\begin{equation}
\vec F(X)
\end{equation}
\begin{equation}
V(X)
\end{equation}
\begin{equation}
V
\end{equation}
\begin{equation}
r_ij
\end{equation}
\begin{equation}
  \vec F_{ij}(r_{ij}) = \vec\nabla V(r_{ij})
\end{equation}
\begin{equation}
  E = \sum_i\sum_{j \neq i}{V(r_{ij})}
\end{equation}
\begin{equation}
  V_\text{LJ}(r_{ij}) = 4 \epsilon \left[\left(\frac{\sigma}{r_{ij}}\right)^{12} - \left(\frac{\sigma}{r_{ij}}\right)^{6}\right]
\end{equation}
\begin{equation}
V(r_ij)
\end{equation}
\begin{equation}
r_\text{cut}
\end{equation}
\begin{equation}
  E = \sum_i\sum_{j \neq i}{V_2\left(r_{ij}\right)} + \sum_i\sum_{j \neq i}\sum_{\substack{k \neq i \\ k \neq j}}{V_3\left(r_{ij}, r_{ik}, \theta_{ijk}\right)} + \dots
\end{equation}
\begin{equation}
V_{\alpha\beta}(r_{ij})
\end{equation}
\begin{equation}
i
\end{equation}
\begin{equation}
F_\alpha
\end{equation}
\begin{equation}
\rho_\beta(r_{ij})
\end{equation}
\begin{equation}
  E = \sum_i\left[F_\alpha\left(\sum_{j\neq i}{\rho_\beta\left(r_{ij}\right)}\right) + \frac{1}{2}\sum_{j\neq i}{V_{\alpha\beta}\left(r_{ij}\right)}\right]
\end{equation}
\begin{equation}
\alpha
\end{equation}
\begin{equation}
\beta
\end{equation}
%% \begin{equation}
%%   E = \sum_i\left[F_\alpha\left(\bar{\rho_i}\right) + \frac{1}{2}\sum_{j\neq i}{V_{ij}\left(r_{ij}\right)}\right]
%% \end{equation}
\begin{equation}
E_\text{system}
\end{equation}
\begin{align}
  \label{reaxformulation}
  E_\text{system} =~& E_\text{bond} + E_\text{lp} + E_\text{over} + E_\text{under} + E_\text{val} + E_\text{pen} + E_\text{coa} + E_\text{C2} \\
  \nonumber  & + E_\text{tors} + E_\text{conj} + E_\text{H-bond} + E_\text{vdWaals} + E_\text{Coulomb}
\end{align}
\begin{equation}
\sigma
\end{equation}
\begin{equation}
\pi
\end{equation}
\begin{equation}
\pi
\end{equation}
\begin{equation}
E_\text{bond}
\end{equation}
\begin{equation}
E_\text{lp}
\end{equation}
\begin{equation}
E_\text{over}
\end{equation}
\begin{equation}
E_\text{under}
\end{equation}
\begin{equation}
\pi
\end{equation}
\begin{equation}
E_\text{val}
\end{equation}
\begin{equation}
E_\text{pen}
\end{equation}
\begin{equation}
E_\text{coa}
\end{equation}
\begin{equation}
_2
\end{equation}
\begin{equation}
E_\text{C2}
\end{equation}
\begin{equation}
_2
\end{equation}
\begin{equation}
E_\text{tors}
\end{equation}
\begin{equation}
E_\text{conj}
\end{equation}
\begin{equation}
E_\text{H-bond}
\end{equation}
\begin{equation}
E_\text{vdWaals}
\end{equation}
\begin{equation}
E_\text{Coulomb}
\end{equation}
\begin{equation}
t_\text{relax}
\end{equation}
\begin{equation}
\tau
\end{equation}
\begin{equation}
t_\text{relax}
\end{equation}
\begin{equation}
\tau
\end{equation}
\begin{equation}
 t_\text{relax}=\SI{50}{\pico\second}
\end{equation}
\begin{equation}
\tau=\SI{0.02}{\femto\second}
\end{equation}
\begin{equation}
\Downarrow
\end{equation}
\begin{equation}
\sigma_z = \SI{1.2}{\angstrom}
\end{equation}
\begin{equation}
\sigma_z = \SI{6.4}{\angstrom}
\end{equation}
\begin{equation}
\sigma_z = \SI{8.0}{\angstrom}
\end{equation}
\begin{equation}
\hat{=}
\end{equation}
\begin{equation}
\kappa=\num{3.9}
\end{equation}
\begin{equation}
\kappa
\end{equation}
\begin{equation}
\kappa
\end{equation}
\begin{equation}
\kappa
\end{equation}
\begin{equation}
X_n
\end{equation}
\begin{equation}
X_m
\end{equation}
\begin{equation}
r_{n m}
\end{equation}
\begin{equation}
  \frac{d \rho(X_m,t)}{d t} = \sum_n{r_{n m}\rho(X_n,t) - \sum_n{r_{m n}\rho(X_m,t)}}
\end{equation}
\begin{equation}
X_0
\end{equation}
\begin{equation}
\mathbb{X}
\end{equation}
\begin{equation}
t_0 = 0
\end{equation}
\begin{equation}
X_n \in \mathbb{X}
\end{equation}
\begin{equation}
E_i
\end{equation}
\begin{equation}
X_{n-1}
\end{equation}
\begin{equation}
X_n^i
\end{equation}
\begin{equation}
  E_i : X_{n-1} \rightarrow X_n^i \in \mathbb{X} ~~,\quad i \in [1, N]
\end{equation}
\begin{equation}
E_i
\end{equation}
\begin{equation}
r_i
\end{equation}
\begin{equation}
R_i
\end{equation}
\begin{equation}
R_N
\end{equation}
\begin{equation}
  R_i = \sum_{j \le i}{r_j}
\end{equation}
\begin{equation}
u
\end{equation}
\begin{equation}
X_n^i
\end{equation}
\begin{equation}
X_n
\end{equation}
\begin{equation}
  X_n = X_n^i : R_{i-1} \le u R_N < R_i ~~,\quad u \in [0,1)~\text{gleichverteilt}
\end{equation}
\begin{equation}
R_N
\end{equation}
\begin{equation}
  t_n = t_{n-1} + \frac{-\ln(u')}{R_N} ~~,\quad u' \in [0,1)~\text{gleichverteilt}
\end{equation}
\begin{equation}
X_{n+1}
\end{equation}
\begin{equation}
N=0
\end{equation}
\begin{equation}
R_N=0
\end{equation}
\begin{equation}
t_n > t_\text{fin}
\end{equation}
\begin{equation}
E_i
\end{equation}
\begin{equation}
n
\end{equation}
\begin{equation}
X_n
\end{equation}
\begin{equation}
\alpha
\end{equation}
\begin{equation}
\tau
\end{equation}
\begin{equation}
t_\text{relax}
\end{equation}
\begin{equation}
\tau_p
\end{equation}
\begin{equation}
R
\end{equation}
\begin{equation}
R
\end{equation}
\begin{equation}
m
\end{equation}
\begin{equation}
\vec r
\end{equation}
\begin{equation}
\vec p
\end{equation}
\begin{equation}
\vec{F}(R)
\end{equation}
\begin{equation}
N
\end{equation}
\begin{equation}
V
\end{equation}
\begin{equation}
E
\end{equation}
\begin{equation}
  N = \text{const.}
  \qquad
  V = \text{const.}
  \qquad
  E = \text{const.}
\end{equation}
\begin{equation}
i
\end{equation}
\begin{equation}
  \dot{\vec r_i} = \frac{\vec p_i}{m_i}
\end{equation}
\begin{equation}
  \dot{\vec p_i} = m \vec a_i = \vec F_i(R)
\end{equation}
\begin{equation}
T
\end{equation}
\begin{equation}
  N = \text{const.}
  \qquad
  V = \text{const.}
  \qquad
  T = \text{const.}
\end{equation}
\begin{equation}
T_\text{Ziel}
\end{equation}
\begin{equation}
  \overline{E_{kin}} = \frac{1}{2} \overline{m v^2} = \frac{d}{2} k_B T_\text{Ziel}
\end{equation}
\begin{equation}
T_\text{Ziel}
\end{equation}
\begin{equation}
\tau
\end{equation}
\begin{equation}
  \vec v_i' = \vec v_i \cdot \sqrt{1 + \frac{\Delta t}{\tau} \left(\frac{T_\text{Ziel}}{T} - 1\right)}
\end{equation}
\begin{equation}
s
\end{equation}
\begin{equation}
  \dot{\vec p_i} = \vec{F_i} - s \vec{p_i}
\end{equation}
\begin{equation}
s
\end{equation}
\begin{equation}
\tau
\end{equation}
\begin{equation}
M
\end{equation}
\begin{equation}
M
\end{equation}
\begin{equation}
\tau
\end{equation}
\begin{equation}
  \dot s = \frac{1}{\tau M} \left(\sum_i{\frac{p_i^2}{2m_i}} - N d k_B T\right)
\end{equation}
\begin{equation}
  N = \text{const.}
  \qquad
  p = \text{const.}
  \qquad
  T = \text{const.}
\end{equation}
\begin{equation}
  P V = N k_B T + \frac{1}{d} \sum_{i=1}^N{\vec{r}_i \cdot \vec{F}_i}
\end{equation}
\begin{equation}
\vec X_0
\end{equation}
\begin{equation}
\nabla E(X)
\end{equation}
\begin{equation}
\alpha
\end{equation}
\begin{equation}
\Rightarrow
\end{equation}
\begin{equation}
\Rightarrow
\end{equation}
\begin{equation}
  \vec X_i = \vec X_{i-1} - \alpha \nabla E(\vec X_{i-1})
\end{equation}
\begin{equation}
|\vec X_i - \vec X_{i-1}| < X_\text{tol}
\end{equation}
\begin{equation}
\max_k{|X_{i,k} - X_{i-1,k}|} < X_\text{tol}
\end{equation}
\begin{equation}
|\nabla E(\vec X_{i-1})| < E_\text{tol}
\end{equation}
\begin{equation}
i > i_\text{tol}
\end{equation}
\begin{gather}
  \min_\alpha f(X_i+\alpha \vec s_i) \rightarrow \alpha_i \\
  \vec X_i = \vec X_{i-1} - \alpha_i \vec s_i
\end{gather}
\begin{equation}
  \vec s_i = \Delta \vec X_i + \beta_i \vec s_{i-1}
\end{equation}
\begin{equation}
  \beta_i = \max \left(0, \frac{\Delta \vec X_i \cdot \left(\Delta \vec X_i - \Delta \vec X_{i-1}\right)}{\Delta \vec X_{i-1} \cdot \Delta \vec X_{i-1}}\right) \text{~(Polak-Ribière)}
\end{equation}
\begin{equation}
Id
\end{equation}
\begin{equation}
_2
\end{equation}
\begin{equation}
t_\text{relax}
\end{equation}
\begin{equation}
\end{equation}
\begin{equation}
t=80
\end{equation}
\begin{equation}
\kappa
\end{equation}

\end{document}

