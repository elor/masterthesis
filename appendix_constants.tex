\chapter{Physikalische Konstanten und Stoffeigenschaften}
\label{appendix_constants}

\footnotetext[1]{Die Bindungslängen wurden über die Kristallstrukturen aus den Gitterkonstanten berechnet}

\begin{table}[H]
  \centering
  \caption{Physikalische Konstanten}
  \oddrowcolors
  \begin{tabular}{|lll|}
    \hline
    \textbf{Größe}                  & \textbf{Wert}                           & \textbf{Referenz}               \\
    \hline
    Avogadro-Konstante $N_\text{A}$ & \SI{6.02214179}{\per\mole}              & \cite{haynes_crc_2011} S.1-1    \\
    Boltzmann-Konstante $k_B$       & \SI{1.3806504e-23}{\joule\per\kelvin}   & \cite{haynes_crc_2011} S.1-2    \\
    Molare Gaskonstante $R$         & \SI{8.31472e23}{\joule\per\mole\kelvin} & \cite{haynes_crc_2011} S.1-2    \\
    Atomare Masseneinheit $u$       & \SI{1.660653872e-27}{\kilo\gram}        & \cite{haynes_crc_2011} S.1-2    \\
    \hline
  \end{tabular}
\end{table}

\begin{table}[H]
  \centering
  \caption{Eigenschaften von Gold}
  \oddrowcolors
  \begin{tabular}{|lll|}
    \hline
    \textbf{Größe}                           & \textbf{Wert}                                  & \textbf{Referenz}               \\
    \hline
    Dichte $\rho$, fest                      & \SI{19.3}{\gpcc}                               & \cite{haynes_crc_2011} S.4-65   \\
    Dichte $\rho_m$, flüssig                 & \SI{17.31}{\gpcc}                              & \cite{haynes_crc_2011} S.4-128  \\
    Schmelztemperatur $T_m$                  & \SI{1064.18}{\celsius} (\SI{1337.33}{\kelvin}) & \cite{haynes_crc_2011} S.4-65   \\
    Atomgewicht $u$                          & \SI{196.967}{\gram\per\mole}                   & \cite{haynes_crc_2011} S.4-65   \\
    Kristallstruktur                         & fcc                                            & \cite{haynes_crc_2011} S.4-147  \\
    Gitterkonstante $a$                      & \SI{4.0786}{\angstrom}                         & \cite{haynes_crc_2011} S.4-147  \\
    Bindungslänge $r_\text{bond}$            & \SI{2.8840}{\angstrom}                         & berechnet\footnotemark[1]       \\
    Linearer Ausdehnungskoeffizient $\alpha$ & \SI{14.2e-6}{\per\kelvin}                      & \cite{haynes_crc_2011} S.12-206 \\
    \hline
  \end{tabular}
\end{table}

\begin{table}[H]
  \centering
  \caption{Eigenschaften von Kupfer}
  \oddrowcolors
  \begin{tabular}{|lll|}
    \hline
    \textbf{Größe}                           & \textbf{Wert}                                  & \textbf{Referenz}               \\
    \hline
    Dichte $\rho$, fest                      & \SI{8.96}{\gpcc}                               & \cite{haynes_crc_2011} S.4-61   \\
    Dichte $\rho_m$, flüssig                 & \SI{7.997}{\gpcc}                              & \cite{haynes_crc_2011} S.4-128  \\
    Schmelztemperatur $T_m$                  & \SI{1084.62}{\celsius} (\SI{1357.77}{\kelvin}) & \cite{haynes_crc_2011} S.4-61   \\
    Atomgewicht $u$                          & \SI{63.546}{\gram\per\mole}                    & \cite{haynes_crc_2011} S.4-61   \\
    Kristallstruktur                         & fcc                                            & \cite{haynes_crc_2011} S.4-146  \\
    Gitterkonstante $a$                      & \SI{3.6150}{\angstrom}                         & \cite{haynes_crc_2011} S.4-146  \\
    Bindungslänge $r_\text{bond}$            & \SI{2.5562}{\angstrom}                         & berechnet\footnotemark[1]       \\
    Linearer Ausdehnungskoeffizient $\alpha$ & \SI{16.5e-6}{\per\kelvin}                      & \cite{haynes_crc_2011} S.12-206 \\
    \hline
  \end{tabular}
\end{table}

\begin{table}[H]
  \centering
  \caption{Eigenschaften von Nickel}
  \oddrowcolors
  \begin{tabular}{|lll|}
    \hline
    \textbf{Größe}      & \textbf{Wert}          & \textbf{Referenz}              \\
    \hline
    Kristallstruktur    & fcc                    & \cite{haynes_crc_2011} S.4-150 \\
    Gitterkonstante $a$ & \SI{3.5238}{\angstrom} & \cite{haynes_crc_2011} S.4-150 \\
    \hline
  \end{tabular}
\end{table}

\begin{table}[H]
  \centering
  \caption{Eigenschaften von Silizium}
  \oddrowcolors
  \begin{tabular}{|lll|}
    \hline
    \textbf{Größe}                & \textbf{Wert}                               & \textbf{Referenz}              \\
    \hline
    Dichte $\rho$, kristallin     & \SI{2.3296}{\gpcc}                          & \cite{haynes_crc_2011} S.4-87  \\
    Dichte $\rho$, amorpher Film  & \SI{2.29}{\gpcc}                            & \cite{remes_optical_1998}     \\
                                  & (\SI{2.2}{\gpcc} - \SI{2.24}{\gpcc})        & \cite{renner_density_1973}     \\
    Schmelztemperatur $T_m$       & \SI{1414}{\celsius} (\SI{1687.15}{\kelvin}) & \cite{haynes_crc_2011} S.4-87  \\
    Atomgewicht $u$               & \SI{28.086}{\gram\per\mole}                 & \cite{haynes_crc_2011} S.4-87  \\
    Kristallstruktur              & diamant                                     & \cite{haynes_crc_2011} S.4-151 \\
    Gitterkonstante $a$           & \SI{5.4305}{\angstrom}                      & \cite{haynes_crc_2011} S.4-151 \\
    Bindungslänge $r_\text{bond}$ & \SI{2.3515}{\angstrom}                      & berechnet\footnotemark[1]      \\
    \hline
  \end{tabular}

\end{table}

\footnotetext[1]{Die Bindungslängen wurden über die Kristallstrukturen aus den Gitterkonstanten berechnet}
