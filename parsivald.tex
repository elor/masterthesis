\section{Parsivald-Modell}
\label{parsivald}

Parsivald (Parallel Atomistic Reaction Simulator for Vapor and Atomic Layer Depositions) entstand 2012 als namenloses Resultat meiner Bachelorarbeit\cite{lorenz_entwicklung_2012} am Fraunhofer ENAS mit dem Ziel der Simulation von Atom\-lagen\-abscheidungs-Prozessen.
Das Programm war beschränkt auf die atomistische Simulation von Metall\-oxid-ALD mittels MEAM-Potentialen, die in dieser Arbeit auf PVD und CVD mit beliebigen Potentialen erweitert wurde.

\subsection{Zielsetzung für Parsivald}

Das vorgestellte Parsivald-Modell vereint die Stärken von MD- und KMC-Methoden, um so Gasphasenabscheidungen in großen Simulationsräumen mit atomistischer Genauigkeit zu simulieren.
Ein wesentliches Ziel ist, die strukturellen Eigenschaften der abgeschiedenen Schichten auch für große Oberflächen in effizienten Rechnungen gut wiederzugeben.
Dabei soll nicht nur die Untersuchung flächiger Abscheidungen sondern auch die Untersuchung des Wachstums an beziehungsweise in dreidimensionalen Nanostrukturen, wie etwa Stufen, Gräben und Poren ermöglicht werden.
Dazu müssen relativ große Strukturen von \SI{10x10}{\nano\meter} bis \SI{1x1}{\micro\meter} und Schichtdicken bis zu \SI{20}{\nano\meter} (\num{200} ALD-Zyklen) in akzeptabler Rechenzeit (einige Tage bis wenige Wochen) simuliert werden können.
Die Simulationsräume enthalten dann bis zu \num{1e9} Atome, liegen also bis zu \num{5} Größenordnungen über den Möglichkeiten reiner MD-Simulationen.
Durch die Nutzung massiver Parallelisierung werden zudem atomistische Simulationen über eine Simulationszeit von mehreren Minuten ermöglicht.

\subsection{Beschreibung}

\begin{figure}[b]
  \centering
  \def\svgwidth{\textwidth}
  \input{img/parsivald-schema-flat.pdf_tex}
  \caption{Simulation einer Adsorption in Parsivald, verteilt auf KMC und MD}
  \label{fig:parsivald-schema}
\end{figure}

Der Grundgedanke von Parsivald besteht in der Aufteilung der Simulation in mehrere Skalen.
So ergeben sich zwei zeitliche Skalen durch die hohe Geschwindigkeit der Adsorptionen im Kontrast zu der niedrigen Frequenz, in der diese in direkter Nachbarschaft auftreten.
Einzelne Adsorptionen werden durch KMC-Ereignisse dargestellt, in welchen die Nachbarschaft des Adsorptionsortes (MD-Box) aus der globalen Struktur extrahiert, einem MD-Prozess zur Berechnung übergeben und anschließend wieder in die globale Struktur zurückgeführt wird (Abbildung~\ref{fig:parsivald-schema}).
Die MD-Simulation führt in der Regel eine Relaxierung der Oberfläche im kanonischen Ensemble durch, doch sind auch Strukturoptimierungen möglich.

\begin{figure}
  \centering
  \def\svgwidth{\textwidth}
  \input{img/parsivald-stephierarchy.pdf_tex}
  \caption[Funktionsweise des Parsivald-Programmes]{
    Funktionsweise des Parsivald-Programmes.
    Event-Typen unterscheiden sich durch die MD-Simulation (Relaxation, Reaktion) oder das Precursorgas.\\
    Siehe auch Abbildung~\ref{fig:parsivald-modes}.
  }
  \label{fig:parsivald-stephierarchy}
\end{figure}

Abbildung~\ref{fig:parsivald-stephierarchy} stellt die Funktionsweise einer gesamten Parsivald-Simulation vor, die im Wesentlichen aus Vor- und Nachbereitung sowie einer Prozessschleife besteht, in der nacheinander komplette KMC-Simulationen (Parsivald-Schritte) durchgeführt werden.
Innerhalb der Parsivald-Schritte werden mögliche Adsorptionen, Ligandenaustausch-Reaktionen oder Relaxationen als KMC-Ereignisse verwaltet, welche auf eine dünne Oberflächenschicht begrenzt sind.
Nach jedem Ereignis wird eine rechenaufwendige Konnektivitätsprüfung an den Ereignisorten durchgeführt, um desorbierte Atome zu erkennen und von der Oberfläche auszuschließen.
Damit werden Diffusionen in das Material oder auf seiner Oberfläche unterbunden, dafür jedoch eine effiziente Host-Worker-Parallelisierung ermöglicht.
%%Ereignisse mit überlappenden MD-Boxen werden in Warteschlangen verwaltet, um die Reihenfolge der KMC-Ereignisse zu bewahren.

Als Eingaben der Simulation dienen das Substrat, eine Sammlung verschiedener Prozessparameter wie Prozesstemperaturen, Expositionszeiten und Abscheidungsmodi (Abbildung~\ref{fig:parsivald-modes}), MD-Befehlslisten mit Platzhaltern (MD-Masken) sowie die Potentialparametrisierung.
In die MD-Masken werden im Verlauf der Simulation die Eigenschaften der Ereignisse eingetragen, bevor sie als Steuerbefehle an die MD-Simulationen übergeben werden.
Die Ausgabe erfolgt kontinuierlich in Ereignis-Logs sowie nach jedem Parsivald-Schritt in einem atomistischen Dateiformat, über das strukturelle Eigenschaften bestimmt werden können.

Eine typische Abscheidungs-Simulation verläuft nach folgendem Schema:
Zuerst wird das Substrat aus einer Datei gelesen und periodisch auf die Größe des Simulationsraums erweitert.
Das Substrat muss xy-periodische Anschlussbedingungen und eine Mindesthöhe entsprechend der Größe der MD-Box erfüllen und stabil gegenüber den genutzten MD-Potentialen sein.
Es lassen sich ansonsten beliebig gemischte und strukturierte Substrate verwenden.
Pro Precursorart und Schritt (Halbzyklus für ALD) wird eine Ereignismaske vorbereitet, welche die physikalischen und numerischen Parameter inklusive der MD-Masken beinhaltet.
Anschließend beginnt die Hauptschleife mit dem ersten Parsivald-Schritt, in dem mögliche Ereignisse auf der Oberfläche gesucht werden, die vom KMC-Algorithmus nacheinander ausgewählt und von MD-Workern möglichst parallel simuliert werden.
Bei Erreichen der vorgegebenen Simulationszeit startet der nächste Schritt im Zyklus.
Die Abscheidungssimulation endet nach einer vorgegebenen Anzahl von Zyklen oder bei vollständiger Füllung des Simulationsraumes mit Atomen.

Mit diesem Schema lassen sich auch CVD- und PVD-Prozesse simulieren, indem der Zyklus dem zu simulierenden Prozesse angepasst wird (Abbildung~\ref{fig:parsivald-modes}).
%%auf einen einzigen Schritt beschränkt wird und bei CVD-Simulationen mehrere Arten von Ereignissen gleichzeitig betrachtet.
%%Der Zyklen-Timeout kann dann zur Kontrolle der Ausgabefrequenz atomistischer Strukturen genutzt werden.
Die Ereignisraten ergeben sich für PVD-Prozesse aus der Adsorptionsrate der Atome auf der Oberfläche, für CVD und ALD hingegen aus der Arrhenius-Gleichung.
%% Für ALD und CVD lässt sich die Wachstumsrate aus der Reaktionskinetik abschätzen, während sie bei PVD-Simulationen durch den entsprechenden Prozessparameter festgelegt wird.

\begin{figure}[hbp]
  \captionsetup[subfigure]{singlelinecheck=false}
  \begin{subfigure}[t]{5.7cm}
    \def\svgwidth{\textwidth}
    \input{img/parsivald-modes-ald.pdf_tex}
    \subcaption{ALD-Modus}
  \end{subfigure}
  \hfill
  \begin{subfigure}[t]{4.7cm}
    \def\svgwidth{\textwidth}
    \input{img/parsivald-modes-cvd.pdf_tex}
    \subcaption{CVD-Modus}
  \end{subfigure}
  \hfill
  \begin{subfigure}[t]{3cm}
    \def\svgwidth{\textwidth}
    \input{img/parsivald-modes-pvd.pdf_tex}
    \subcaption{PVD-Modus}
  \end{subfigure}
  \caption[Prozesszyklen für ALD, CVD und PVD]{
    Prozesszyklen für ALD, CVD und PVD.
    Atomistische und statistische Ausgaben erfolgen nach jedem Schritt.
    ALD (a) nutzt einen Schritt pro Halbzyklus.
  }
  \label{fig:parsivald-modes}
\end{figure}

\subsection{Annahmen und Einschränkungen}

Die folgenden Annahmen werden getroffen, um stabile Prozesse mit guter Skalierbarkeit auf großen Zeit- und Raumskalen beschreiben zu können:

\begin{enumerate}
\setlength\itemsep{0ex}
\item Alle Ereignisse finden auf der Oberfläche statt
\item Ereignisse sind zeitlich und räumlich getrennt
\item Oberflächendiffusion ist vernachlässigbar
\item Bulkdiffusion ist vernachlässigbar
\end{enumerate}

Die Beschränkung des Modelles auf diffusionsarme Oberflächenabscheidungen ergibt sich aus der Notwendigkeit, Bereiche des Simulationsraumes im statischen Gleichgewicht, in dem keine Ereignisse statt finden, zugunsten der Rechenzeit bei der MD-Simulation zu vernachlässigen.
Das betrifft die Gasphase und den größten Teil des Bulkmateriales ebenso wie abgeschirmte Bereiche der Oberfläche.
Für Prozesse, welche die oben aufgelisteten Annahmen nicht unterstützen, degeneriert Parsivald zu einer reinen MD-Simulation mit enormem Overhead.
Ein beschränktes Maß an Oberflächendiffusion lässt sich zwar mit längeren Relaxationszeiten und separaten Relaxations-Ereignissen behandeln, allerdings kann aufgrund der begrenzten Größe der MD-Boxen nur die Diffusion weniger Atome behandelt werden kann.

Weiterhin ist Parsivald auf die zugrunde liegende Molekulardynamik beschränkt.
So lassen sich nur Systeme simulieren, für die Potentialparametrisierungen existieren, die sowohl Bulksysteme als auch Oberflächen darstellen können, wie es bei EAM- und vielen ReaxFF-Potentialen der Fall ist.
ReaxFF-Potentiale sind um mehrere Größenordnungen rechenaufwendiger als EAM-Potentiale, welche in gleichem Maße rechenaufwendiger als Paarpotentiale sind, weshalb große Systeme nicht mehr mit reiner Molekulardynamik berechenbar sind und sich die Effizienz von Parsivald bemerkbar macht.

Werden zusätzlich Moleküle und deren Reaktionen mit Oberflächenliganden dargestellt, lassen sich Precursor-Oberflächen-Reaktionen direkt in Parsivald simulieren.
Andernfalls muss der Precursor über sein abzuscheidendes Zentralatom und zusätzliche Mechanismen wie explizite sterische Hinderung angenähert werden.
Die Suche nach Ereignisorten würde dann über exponierte Oberflächenatome und Revisionslisten statt über Oberflächenliganden angenähert.
Eine tatsächliche Anwendbarkeit beider Methoden muss für jeden Prozess einzeln abgeschätzt werden, da die nun fehlenden Precursorliganden strukturell entscheidend für den Aufbau der abgeschiedenen Schicht sein können.

\subsection{Erweiterungen im Rahmen der Masterarbeit}

Mit der vorliegenden Arbeit wurde das Parsivald-Modell und seine Implementierung um PVD- und CVD-Modi (Abbildung~\ref{fig:parsivald-modes}), eventgebundene sterische Hinderung zum Zweck der CVD-Simu\-lation, ein allgemeines Konfigurationsformat und allgemeine MD-Masken erweitert.
Intern kam die Unterstützung verschiedener atomistischer Dateiformate, die Einbettung der LAMMPS-Umgebungs\-variablen zur Potentialsuche, globale und lokale Suchpfade für alle Eingabedateien und ein interner sowie beliebig viele externe Workerpools dazu.
Ein standardisiertes Buildsystem sorgt für schnelle und sichere Kompilierung der Software, und eine Vielzahl externer Werkzeuge zur Vorbereitung und Analyse von Prozessen ermöglicht schnelle Zyklen der Parameteroptimierung und Auswertung.
Die entwickelten Werkzeuge beinhalten Programme zur Konvertierung zwischen atomistischen Dateiformaten, zur Korrektur von Konvertierungsfehlern, zur Bestimmung von Alpha-Formen und radialen Verteilungsfunktionen und deren Auswertung, sowie zur allgemeinen Manipulation und Auswertung von atomistischen Dateien.

Diese Änderungen ermöglichen eine einfachere Vorbereitung, Simulation und Untersuchung verschiedener Prozesse und eine aussagekräftigere Analyse der Ergebnisse.
Auch eine automatisierte Optimierung der Simulation durch selbstständige Anpassung der Prozessparameter wie Temperatur, Druck und Expositionszeit oder MD-spezifischer Parameter wie Thermostatdämpfung, Relaxationsdauer oder Zeitschrittweite ist mittels des Konfigurationsmechanismus' denkbar.

%% Zusätzlicher Aufwand musste beim Einkapseln der MD-Bibliothek LAMMPS betrieben werden (Abschnitt~\ref{lammpssucks}).

\subsection{Behandlung von fehlerhaften Ereignissen}

Parsivald-Simulationsläufe geben neben der atomistischen Struktur verschiedene Statistiken und Werte in Form von Daten- und Logdateien aus.
Das beinhaltet die Zahl der versuchten, erfolgreichen und fehlgeschlagenen Ereignisse, laufende Worker, überlagerte und deshalb zurückgestellte Ereignisse und die Anzahl aller Atome.
Optional lässt sich eine Verteilung der Häufigkeit eines Zugriffes auf Positionen in der xy-Ebene angeben, um bei reaktiven Prozessen die Auswahlkriterien von Ereignisorten zu prüfen.
Anhand der atomistischen Struktur lassen sich Oberflächenrauheiten, Porenverteilungen, Dichten, Schichtdicken und Eigenschaften eventueller Kristalle bestimmen (Abschnitt~\ref{mdmethods}).

Die Auswertung der Abbruchrate, also des Verhältnisses von fehlgeschlagenen Ereignissen zu versuchten Ereignissen, ist besonders bei der Optimierung von Prozessparametern wichtig, da sich über sie Fehler in der Prozesskonfiguration oder in der gebildeten Struktur frühzeitig erkennen lassen.
Ein Abbruch ist dabei eine MD-Simulation, die abstürzt, ihre Zeitbegrenzung erreicht oder zur Desorption von Atomen und Molekülen führt.
Da die LAMMPS-Bibliothek mit einer geringen Wahrscheinlichkeit unvorhergesehen und ohne Fehlernachrichten abstürzen kann, obwohl die Simulation selbst erfolgreich verlaufen wäre, werden fehlgeschlagene MD-Boxen einem zweiten Prozess zur Simulation übergeben.
Schlägt auch diese fehl, liegt mit hoher Wahrscheinlichkeit ein struktureller oder methodischer Fehler vor.
In der aktuellen Implementierung gilt das Herausschlagen von Atomen aus der Oberfläche, wie es bei Sputter-Prozessen statt finden kann, als Abbruch, weshalb in der vorliegenden Arbeit geringe Teilchenenergien oder schnelle Thermostatdämpfungen eingesetzt werden.

Da bei Abbrüchen bereits ausgewählte Ereignisse verworfen und somit die Abscheidungsraten leicht unterschätzt werden, müssen die Ereignisraten dynamisch entsprechend der mittleren oder erwarteten Abbruchrate angepasst werden.
Für CVD-Simulationen ist eine Anpassung der Ereignisraten in der Regel nicht notwendig, da fehlerhafte MD-Ereignisse auf eine versehentliche Auswahl eines Ereignisortes zurück zu führen sind.
Die Auswahlkriterien der Ereignisorte sollten alle potentiellen Ereignisse beinhalten, wobei einige weitere Orte versehentlich ausgewählt werden können, die keine Adsorptionen zulassen.
Eine präzisere Beschreibung der Nachbarschaft und des Bindungszustandes eines Atomes könnte helfen, die Auswahlkriterien zu präzisieren.
Bisherige Parsivald-Simulationen zeigen Abbruchraten von \SI{<1}{\percent} für stabile Prozesse, wo hingegen fehlerhaft eingestellte Simulationen Werte \SI{>10}{\percent} aufweisen.
