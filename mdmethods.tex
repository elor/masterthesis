\section{MD-Simulationen: Methoden und Auswertungen}
\label{mdmethods}

Dieser Abschnitt soll eine Übersicht über die molekulardynamischen Vorgehensweisen geben, welche im folgenden Kapitel~\ref{results} zur Anwendung kommen.
Diese Methoden werden für die Präparation von Strukturen, deren thermische und strukturelle Untersuchung sowie innerhalb der Parsivald-Simulationen selbst eingesetzt.

\subsection{Zeitskalen in MD-Simulationen}

MD-Simulationen werden maßgeblich durch den Zeitschritt $\Delta t$, die Thermostatdämpfung $\tau$, die Barostatdämpfung $\tau_p$ und die Relaxationszeit $t_\text{relax}$ bestimmt, welche systemabhängige Maximalwerte nicht überschreiten dürfen.
$\Delta t$ ist so zu wählen, dass schnelle Bewegungen nur geringe numerische Fehler verursachen, wofür je nach Art des numerischen Integrators mindestens \SI{20}{\percent} der Periodendauer der schnellsten Schwingung als Richtwert angenommen wurde.

$\tau$ und $\tau_p$ sollten groß genug gewählt werden, um zwischen den Zeitschritten nur geringe Änderungen vorzunehmen, da sonst die Dynamik des Systems gestört wird.
Typische Werte sind $\tau = 100 \Delta t$ und $\tau_p = 1000 \Delta t$.
Die Relaxationszeit $t_\text{relax}$ muss zuletzt oberhalb der vom Thermostat und von Barostat erzeugten Oszillationen der Temperatur und der Dichte liegen.

\subsection{Relaxierungen}

Ausgehend von einer präparierten Struktur wird das System auf die Relaxationstemperatur erwärmt, dort im kanonischen Ensemble für die Relaxationsdauer simuliert und bei Bedarf wieder auf eine Zieltemperatur abgekühlt.
Erwärmung und Abkühlung sind zwar optional, allerdings bei einigen Strukturen zur Reduktion numerischer Fehler notwendig, wie etwa zur Wahrung der Anschlussbedingungen der MD-Simulationen innerhalb von Parsivald.
Vergleiche zwischen Parsivald und reinen MD-Simulationen sollen die Ähnlichkeit der abgeschiedenen Strukturen prüfen, um Fehler aufgrund der gewählten Temperaturen zu vermeiden.

\subsection{Strukturanalysen}

Zur Auswertung einer Struktur bietet sich zuerst eine visuelle Beurteilung an, die mit entsprechenden Visualisierungsprogrammen vorgenommen werden kann.
Im Anschluss können die Dichte, Radiale Verteilungsfunktion, Bindungslänge und Koordinationszahl bestimmt werden, um die Struktur weiter zu charakterisieren und Aussagen über Kristallbildung treffen zu können.
Bei Kristallen kann eine Alpha-Form (Abschnitt~\ref{dataalphaform}) mit einem $\alpha$-Wert knapp oberhalb der Bindungslänge genutzt werden, um Fehlstellen im Gitter oder Gitterversetzungen zu identifizieren.

\subsection{Bestimmung der Dichte und Temperatur}

Zur Bestimmung der Dichte des Blocks eines amorphen Materials stehen verschiedene Möglichkeiten zur Verfügung, die letztendlich alle auf Zählung der Atome innerhalb eines begrenzten Volumens zurückgeführt werden können.
Da das Volumen des Simulationsraumes im isotherm-isobaren Ensemble durch das Barostat oszilliert, muss sie zeitlich über einen größeren Zeitraum als die Dämpfungskonstante $\tau_p$ gemittelt werden, da sie sonst nicht über die gesamte Periode der Oszillation betrachtet wird.

Ebenso muss zur Bestimmung der Temperatur auf das Konvergenzverhalten des Thermostates geachtet werden.
Die Temperatur wird aus dem Mittelwert der kinetischen Energie aller Teilchen bestimmt, welche jedoch schwanken kann, weshalb wiederum die Bildung eines Zeitmittelwertes notwendig wird.

\subsection{Radiale Verteilungsfunktionen, Bindungslänge und Koordinationszahl}

Die Radiale Verteilungsfunktion (RDF) $g_{\alpha\beta}(r)$ zweier Atomsorten $\alpha$ und $\beta$ ergibt sich als Histogramm der Zahl der Atome $n_{\alpha\beta}(r)$ in Abhängigkeit des Abstandes $r$, gewichtet über das Volumen der Bins, welche durch die radiale Beschreibung die Form von Kugelschalen annehmen.
\begin{equation}
  g_{\alpha\beta}(r) = \frac{1}{\rho_\alpha} \frac{d n_{\alpha\beta}(r)}{4 \pi r^2 dr} ~~,\quad \rho_\alpha = \frac{N_\alpha}{V}
\end{equation}
Ihr erstes Maximum bildet die Bindungslänge, während das Integral bis zur Koordinationslänge (erstes Minimum nach der Bindungslänge) die erste Koordinationszahl ergibt.
An der RDF-Darstellung lässt sich außerdem die strukturelle Ordnung erkennen:
RDF von perfekten Kristallen bestehen ausschließlich aus Spitzen an den Gitterpunkten, die auch für große Abstände noch klar erkennbar sind.
Bei amorphen Materialien hingegen konvergiert die Funktion schnell gegen den Grenzwert von 1, der anzeigt, dass keine langreichweitige Ordnung mehr vorhanden ist.

Beispiele radialer Verteilungsfunktionen finden sich in Abbildungen~\ref{fig:siliconrdf},~\ref{fig:amorphousrdf} und~\ref{fig:siliconresults-b}.

\subsection{Oberfläche, Schichtdicke, Rauheit und Porösität}
\label{mdmethods-surface}

Zur Bestimmung der Schichtdicke, Oberflächenrauheit oder Porösität muss zuerst die Oberfläche einer Schicht aus der Alpha-Form mittels Delaunay-Triangulation in Form einer Punktwolke bestimmt werden (Abschnitt~\ref{datadelaunay}).
Im Anschluss lässt sich aus der entstandenen Punktwolke der tiefste und höchste Punkt der Oberfläche sowie die Spannweite in z-Richtung bestimmen und durch Bildung des Mittelwertes die mittlere Schichtdicke ermitteln.
Durch Zählung der chemischen Elemente in der Punktwolke lässt sich zudem die Bedeckung der Oberfläche mit chemischen Gruppen bestimmen.
Durch Begutachtung der Oberflächen können außerdem Rückschlüsse auf die Porösität gezogen werden, indem neben der eigentlichen Oberfläche auch sämtliche Poren und Hohlräume innerhalb der Struktur aus der Alpha-Form gelesen werden.

\subsection{Reaktionen und Stabilität von Molekülen}

Einzelne oder mehrere Precursormoleküle werden mit externer Software (Materials Studio\cite{biovia_materials_2014} und Packmol\cite{martinez_packmol:_2009}) präpariert, mit Geschwindigkeiten im Rahmen der Zieltemperatur versehen und im mikrokanonischen Ensemble im Vakuum simuliert, um Aussagen über ihre Stabilität zu erhalten.
Diese wird bei einzelnen Datensätzen manuell durch Begutachtung der dynamischen Eigenschaften und der entstandenen Moleküle ermittelt, bei vielen Datensätzen wird eine automatische Auswertung der Konnektivität der Moleküle durchgeführt, wodurch Aussagen über die entstandenen Moleküle getroffen werden können.
Die gleichen Untersuchungen geben Aufschluss über den Erfolg einer Reaktion.
