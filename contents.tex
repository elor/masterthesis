%%======================================================================
%%      Inhaltsverzeichnis
%%======================================================================
%\clearpage

%\cleardoubleemptypage

%\pdfbookmark{Inhaltsverzeichnis}{Inhaltsverzeichnis}
\tableofcontents

%%======================================================================
%%      Abbildungsverzeichnis
%%======================================================================
%\cleardoublepage
\markboth{Abbildungsverzeichnis}{Abbildungsverzeichnis}
\listoffigures
\todo[inline]{Beschreibungen der Abbildungen überarbeiten}

%%======================================================================
%%      Tabellenverzeichnis
%%======================================================================
%\cleardoublepage
\markboth{Tabellenverzeichnis}{Tabellenverzeichnis}
\listoftables
\todo[inline]{Beschreibungen der Tabellen überarbeiten}

%======================================================================
%  Literaturverzeichnis
%======================================================================
%\cleardoublepage
%\manualmark
%\markboth{Literaturverzeichnis}{Literaturverzeichnis}
%\bibliographystyle{unsrt}
%\bibliographystyle{own}
%\bibliography{literatur}

%%======================================================================
%%      Algorithmenverzeichnis
%%======================================================================
%% \renewcommand{\listalgorithmname}{Algorithmenverzeichnis}
%% \addcontentsline{toc}{chapter}{Algorithmenverzeichnis}
%% \listofalgorithms

%%======================================================================
%%      Abkuerzungsverzeichnis
%%======================================================================
%\cleardoublepage
\chapter*{Abkürzungsverzeichnis}
\addcontentsline{toc}{chapter}{Abkürzungsverzeichnis}
\markboth{Abkürzungsverzeichnis}{Abkürzungsverzeichnis}
\def\listacronymname{Abkürzungsverzeichnis}

\newcolumntype{Y}{>{\flushleft\arraybackslash}X}

%\begin{acronym}[SQL]
% \acro{CNTs}{Kohlenstoffnanoröhrchen (engl. Carbon Nanotubes, CNTs)}
% \acro{DFT}{Dichtefunktionaltheorie}
%\end{acronym}

\def\arraystretch{1.3}
\begin{longtable}{lll}

AFM       & Atomic Force Microscopy             & Rasterkraftmikroskopie                                      \\
ALD       & Atomic Layer Deposition             & Atomlagenabscheidung                                        \\
CG        & Conjugated Gradients                & Konjugierte Gradienten                                      \\
CLI       & Command Line Interface              & Kommandozeile                                               \\
CVD       & Chemical Vapor Deposition           & Chemische Gasphasenabscheidung                              \\
DFT       & Density Functional Theory           & Dichtefunktionaltheorie                                     \\
EAM       & Embedded Atom Method                & Eingebettete-Atom-Methode                                   \\
FEM       & Finite Element Method               & Finite-Elemente-Methoden                                    \\
GMR       & Giant Magnetoresistance             & Riesenmagnetowiderstand                                     \\
GPC       & Growth per Cycle                    & Schichtwachstum pro ALD-Zyklus                              \\
KMC       & Kinetic Monte Carlo                 & Kinetische Monte Carlo-Methoden                             \\
MMC       & Metropolis Monte Carlo              & Metropolis Monte Carlo-Methoden                             \\
MD        & Molecular Dynamics                  & Molekulardynamik                                            \\
MEAM      & Modified Embedded Atom Method       & Modifizierte Eingebettete-Atom-Methode                      \\
MOSFET    & Metal Oxide Field Effect Transistor & Metalloxid-Feldeffekt-Transistor                            \\
MPI       & Message Passing Interface           & Eine Kommunikationsschnittstelle                            \\
NB        & Neighborhood                        & (atomare) Nachbarschaft                                     \\
PVD       & Physical Vapor Deposition           & Physikalische Gasphasenabscheidung                          \\
RAM       & Random-Access Memory                & Arbeitsspeicher                                             \\
RDF       & Radial Distribution Function        & Radiale Verteilungsfunktion                                 \\
RMS       & Root Mean Square                    & Standardabweichung                                          \\
ReaxFF    & Reactive Force Fields               & Reaktive Kraftfelder                                        \\
TMA       & Trimethylaluminum (\ce{Al(CH3)3})   & Trimethylaluminium (\ce{Al(CH3)3})                          \\
NPT       & \multicolumn{2}{l}{Großkanonisches Ensemble: Teilchenzahl, Druck und Temperatur sind konstant}    \\
NVE       & \multicolumn{2}{l}{Mikrokanonisches Ensemble: Teilchenzahl, Volumen und Energie sind konstant}    \\
NVT       & \multicolumn{2}{l}{Kanonisches Ensemble: Teilchenzahl, Volumen und Temperatur sind konstant}      \\
Parsivald & \multicolumn{2}{l}{Parallel Atomistic Reaction Simulator for Vapor and Atomic Layer Depositions}  \\

\end{longtable}

\begin{comment}
  Liste der Abkürzungen, die nicht weiter erklärt werden:
  (Comment-Umgebung, damit sie vom Parser trotzdem erfasst werden)

  ENAS      & \multicolumn{2}{l}{Fraunhofer-Institut für Elektronische Nano-Systeme}                            \\
  LAMMPS    & \multicolumn{2}{l}{Large-scale Atomic/Molecular Massively Parallel Simulator (MD-Bibliothek)}     \\
  AMBER
  GROMACS
  BIOVIA
  TU
  EU
  ACCELERATE
\end{comment}

\todo[inline]{NB vorstellen?}
\todo[inline]{MMC: schönerer Name?}
\todo[inline]{Beschreibung der Ensembles hier oder nur im Text?}
\todo[inline]{Prüfen, ob die Abkürzungen im Text ordentlich eingeführt werden}

\chapter*{Symbolverzeichnis}
\addcontentsline{toc}{chapter}{Symbolverzeichnis}
\markboth{Abkürzungsverzeichnis}{Symbolverzeichnis}
\def\listacronymname{Symbolverzeichnis}

\begin{tabularx}{\linewidth}{ll}
$E_\text{A}$       & Aktivierungsenergie            \\
$p_\text{partial}$ & Partialdruck                   \\
$\sigma_z$         & RMS-Rauheit einer Oberfläche   \\
$\tau$             & Dämpfungszeit von Thermostaten \\
\end{tabularx}
\todo[inline]{die wichtigsten Symbole automatisch aus allen Formeln exportieren}
\todo[inline]{Verweise von Gleichungen durch Lokalität ersetzen}
\todo[inline]{Kein neuer Absatz nach Gleichungen}

%\printglosstex(acr)

%%======================================================================
%%      Ende
%%======================================================================
\cleardoublepage
\pagenumbering{arabic} % Fäng erneut bei 1 an.
