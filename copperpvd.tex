\section{Kupfer-PVD}
\label{copperpvd}

Ein zweites PVD-System stellt Kupfer dar, für das eine Vielzahl an unterschiedlichen Parametrisierungen vorliegt (Tabelle \ref{tab:copperpots}).
Es stellt sich die Aufgabe, darunter eine passende Parametrisierung zu suchen und zu entscheiden, inwiefern eine Vorauswahl anhand weniger Parameter \todo{chword} möglich ist.

\begin{table}[hbtp]
  \caption[EAM-Parametrisierungen für Kupfersysteme]{EAM-Parametrisierungen für Kupfersysteme.}
  \label{tab:copperpots}
  \rowcolors{0}{white}{lightgray}
  \begin{tabularx}{\textwidth}{|lXc|}
    \hline
    \textbf{Bezeichnung} & \textbf{Anwendung \& Kommentare} & \textbf{Ref.} \\
    \hline
    CuAg.eam.alloy & Strukturelle und thermische Eigenschaften von \ce{Cu-Al} & \cite{williams_embedded-atom_2006} \\
    cu\_ag\_ymwu.eam.alloy & Mono-, Di-, Trimere und Inseln von \ce{Cu} auf \ce{Ag} & \cite{wu_cu/ag_2009} \\
    Cu\_smf7.eam & Oberflächen von \ce{Ni-Cu}-Legierungen bei \SI{800}{\kelvin} & \cite{foiles_calculation_1985} \\
    Cu\_u3.eam & Oberflächen und Bulks verschiedener Legierungen & \cite{foiles_embedded-atom-method_1986} \\
    Cu\_u6.eam & Aktivierungsenergie für Eigendiffusionen & \cite{adams_self-diffusion_1989} \\
    Cu-Zr\_2.eam.fs & Flüssige und amorphe \ce{Cu-Zr}-Legierungen & \cite{mendelev_development_2009} \\
    Cu-Zr.eam.fs & Flüssige und amorphe \ce{Cu-Zr}-Legierungen & \cite{mendelev_using_2007} \\
    Mendelev\_Cu2\_2012.eam.fs & Unterkühlte \ce{Al-Cu}-Schmelzen. Basiert auf \cite{mendelev_analysis_2008} & \cite{_interatomic_2014} \\
    \hline
  \end{tabularx}
  
\end{table}

Viele der Parametersätze wurden für Legierungen angepasst, die Cu-Zr-Potentiale sind sogar mit der Warnung versehen, man könne ein reines Metall damit nicht mehr verlässlich simulieren.
Ein Hinweis auf die Kompatibilität mit LAMMPS oder den untersuchten Systemen ließ sich häufig nicht finden, weshalb Testrechnungen an Kupfer-Bulks und -Oberflächen durchgeführt wurden.

\subsection{Voruntersuchungen}

Mit Ausnahme der drei Parametersätze Cu\_u3.eam, Cu\_u6.eam und Cu\_smf7.eam konnten die Potentiale nicht von LAMMPS zur Simulation genutzt werden.
Es zeigten sich dabei Probleme beim Laden der Dateien, kryptische Fehlerausgaben nach einigen Schritten oder ein Aufhängen der Simulation ohne Vorankündigung.
Die Ergebnisse der verbliebenen Potentiale sind jedoch in guter Übereinstimmung mit Literaturwerten (Tabelle \ref{tab:copperpreresults}).
\todo[inline]{Dichte}
Im Gegensatz zu den ebenfalls durch EAM-Potentiale simulierten Goldsystemen wurde der Schmelzpunkt nicht zuverlässig simuliert (Abbildung \ref{fig:copperthermo}), was durch die sonst geringen Simulationstemperaturen vernachlässigbar ist.

\begin{table}[hbtp]
  \rowcolors{0}{white}{lightgray} 
  \caption[Eigenschaften von Kupfer]{Vergleich der Eigenschaften von Kupfer mit experimentellen und Literaturdaten als Voruntersuchung des PVD-Prozesses\todo[inline]{ref}}
  \label{tab:copperpreresults}
  \begin{tabularx}{\textwidth}{|lXXXX|}
    \hline
    \textbf{unters. Größe} & \textbf{Experiment} & \textbf{Cu\_smf7.eam} & \textbf{Cu\_u3.eam} & \textbf{Cu\_u6.eam} \\
    \hline
    Koordination   &  \SI{12.00}{} & \SI{12.00}{} & \SI{12.00}{} & \SI{12.00}{} \\
    Bindungslänge  &  \SI{2.556}{\angstrom} & \SI{2.558}{\angstrom} (\SI{0.08}{\percent}) & \SI{2.558}{\angstrom} (\SI{0.08}{\percent}) & \SI{2.558}{\angstrom} (\SI{0.08}{\percent}) \\
    Dichte         & \SI{8.92}{\gram\per\cubic\centi\meter} & \SI{8.908}{\gram\per\cubic\centi\meter} (\SI{-0.13}{\percent}) & \SI{8.915}{\gram\per\cubic\centi\meter} (\SI{-0.06}{\percent}) & \SI{8.910}{\gram\per\cubic\centi\meter}  (\SI{-0.11}{\percent}) \\
    \hline
  \end{tabularx}
\end{table}

\todo[inline]{Oberflächenvalidierung?}

\begin{figure}[tbp]
  \centering
  \captionsetup[subfigure]{singlelinecheck=false}
  \def\subfigwidth{7cm}
  \begin{subfigure}[t]{\subfigwidth}
    \includegraphics[width=\textwidth]{Cu_u6_meltingpoint}
    \subcaption{Phasenübergang mit Cu\_u6.eam bei unterschiedlichen $t_\text{relax}$}
  \end{subfigure}
  \hfill
  \begin{subfigure}[t]{\subfigwidth}
    \includegraphics[width=\textwidth]{Cu_smf7_meltingpoint}
    \subcaption{Phasenübergang mit Cu\_smf7.eam bei unterschiedlichen $t_\text{relax}$}
  \end{subfigure}
  \caption[Abweichung der Schmelztemperaturen bei Kupfer-MD]{
    Abweichung der Schmelztemperatur mit verschiedenen Parametrisierungen.
    Experimentelle Werte von Brillo et al.\cite{brillo_density_2006}.
  }
  \label{fig:copperthermo}
\end{figure}

\subsection{Prozess-Simulation}

Aufgrund der Ähnlichkeit des Gold-PVD-Prozesses wurden dessen Parameter für die Kupfer-PVD übernommen und auf dessen Eigenschaften leicht angewandt.
So liegen kleinere Bindungslängen und geringere Massen vor, die beispielsweise zu erhöhten \todo{wirklich?}Auftreffgeschwindigkeiten führen.

Zu Beginn der Simulation ergeben sich hohe Abbruchquoten von \SI{25}{\percent}, die im Laufe der Simulation nachlassen.
Genauere Untersuchungen zeigen, dass bei Ankunft eines neuen Kupferatomes auf der glatten Gitteroberfläche ein vorhandenes Atom herausgeschlagen wird.
Sobald die Oberfläche mit genügend Off-Lattice-Atomen versehen ist, verschwindet dieser Effekt.
Die kritische Bedeckung liegt zwischen \SI{0.34}{\per\nano\meter\squared} und \SI{0.74}{\per\nano\meter\squared}.
Das deckt sich gut mit der maximalen MD-Ereignisdichte von \SI{0.098}{\per\nano\meter\squared} unter der Annahme, dass mehrere Atome an der Abschwächung des Effektes beteiligt sind\todo{Formulierung}.
Es ist also zu vermuten, dass perfekte Gitterkonfigurationen nicht robust gegenüber gerichteten Energieeinträgen ist, kleine Perturbationen der Atompositionen aber zur gleichmäßigeren Verteilung der eingebrachten Energien führen.
\todo[inline]{Entropie?}

\begin{figure}[bt]
  \captionsetup[subfigure]{singlelinecheck=false}
  \def\subfigwidth{0.31\textwidth}
  \begin{subfigure}[t]{\subfigwidth}
    \includegraphics[width=\textwidth]{Au_substrate_flat}
    \subcaption{Glattes Gold-Substrat}
    \label{fig:coppersubstrate-a}
  \end{subfigure}
  \hfill
  \begin{subfigure}[t]{\subfigwidth}
    \includegraphics[width=\textwidth]{Au_substrate_step30}
    \subcaption{Gold-Stufe, \SI{30}{\degree}}
    \label{fig:coppersubstrate-b}
  \end{subfigure}
  \hfill
  \begin{subfigure}[t]{\subfigwidth}
    \includegraphics[width=\textwidth]{Au_substrate_tip60}
    \subcaption{Gold-Spitze, \SI{60}{\degree}}
    \label{fig:coppersubstrate-c}
  \end{subfigure}
  \caption[Strukturierte Coppersubstrate]{Coppersubstrate mit unterschiedlicher Struktur und Breite und Tiefe von \SI{100}{\angstrom}.
    Abscheidungen wurden auf glatten Substraten, Stufen und Spitzen vorgenommen.}
  \label{fig:coppersubstrate}
\end{figure}

\begin{figure}[bt]
  \captionsetup[subfigure]{singlelinecheck=false}
  \def\subfigwidth{0.31\textwidth}
  \begin{subfigure}[t]{\subfigwidth}
    \includegraphics[width=\textwidth]{Au_deposition_flat}
    \subcaption{Abscheidung auf glattem Gold-Substrat}
    \label{fig:copperdepositions-a}
  \end{subfigure}
  \hfill
  \begin{subfigure}[t]{\subfigwidth}
    \includegraphics[width=\textwidth]{Au_deposition_step30}
    \subcaption{Abscheidung auf Gold-Stufe, \SI{30}{\degree}}
    \label{fig:copperdepositions-b}
  \end{subfigure}
  \hfill
  \begin{subfigure}[t]{\subfigwidth}
    \includegraphics[width=\textwidth]{Au_deposition_tip60}
    \subcaption{Abscheidung auf Gold-Spitze, \SI{60}{\degree}}
    \label{fig:copperdepositions-c}
  \end{subfigure}
  \caption[Abscheidung auf strukturierten Substraten]{
    Ergebnis der Abscheidung.
    Die Substratstruktur bleibt erkennbar, wird aber nach oben verstärkt, ansonsten aber kristallin und glatt fortgesetzt.
  }
  \label{fig:copperdepositions}
\end{figure}

Das Kristallsubstrat wird auch hier fortgesetzt, jedoch verstärken sich Neigungswinkel an Stufen und Spitzen zunehmend.
Nach längeren Laufzeiten entstehen somit Überhänge, die durch Abschluss von unten zu Hohlräumen innerhalb der Struktur führen, welche in der Realität durch thermische Relaxation geschlossen würden.
Dahinter steht einerseits die Notwendigkeit, Gold-Atome bei Ankunft auf der Oberfläche ausreichend diffundieren zu lassen, was beim aktuellen Modell nur innerhalb der Reaktionsraumgrenzen geschieht.

Andererseits liegt ein methodischer Fehler bei Nutzung von Binning-Methoden vor:
Die Oberfläche wird aus algorithmischen Gründen nur entlang der z-Achse bestimmt, woraufhin das neue Atom oberhalb eines Referenzatomes auf der Oberfläche platziert wird.
An Stufen und Kanten werden hierbei höhere Reaktionsorte bevorzugt, an denen das Atom mit \todo{phrasing}statistischer Wahrscheinlichkeit verbleibt.

Einen Lösungsansatz stellt die ausführliche Parametrisierung der Oberfläche dar, beispielsweise per Alpha-Form (siehe Abschnitt \ref{dataalphaform}, über die man die Ereigniswahrscheinlichkeit entsprechend der Einbettungsenergie variierte, angenähert über die Oberflächenkrümmung.
