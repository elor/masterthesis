%%======================================================================
%%      Inhaltsverzeichnis
%%======================================================================
%\clearpage

%\cleardoubleemptypage

%\pdfbookmark{Inhaltsverzeichnis}{Inhaltsverzeichnis}
\tableofcontents

%%======================================================================
%%      Abbildungsverzeichnis
%%======================================================================
%\cleardoublepage
\markboth{Inhaltsverzeichnis}{Abbildungsverzeichnis}
\listoffigures

%%======================================================================
%%      Tabellenverzeichnis
%%======================================================================
%\cleardoublepage
\markboth{Abbildungsverzeichnis}{Tabellenverzeichnis}
\listoftables

%======================================================================
%  Literaturverzeichnis
%======================================================================
%\cleardoublepage
%\manualmark
%\markboth{Literaturverzeichnis}{Literaturverzeichnis}
%\bibliographystyle{unsrt}
%\bibliographystyle{own}
%\bibliography{literatur}

%%======================================================================
%%      Algorithmenverzeichnis
%%======================================================================
%% \renewcommand{\listalgorithmname}{Algorithmenverzeichnis}
%% \addcontentsline{toc}{chapter}{Algorithmenverzeichnis}
%% \listofalgorithms

%%======================================================================
%%      Abkuerzungsverzeichnis
%%======================================================================
%\cleardoublepage
\chapter*{Abkürzungsverzeichnis}
\addcontentsline{toc}{chapter}{Abkürzungsverzeichnis}
\markboth{Abkürzungsverzeichnis}{Abkürzungsverzeichnis}
\def\listacronymname{Abkürzungsverzeichnis}

\newcolumntype{Y}{>{\flushleft\arraybackslash}X}

%\begin{acronym}[SQL]
% \acro{CNTs}{Kohlenstoffnanoröhrchen (engl. Carbon Nanotubes, CNTs)}
% \acro{DFT}{Dichtefunktionaltheorie}
%\end{acronym}

\def\arraystretch{1.3}
\begin{tabular}{lll}
ALD       & Atomic Layer Deposition             & Atomlagenabscheidung                                       \\
CFD       & Computational Fluid Dynamics        & Eine Methode zur Gasfluss/Simulation                       \\
CVD       & Chemical Vapor Deposition           & Chemische Gasphasenabscheidung                             \\
DFT       & Density Functional Theory           & Dichtefunktionaltheorie                                    \\
EAM       & Embedded Atom Method                & Eingebettete-Atom-Methode                                  \\
GPC       & Growth per Cycle                    & Schichtwachstum pro ALD-Zyklus                             \\
KMC       & Kinetic Monte Carlo                 & Kinetische Monte Carlo-Methoden                            \\
MMC       & Metropolis Monte Carlo              & Metropolis Monte Carlo-Methoden                            \\
MD        & Molecular Dynamics                  & Molekulardynamik                                           \\
MEAM      & Modified Embedded Atom Method       & Modifizierte Eingebettete-Atom-Methode                     \\
PVD       & Physical Vapor Deposition           & Physikalische Gasphasenabscheidung                         \\
RAM       & Random-Access Memory                & Arbeitsspeicher                                            \\
RDF       & Radial Distribution Function        & Radiale Verteilungsfunktion                                \\
RMS       & Root Mean Square                    & Standardabweichung, häufig für Rauheiten                   \\
ReaxFF    & Reactive Force Fields               & Reaktive Kraftfelder                                       \\
TMA       & Trimethylaluminum (\ce{Al(CH3)3})   & Trimethylaluminium (\ce{Al(CH3)3})                         \\
NPT       & \multicolumn{2}{l}{Bezeichnung für das Isotherm-Isobare Ensemble}                                \\
NVE       & \multicolumn{2}{l}{Bezeichnung für das Mikrokanonische Ensemble}                                 \\
NVT       & \multicolumn{2}{l}{Bezeichnung für das Kanonische Ensemble}                                      \\
Parsivald & \multicolumn{2}{l}{Parallel Atomistic Reaction Simulator for Vapor and Atomic Layer Depositions} \\
\end{tabular}

\begin{comment}
  Liste der Abkürzungen, die nicht weiter erklärt werden:
  (Comment-Umgebung, damit sie vom Parser trotzdem erfasst werden)

  NB     & Neighborhood            & (atomare) Nachbarschaft                                             \\
  CLI    & Command Line Interface  & Kommandozeile                                                       \\
  FEM    & Finite Element Method   & Finite-Elemente-Methoden                                            \\
  ENAS   & \multicolumn{2}{l}{Fraunhofer-Institut für Elektronische Nano-Systeme}                        \\
  LAMMPS & \multicolumn{2}{l}{Large-scale Atomic/Molecular Massively Parallel Simulator (MD-Bibliothek)} \\
  LMP
  GMR    & Giant Magnetoresistance & Riesenmagnetowiderstand                                             \\
  TMR
  AMBER
  CHARMM
  EON
  CRC
  LJ
  NIST
  NO
  OH
  GROMACS
  BIOVIA
  TU
  CG     & Conjugated Gradients    & Konjugierte Gradienten                                              \\
  EU
  ACCELERATE
  XXR, SI - latex table definitions
  Mathematische hyperref-Symbole: TE TMD RE
\end{comment}

\chapter*{Symbolverzeichnis}
\addcontentsline{toc}{chapter}{Symbolverzeichnis}
\markboth{Symbolverzeichnis}{Symbolverzeichnis}

\begin{longtable}{ll}
%% $\alpha$          & Linearer Ausdehnungskoeffizient                                     \\
%% $\alpha$          & Schrittweite bei MD-Minimierungen                                   \\
$\alpha$             & Parameter der Alpha-Form                                            \\
%% $d$               & Anzahl der Dimensionen                                              \\
$\Delta t$           & Zeitschrittweite                                                    \\
$E$                  & Energie                                                             \\
$E_i$                & Ein KMC-Ereignis                                                    \\
%% $F_\alpha$        & Einbettungsfunktion des EAM-Potentiales                             \\
$g_{\alpha\beta}(r)$ & Radiale Verteilungsfunktion der Atomsorten $\alpha$ und $\beta$     \\
$\kappa$             & Dielektrizitätskonstante eines Materiales                           \\
$k_B$                & Boltzmann-Konstante                                                 \\
$m$                  & Masse eines Teilchens                                               \\
$M$                  & Virtuelle Masse von Thermostaten                                    \\
$\nabla$             & Nabla-Operator                                                      \\
%% $n_\text{cyc.}$   & Zahl der simulierten KMC-Zyklen                                     \\
$p$                  & Anzahl paralleler Prozesse                                          \\
$p$                  & Druck                                                               \\
$p_\text{max,1}$     & Obere Schranke für $p$ für Simulationsräume                         \\
$p_\text{max,2}$     & Obere Schranke für $p$ für große Simulationsräume                   \\
$r_c$                & Cutoff-Radius eines Kraftfeldes                                     \\
$r_d$                & Umkugel-Radius eines Delaunay-Simplexes                             \\
$\rho_\text{worker}$ & Relative Workerdichte auf der Oberfläche des Simulationsraumes      \\
$R_i$                & Akkumulierte Ereignisrate von $E_i$                                 \\
$r_{ij}$             & Radialer Abstand zwischen Teilchen $i$ und $j$                      \\
$r_i$                & Ereignisrate von $E_i$                                              \\
$R_N$                & Summe aller Ereignisraten                                           \\
$r_s$                & Radius für Suchoperationen                                          \\
$r_\text{bond}$      & Bindungslänge                                                       \\
$\sigma_z$           & RMS-Rauheit einer Oberfläche                                        \\
$\tau$               & Dämpfungszeit von Thermostaten                                      \\
$\tau_p$             & Dämpfungszeit von Barostaten                                        \\
$T_1$                & Serielle Laufzeit einer Rechnung                                    \\
$T_p$                & Parallele Laufzeit einer Rechnung für $p$ Prozesse                  \\
$T$                  & Temperatur                                                          \\
$T_\text{E}$         & Laufzeit zur Auswahl eines KMC-Ereignisses                          \\
$T_\text{MD}$        & Laufzeit einer MD-Simulation                                        \\
$t_\text{relax}$     & Relaxationszeit einer MD-Simulation                                 \\
$u$,~$u'$            & $[0,1)$-gleichverteilte Zufallszahlen                               \\
$\vec F(X)$          & Kraft auf ein Teilchen                                              \\
$\vec p$             & Impulsvektor                                                        \\
$\vec v$             & Geschwindigkeitsvektor                                              \\
$V$                  & Volumen                                                             \\
$V(X)$               & Potentielle Energie eines Teilchens                                 \\
$w_\text{eff}$       & Maximale Breite eines Simulationsraumes für effiziente Simulationen \\
$w_\text{MD}$        & Breite der MD-Box                                                   \\
$w_\text{sim}$       & Breite des Simulationsraumes                                        \\
$X$                  & Zustand eines Systemes                                              \\
$\mathbb{X}$         & Zustandsraum eines Systems                                          \\
\end{longtable}
\todoline{Sortieren und Aussortieren}

%%======================================================================
%%      Ende
%%======================================================================
