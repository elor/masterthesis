\section{Silizium-PVD und Siliziumoxid-CVD}
\label{siliconpvd}

Anhand von Silizium-PVD sollen einerseits amorphe Schichten abgeschieden, andererseits zum \ce{SiH4}-\ce{O2}-CVD-Prozess hingeführt werden.
Als Substrat werden kristalline Strukturen verschiedener Kristallebenen ([100], [110] und [111]) untersucht.

\subsection{Potentialdateien}

Für die folgenden Simulationen werden ReaxFF-Potentiale mit der Aussicht genutzt, auch Reaktionen zwischen Oberflächenatomen und Precursormolekülen simulieren zu können.
Da die ReaxFF-Formulierung erst innerhalb des letzten Jahrzehntes an Popularität gewonnen hat\todo{ref auf van duin}, sind die verfügbaren Potentiale auf einzelne Probleme angepasst und oftmals noch nicht in den Potentialdatenbanken zu finden.
Tabelle \ref{tab:siliconpotentials} listet die im Rahmen der Arbeit untersuchten ReaxFF-Parametrisierungen auf.

\begin{table}
  \caption[Silizium-Potentialdateien]{Potentialdateien für Silizium- und Siliziumoxidsysteme}
  \label{tab:siliconpotentials}
  \rowcolors{0}{white}{lightgray}
  \begin{tabularx}{1\textwidth}{|lXc|}
    \hline
    \textbf{Bezeichnung} & \textbf{Anwendung \& Kommentare} & \textbf{Ref.} \\
    \hline
    Al\_AlO\_AlN    & \ce{Al}, \ce{Al2O3}, \ce{AlN}. Basiert auf einer Si-Parametrisierung                                      & \cite{plimpton_lammps_2014} \\
    chenoweth       & Zersetzung von Polydimethylsiloxane bei hohen Drücken und Temperaturen. Ergänzung von \ce{C-Si}-Bindungen & \cite{chenoweth_simulations_2005} \\
    kulkarni        & Reaktion von Sauerstoff mit \ce{OH}-terminierten Siliziumoxid-Oberflächen                                 & \cite{kulkarni_oxygen_2013} \\
    lg              & ``low gradients''. Siehe liu\_nitramines. Fehlerhafte Version aus LAMMPS                                  & \cite{liu_reaxff-lg:_2011} \\
    liu\_ettringite & Verspannung von Ettringit (\ce{Ca6[Al(OH)6]2(SO4)3 26H2O}). Basiert auf Si-Parametrisierung               & \cite{liu_development_2012} \\
    liu\_nitramines & Dichtebestimmung von Nitramin-Molekülen bei hohen Drücken. Dichte erhöht durch Van-der-Waals-Korrekturen  & \cite{liu_reaxff-lg:_2011} \\
    narayanan       & Präparation mit \ce{Li-Al}-Silikaten. Für Phasenübergänge von Eukryptit-Kristallen (\ce{LiAl[SiO4]})      & \cite{narayanan_reactive_2012} \\
    newsome         & Oxidation von \ce{SiC}-Oberflächen mit \ce{O2} und \ce{H2O} bei \SIrange{500}{5000}{\kelvin}              & \cite{newsome_oxidation_2012} \\
    nielson         & Reaktionskinetik an Metallkatalysatoren bei hohen Temperaturen                                            & \cite{nielson_development_2005} \\
    zhang           & Zersetzung energetischer Moleküle (Nitramin-Explosionen)                                                  & \cite{zhang_carbon_2009} \\
    \hline
  \end{tabularx}
\end{table}

\subsection{Potentialuntersuchungen}

Neben den Kristall-Optimierungen und -Relaxierungen werden auch Relaxierungen der Oberfläche und Präparationen amorpher Strukturen sowie Precursor-Simulationen als Potentialtests genutzt.
Zeigen diese gute Ergebnisse, kommen noch Simulationen von Precursor-Reaktionen und Precursor-Oberflächen-Reaktionen hinzu.
Das Ziel dieser Bemühungen ist eine Abschätzung der Darstellbarkeit des Prozesses per Molekulardynamik:
Sind neben den Bulks auch die die Precursormoleküle und Oberflächen sowie deren Reaktionen miteinander mit hoher Zuverlässigkeit darstellbar, ließe sich der Prozess ohne grobe Vereinfachung der Precursorchemie simulieren.
Die Ergebnisse dieser Betrachtungen wurden in Tabelle \ref{tab:siliconpreresults} zusammen gefasst und für ausgewählte Systeme in Tabelle \ref{tab:silicondetailedresults} näher analysiert.

\begin{table}
  \caption[Ergebnisse der Silizium-Potential-Untersuchungen]{Zusammenfassung Ergebnisse der Untersuchung der Potenialuntersuchung für Silizium-Systeme.
    Detailliertere Ergebnisse finden sich im \todo[inline]{Anhang}
  }
  \label{tab:siliconpreresults}
  \rowcolors{0}{white}{lightgray}
  \begin{tabularx}{\textwidth}{|lCCCCCCCC|}
    \hline
    \textbf{Bezeichnung} & LMP & c-\ce{Si} & c-\ce{SiO2} & a-\ce{Si} & \ce{SiH4} & \ce{+O2} & PVD\footnotemark[1] & CVD\footnotemark[2] \\
    \hline          % LAMMPS &  c-Si  & c-SiO2 &  a-Si  & Silane  & + O2  &  PVD   &  CVD   \\
    Al\_AlO\_AlN    & \cmark &        &(\cmark)& \cmark?& \cmark &        & \cmark & \cmark?\\
    chenoweth       &        &        &        &        &        &        &        &        \\
    kulkarni        & \cmark & \cmark & \cmark & \cmark?& \cmark &(\cmark)& \cmark & \cmark?\\
    lg              &        &        &        &        &        &        &        &        \\
    liu\_ettringite & \cmark &        & \cmark & \cmark?&        &        & \cmark & \cmark?\\
    liu\_nitramines &        &        &        &        &        &        &        &        \\
    narayanan       & \cmark &        & \cmark & \cmark?&        &        & \cmark & \cmark?\\
    newsome         & \cmark &        &(\cmark)& \cmark?&        &(\cmark)& \cmark & \cmark?\\
    nielson         & \cmark & \cmark & \cmark & \cmark?& \cmark &        & \cmark & \cmark?\\
    zhang           & \cmark &        &        &        & \cmark & \cmark &        &        \\
    \hline
  \end{tabularx}
  \todo[inline]{Gruende und Daten -> Anhang}
\end{table}

\footnotetext[1]{PVD: a-\ce{Si}-PVD mit Parsivald}
\footnotetext[2]{PVD: a-\ce{SiO2}-CVD mit Parsivald}

Schluss.


\subsection{presentation transcript}

