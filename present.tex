\section{Überblick über bisherige Methoden}

In der Vergangenheit wurden reine molekulardynamische und kinetische Monte Carlo-Methoden wiederholt zur Simulation von verschiedenen Aspekten von Gasphasenabscheidungen zu Rate gezogen.
Im Folgenden soll ein Überblick über die bisherigen Untersuchungen gegeben werden.

\subsection{Bisherige KMC-Simulationen}

\subsubsection{Dwivedi: gitterbasierte 2D-ALD}

\todo{Titel erwähnen?}Dr. V. Dwivedi und Prof. Dr. A. Adomaitis \cite{dwivedi_multiscale_2009}\cite{dwivedi_multiscale_2009-1}\cite{dwivedi_multiscale_2010} haben ein zweidimensionales KMC-Modell von \ce{Al2O3}-ALD genutzt, um die Reaktions- und Wachstumsraten in Abhängigkeit der Partialdrücke der Precursorgase zu gewinnen und in eine Gasflusssimulation zur Beschichtung von Mikroporen einzubinden.
Dazu werden die Plätze eines hexagonalen Gitters auf die Spezies Sauerstoff, Hydroxyl und Aluminium aufgeteilt, Zustandsübergänge in Abhängigkeit der 6 Nachbarzellen definiert und Übergangsraten ausgehend von den Energiebarrieren der entsprechenden chemischen Reaktionen ermittelt.
Aussagen über die Schichthöhe ergeben sich über eine Festlegung Zellhöhe anhand der Einheitszelle eines \ce{Al2O3}-Kristalles und eine vertikale Bestimmung der höchsten besetzten Zelle.

Damit lassen sich zwar GPC-Werte und Precursorverbrauch annähernd bestimmen, doch ist keine strukturelle Aussage möglich, obwohl elegant gewählte Ereignisse die Ziel-Stöchiometrie erzwingen.
\todo{Sollte das hier erwähnt werden?}Eigene Tests des Modelles schlugen zudem häufig fehl, da das System eine Tendenz zur Bildung von \ce{Al-O}-Ketten und zur Verstärkung von Kratern hat, wobei die Ursache nicht klar dem eigenen Code oder dem Dwivedi-Modell zuzuordnen ist.

\subsubsection{Mazaleyrat: gitterbasierte 3D-ALD}

Ähnlich zum Dwivedi-Modell nutzt auch das Mazaleyrat-Modell \cite{mazaleyrat_methodology_2005} ein Simulationsgitter, das aber in drei Dimensionen definiert ist, auf dem \ce{MgAl2O4}-Spinell-basiert und als Annäherung der eigentlichen Atompositionen verstanden wird.
Durch geschickte Rotation und Indizierung des Gitters wird eine effiziente Methode geschaffen, in ALD-Simulationen Aluminiumoxid-Schichten aufzuwachsen, deren strikte kristalline Konfiguration die eigentlich amorphe Struktur nicht darstellen kann, obwohl das Ziel des Modelles in der Abscheidung von \ce{Al2O3}-Dielektrika auf Silizium-Oberflächen liegt.

Eine Betrachtung der Dichte, Rauheit, dem Besetzungsgrad der Gitterplätze und dem Wachstumsverhalten über den dritten Schritt hinaus war von den Autoren nicht vorgesehen.

\subsubsection{Stamatakis: Zagros-KMC-Code}

Zwar handelt es sich beim Zagros-Modell asd

-> Graphbasiert, Vorausgewählte Ereignisse, Gitter

\todo[inline]{schreiben}

\subsection{Bisherige MD-Simulationen}

relativ verlässliche strukturelle Aussagen möglich, aber keine Stufen oder größere Strukturen

ReaxFF -> Precursorsimulationen möglich

Refs

\todo[inline]{schreiben}

\subsection{Zielsetzung für Parsivald}

- Feature Scale (Mikrometer-Skala)

- MD-Genauigkeit

- Effizienz durch KMC-Modell
