\pagestyle{scrheadings}  % Standart  Kopf- und Fußzeile
\setkomafont{pageheadfoot}{\small\scshape} % for new Koma Script

% ---------------------------------------------------------------------
%\setcounter{secnumdepth}{2}
%\setcounter{chapter}{-1}
\setcounter{tocdepth}{2}

% ---------------------------------------------------------------------
%\sloppy    % weniger Worttrennungen, größere Wortabstände
\fussy      % viele Worttrennungen, "schönere" Wortabstände

% ---------------------------------------------------------------------
%\flushbottom      % Ausrichtung der Seitenenden jeweils auf
        % gleicher Höhe

% ---------------------------------------------------------------------
%\sloppypar    % Das hier relaxt die Einstellungen zum Wortabstand
      % extrem. Damit ragen keine Worte über den rechten
      % Zeilenabstand hinaus. Dafür muß stärker auf
      % Wortabstand geachtet werden, der kann dann ziemlich
      % groß werden. Man erhält aber keine Meldung mehr
      % über underfull boxes.

%Hiermit kann man das gleiche mit weniger Holzhammer erreichen:
%\setlength{\tolerance}{2000}           % Strafpunkt für Zeilenumbruch
%\setlength{\emergencystretch}{3pt}     % Soweit dürfen einzelne Worte mehr
          % auseinandergezogen werden
%\setlength{\hfuzz}{1pt}                % Macht den rechten Rand um bis zu 1pt
          % flatterig.

% ---------------------------------------------------------------------
%Vermeiden einzelner Zeilen am Ende einer Seite oder oben auf einer neuen Seite
\clubpenalty10000
\widowpenalty10000

% eigene Definitionen
\newcommand{\BigO}[1]{$\text{O}(#1)$}
\newcommand{\mergedcell}[2][l]{\begin{tabular}[#1]{@{}l@{}}#2\end{tabular}}
\newcommand{\cmark}{\ding{51}}
\newcommand{\ccross}{\ding{55}}
\newcommand{\angled}[1]{\rotatebox{70}{#1}}

\definecolor{Green}{rgb}{0.7 1.0 0.7}
\definecolor{Yellow}{rgb}{1.0 1.0 0.7}
\definecolor{Red}{rgb}{1.0 0.7 0.7}
\definecolor{lightgray}{gray}{0.9}

\newcolumntype{C}{>{\centering\arraybackslash}X}

\newcommand{\cG}[1]{\cellcolor{Green}#1}
\newcommand{\cY}[1]{\cellcolor{Yellow}#1}
\newcommand{\cR}[1]{\cellcolor{Red}#1}

\algnewcommand\algorithmicinput{\textbf{Input: }}
\algnewcommand\Input{\item[\algorithmicinput]}
\algnewcommand\algorithmicresult{\textbf{Result: }}
\algnewcommand\Result{\item[\algorithmicresult]}
\algnewcommand\algorithmicassumption{\textbf{Assumption: }}
\algnewcommand\Assumption{\item[\algorithmicassumption]}

\DeclareSIUnit\gpcc{\gram\per\cubic\centi\meter}
\newcommand{\oddrowcolors}[0]{\rowcolors{0}{white}{lightgray}}
