\section{Silizium-PVD}
\label{siliconpvd}

Anhand von Silizium-PVD sollen einerseits amorphe Schichten abgeschieden, andererseits zur \ce{SiH4}-\ce{O2}-CVD hingeführt werden.
Nach Voruntersuchungen verschiedener ReaxFF-Para\-metrisierungen werden Abscheidungen auf kristallinen Silizium-Substraten unterschiedlicher Schnittebenen simuliert.
ASD

\subsection{Potentialdateien}

Für die folgenden Simulationen werden ReaxFF-Potentiale mit der Aussicht genutzt, auch Reaktionen zwischen Oberflächenatomen und Precursormolekülen simulieren zu können.
Da die ReaxFF-Formulierung erst innerhalb des letzten Jahrzehntes an Popularität gewonnen hat\todo{ref auf van duin}, sind die verfügbaren Potentiale auf einzelne Probleme angepasst und oftmals noch nicht in den Potentialdatenbanken zu finden.
Tabelle \ref{tab:siliconpotentials} listet die im Rahmen der Arbeit untersuchten ReaxFF-Parametrisierungen auf.

\begin{table}
  \caption[Silizium-Potentialdateien]{Potentialdateien für Silizium- und Siliziumoxidsysteme}
  \label{tab:siliconpotentials}
  \oddrowcolors
  \begin{tabularx}{1\textwidth}{|lXc|}
    \hline
    \textbf{Bezeichnung} & \textbf{Anwendung \& Kommentare} & \textbf{Ref.} \\
    \hline
    Al\_AlO\_AlN    & \ce{Al}, \ce{Al2O3}, \ce{AlN}. Basiert auf einer Si-Parametrisierung                                      & \cite{plimpton_lammps_2014} \\
    chenoweth       & Zersetzung von Polydimethylsiloxane bei hohen Drücken und Temperaturen. Ergänzung von \ce{C-Si}-Bindungen & \cite{chenoweth_simulations_2005} \\
    kulkarni        & Reaktion von Sauerstoff mit \ce{OH}-terminierten Siliziumoxid-Oberflächen                                 & \cite{kulkarni_oxygen_2013} \\
    lg              & ``low gradients''. Siehe liu\_nitramines. Fehlerhafte Version aus LAMMPS                                  & \cite{liu_reaxff-lg:_2011} \\
    liu\_ettringite & Verspannung von Ettringit (\ce{Ca6[Al(OH)6]2(SO4)3 26H2O}). Basiert auf Si-Parametrisierung               & \cite{liu_development_2012} \\
    liu\_nitramines & Dichtebestimmung von Nitramin-Molekülen bei hohen Drücken. Dichte erhöht durch Van-der-Waals-Korrekturen  & \cite{liu_reaxff-lg:_2011} \\
    narayanan       & Präparation mit \ce{Li-Al}-Silikaten. Für Phasenübergänge von Eukryptit-Kristallen (\ce{LiAl[SiO4]})      & \cite{narayanan_reactive_2012} \\
    newsome         & Oxidation von \ce{SiC}-Oberflächen mit \ce{O2} und \ce{H2O} bei \SIrange{500}{5000}{\kelvin}              & \cite{newsome_oxidation_2012} \\
    nielson         & Reaktionskinetik an Metallkatalysatoren bei hohen Temperaturen                                            & \cite{nielson_development_2005} \\
    zhang           & Zersetzung energetischer Moleküle (Nitramin-Explosionen)                                                  & \cite{zhang_carbon_2009} \\
    \hline
  \end{tabularx}
\end{table}

\subsection{Voruntersuchungen}

In Ergänzung zu den bisherigen Voruntersuchungen, die hauptsächlich aus Optimierungen und Relaxierungen von Kristallen bestanden, kommen hier auch Simulationen der Precursormoleküle für \ce{SiO2}-CVD, \ce{SiH4} und \ce{O2}, sowie der Oberfläche und amorpher Strukturen hinzu, wie sie bereits in Abschnitt \ref{mdtests} vorgestellt wurden.
Diese zusätzlichen Untersuchungen haben umfassendere Aussagen über die Anwendbarkeit der Parametrisierungen für vollständige Abscheidungssimulationen zum Ziel.
Lassen sich auch Reaktionen zwischen Precursormolekülen und der Oberfläche zuverlässig simulieren, könnte man die Precursormoleküle in Parsivald direkt auf der Oberfläche aufbringen, statt sie auf ihre Zentralatome zu reduzieren.
Die Ergebnisse dieser Betrachtungen wurden in Tabelle \ref{tab:siliconpreresults} zusammen gefasst und im weiteren kurz diskutiert.

\begin{table}
  \begin{threeparttable}
    \caption[Ergebnisse der Silizium-Potential-Untersuchungen]{Zusammenfassung Ergebnisse der Untersuchung der Potenialuntersuchung für Silizium-Systeme.
      Detailliertere Ergebnisse finden sich im \todo[inline]{Anhang}
    }
    \label{tab:siliconpreresults}

    \oddrowcolors
    \begin{tabularx}{\textwidth}{|lCCCCCCCC|}
      \hline
      \textbf{Bezeichnung}    & LMP    & c-\ce{Si} & c-\ce{SiO2} & a-\ce{Si} & \ce{SiH4} & \ce{+O2} & PVD\tnote{a} & CVD\tnote{b} \\
      \hline                % & LAMMPS & c-Si      & c-SiO2      & a-Si      & Silane    & +O2      & PVD          & CVD          \\
      Al\_AlO\_AlN            & \cmark & ~         & (\cmark)    & \cmark?   & \cmark    & ~        & \cmark       & \cmark?      \\
      chenoweth               & ~      & ~         & ~           & ~         & ~         & ~        & ~            & ~            \\
      kulkarni                & \cmark & \cmark    & \cmark      & \cmark?   & \cmark    & (\cmark) & \cmark       & \cmark?      \\
      lg                      & ~      & ~         & ~           & ~         & ~         & ~        & ~            & ~            \\
      liu\_ettringite         & \cmark & ~         & \cmark      & \cmark?   & ~         & ~        & \cmark       & \cmark?      \\
      liu\_nitramines         & ~      & ~         & ~           & ~         & ~         & ~        & ~            & ~            \\
      narayanan               & \cmark & ~         & \cmark      & \cmark?   & ~         & ~        & \cmark       & \cmark?      \\
      newsome                 & \cmark & ~         & (\cmark)    & \cmark?   & ~         & (\cmark) & \cmark       & \cmark?      \\
      nielson                 & \cmark & \cmark    & \cmark      & \cmark?   & \cmark    & ~        & \cmark       & \cmark?      \\
      zhang                   & \cmark & ~         & ~           & ~         & \cmark    & \cmark   & ~            & ~            \\
      \hline
    \end{tabularx}
    \todo[inline]{Gruende und Daten -> Anhang}

    \begin{tablenotes}
      \item[a] PVD: a-\ce{Si}-PVD mit Parsivald
      \item[b] CVD: a-\ce{SiO2}-CVD mit Parsivald
    \end{tablenotes}
  \end{threeparttable}
\end{table}

\subsubsection{Kompatibilität mit der Molekulardynamiksoftware (LMP)}

Einige Potentialdateien sind aus verschiedenen Gründen nicht mit der aktuellen Version der LAMMPS-Bibliothek kompatibel, was sich in harten Abbrüchen des Programmes äußert und sie nicht nutzbar für weitere Untersuchungen macht.
Sie sollen hier der vollständigkeit halber Erwähnung finden.
Andere Dateien lassen sich zwar laden und benutzen, äußern jedoch regelmäßige Warnungen über ihre van-der-Waals-Parameter, die aber nicht zu sonstigen Fehlern führen und meist nur Stickstoff- oder Platzhalteratome\footnote{Typischerweise beinhalten ReaxFF-Dateien wechselwirkungsfreie Platzhalteratome unter dem Pseudo-Element \ce{X} zum Zweck des Ausschlusses einzelner Partikel aus der Simulation} betrifft.

\subsubsection{Kristalleigenschaften (c-\ce{Si}, c-\ce{SiO2})}

Diese Untersuchungen decken sich mit dem bisherigen Vorgehen:
Eine Kristallstruktur wird unterhalb des Schmelzpunktes relaxiert, minimiert und anschließend hinsichtlich Koordination, Dichte und Bindungslänge analysiert.
Alle untersuchten Parametersätze erzeugen die korrekte Koordinationszahl von 4.0, mit klaren Spitzen in den radialen Verteilungsfunktionen und nahezu passenden Bindungslängen (Abbildung \ref{fig:sisibondlengths}).
Der Newsome-Parametersatz generiert einen RDF-Peak vor der eigentlichen Bindungslänge, der der Vollständigkeit halber erwähnt sein soll, bei Bestimmung der Bindungslänge aus der RDF aber nicht über die Hauptbindungslänge dominiert.
Er führt bei amorphen Systemen jedoch zu einer Unterschätzung der Bindungslänge und Koordinationszahl (Tabelle \ref{tab:amorphoussilicon}).
Im Allgemeinen stimmen die Kristalleigenschaften gut mit den erwarteten Werten aus der Standardliteratur überein.
Anhang \ref{appendix:silicon} beinhaltet Abbildungen eines Vergleiches der Bindungslängen sowie eine exemplarische radiale Verteilungsfunktion des Kulkarni-Potentiales im kanonischen und im großkanonischen Ensemble.

\subsubsection{Amorphes Silizium (a-\ce{Si})}

Durch langsame Relaxation zufällig positionierter Siliziumatome wurde amorphes Silizium simuliert, das wie die Kristalle zuvor auf Dichte und Bindungslängen untersucht wurde.
Deren Werte variieren stärker als für kristalline Systeme, liegen jedoch nah am experimentell bestimmten Wert von \SI{2.3}{\gpcc}\cite{remes_optical_1998}, anstatt Cluster und Hohlräume zu bilden.

\subsubsection{Stabilität der Precursormoleküle (\ce{SiH4}, \ce{O2})}

\todo[inline]{continuehere}

\subsubsection{Reaktion der Precursormoleküle (\ce{SiH4 + O2})}

\subsubsection{Abscheidungssimulationen (PVD, CVD)}



\subsection{presentation transcript}

