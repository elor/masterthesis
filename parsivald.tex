\section{Parsivald-Modell}
\label{parsivald}

\todo[inline]{Soll ich die Zellen beschrieben?}

\subsection{Beschreibung}

Parsivald entstand 2012 als namenloses Resultat meiner Bachelorarbeit\cite{lorenz_entwicklung_2012} am Fraunhofer ENAS mit dem Ziel der Simulation von Atom\-lagen\-abscheidungs-Prozessen.
Das Programm war beschränkt auf die atomistische Simulation von Metall\-oxid-ALD mittels MEAM-Potentialen.
Hauptaufgabe von Parsivald ist es, Atomlagenabscheidungen auf großen Oberflächen per Kombination von KMC- und MD-Methoden mit atomistischer \todo{Genauigkeit}Präzision simulieren zu können.
Abbildung \ref{fig:parsivald-stephierarchy} stellt die Funktionsweise des Parsivald-Modelles vor, die im Wesentlichen aus Vor- und Nachbereitung sowie einer Prozessschleife besteht.
Innerhalb der Hauptschleife werden nacheinander in mehreren Prozessschritten mögliche Adsorptionen, Ligandenaustausch-Reaktionen oder Relaxationen mit KMC-Mechanismen ausgewählt, mittels Molekulardynamik lokal simuliert und die Ergebnisse schließlich in die globale Struktur zurückgeführt (Abbildung \ref{fig:parsivald-schema}).
KMC-Ereignisse sind in diesem Modell auf eine dünne Oberflächenschicht begrenzt, wo sie jeweils innerhalb eines Kastens beschränker Größe, der MD-Box, wirken (Abbildung \ref{fig:parsivald-schema}).
Das setzt zwar diffusionsarme Prozesse voraus, ermöglicht dafür jedoch die effiziente Beschreibung von Atomlagenabscheidungen mit vielen Millionen Atomen auf der Mikrometerskala.

\begin{figure}
  \centering
  \def\svgwidth{\textwidth}
  \input{img/parsivald-schema-flat.pdf_tex}
  \caption[Parsivald-Schema]{
    Auswahl und Durchführung eines Oberflächenereignisses, verteilt auf KMC und MD.
  }
  \label{fig:parsivald-schema}
\end{figure}

\begin{figure}
  \centering
  \def\svgwidth{\textwidth}
  \input{img/parsivald-stephierarchy.pdf_tex}
  \caption[Parsivald-Funktionsweise asd]{
    Schema der Funktionsweise des Parsivald-Modelles für allgemeine Prozessarten.
    \\
    Event-Typen unterscheiden sich durch Art der MD-Simulation (Relaxation, Abscheidung) oder Art des Precursors.
    Siehe auch Abbildung \ref{fig:parsivald-modes}.
  }
  \label{fig:parsivald-stephierarchy}
\end{figure}

Als Eingaben der Simulation dienen das Substrat, eine Sammlung verschiedener Prozessparameter wie Umgebungstemperaturen, Expositionszeiten und Abscheidungsmodi (Abbildung \ref{fig:parsivald-modes}), sowie MD-Befehlslisten mit Platzhaltern (MD-Maske) samt Potentialparametrisierung.
Die Konfiguration legt Umgebungseigenschaften, Abscheidungsart (Abbildung \ref{fig:parsivald-modes}), Größe des Simulations- und MD-Raumes und Laufzeitbedingungen des Hostprogrammes fest.
Die Ausgabe erfolgt kontinuierlich in Ereignis-Logs sowie nach jedem Prozess-Schritt in Form aller Atompositionen, aus denen die Schichteigenschaften bestimmt werden können.
Optional lassen sich auch einzelne Events zur späteren Neuberechnung zwecks Fehlersuche speichern.

Eine typische Abscheidungs-Simulation verläuft nach folgendem Schema:
Zuerst wird das Substrat aus einer Datei gelesen und periodisch auf die Größe des Simulationsraums erweitert.
Pro Precursorart und \todo{entweder oder}Halbzyklus wird eine Ereignismaske vorbereitet, die die physikalischen und numerischen Parameter inklusive der angepassten Steuerbefehle für die MD-Bibliothek zwischenspeichert.
Anschließend beginnt die Hauptschleife mit Schritt 1, der den ersten Halbzyklus mit Precursor A darstellt.
Es werden nun mögliche Ereignisorte auf der Oberfläche gesucht, per KMC-Algorithmus Ereignisse an diesen Orten ausgewählt und von MD-Workern simuliert.
Bei Erreichen des Timeouts startet der nächste Schritt oder ein neuer Zyklus.
Nach jedem Schritt werden alle Atome in einem atomistischen Speicherformat zur späteren Analyse auf Festplatte gespeichert.
Die Abscheidungssimulation endet nach einer vorgegebenen Anzahl an Zyklen oder vollständiger Füllung des Simulationsraumes mit Atomen.

Mit diesem Schema lassen sich auch CVD- und PVD-Prozesse simulieren, indem man sich auf einen Schritt pro Zyklus beschränkt und im CVD-Fall mehrere Ereignisarten gleichzeitig nutzt.
Der Zyklus-Timeout kann dann zur Kontrolle der Ausgabefrequenz genutzt werden.
Für ALD und CVD lässt sich die Abscheidungsrate aus der Reaktionskinetik abschätzen, während man für PVD-Simulationen die Rate fest entsprechend der Prozessparameter setzen kann.

\begin{figure}
  \captionsetup[subfigure]{singlelinecheck=false}
  \begin{subfigure}[t]{5.7cm}
    \def\svgwidth{\textwidth}
    \input{img/parsivald-modes-ald.pdf_tex}
    \subcaption{ALD-Modus}
  \end{subfigure}
  \hfill
  \begin{subfigure}[t]{4.7cm}
    \def\svgwidth{\textwidth}
    \input{img/parsivald-modes-cvd.pdf_tex}
    \subcaption{CVD-Modus}
  \end{subfigure}
  \hfill
  \begin{subfigure}[t]{3cm}
    \def\svgwidth{\textwidth}
    \input{img/parsivald-modes-pvd.pdf_tex}
    \subcaption{PVD-Modus}
  \end{subfigure}
  \caption[Parsivald-Modi]{
    Verschiedene Parsivald-Modi im Prozesszyklus.
    Atomistische und statistische Ausgaben erfolgen nach jedem Schritt.
    ALD nutzt einen Schritt pro Halbzyklus.
  }
  \label{fig:parsivald-modes}
\end{figure}

\subsection{Annahmen und Einschränkungen}

\begin{enumerate}
\item Alle Reaktionen finden auf der Oberfläche statt
\item Reaktionen sind zeitlich und räumlich getrennt
\item Oberflächendiffusion ist vernachlässigbar
\item Bulkdiffusion ist vernachlässigbar
\end{enumerate}

Die Beschränkung des Modelles auf diffusionsarme Oberflächenabscheidungen ergibt sich aus der Notwendigkeit, Bereiche des Simulationsraumes, in denen keine Reaktion statt findet, bei der MD-Simulation zu vernachlässigen.
Das betrifft die Gasphase und den größten Teil des Bulks ebenso wie abgeschirmte Bereiche der Oberfläche.
Für Prozesse, die die oben aufgelisteten Annahmen nicht unterstützen, degeneriert Parsivald zu einer reinen MD-Simulation, für die mit Binning und Nachbarschaftslisten effizientere Parallelisierungsmethoden existieren.
Ein beschränktes Maß an Oberflächendiffusion lässt sich jedoch mit  längeren Relaxationszeiten oder separaten Relaxations-Ereignissen annähernd behandeln.

Weiterhin ist Parsivald auf die zugrunde liegende Molekulardynamik beschränkt.
So lassen sich nur Systeme simulieren, für die Potentialparametrisierungen existieren, die sowohl Bulks als auch Oberflächen darstellen können, wie es bei EAM- und vielen ReaxFF-Potentialen der Fall ist.
Diese sind im allgemeinen \todo{schwammig}recht rechenaufwendig, weshalb große Systeme nicht mehr mit reiner Molekulardynamik \todo{phrasing}berechenbar sind und sich die Effizienz von Parsivald bemerkbar macht.

Werden zusätzlich Moleküle und deren Reaktionen mit Oberflächenliganden dargestellt, lassen sich Precursor-Oberflächen-Reaktionen direkt in Parsivald simulieren.
Andernfalls muss der Precursor über sein abzuscheidendes Zentralatom und zusätzliche Mechanismen wie explizite sterische Hinderung angenähert werden.
Die Suche nach Ereignisorten würde dann über exponierte Oberflächenatome und \todo{erklären oder besseres Wort finden}Revisionslisten statt über Oberflächenliganden angenähert.
Eine tatsächliche Anwendbarkeit beider Methoden muss für jeden Prozess einzeln abgeschätzt werden, da die nun fehlenden Precursorliganden Einfluss auf die abgeschiedene Struktur haben \todo{konkreter!}können.

\subsection{Erweiterung im Rahmen der Masterarbeit}

\todo[inline]{das ist wohl zu viel}
Seit Beginn dieser Arbeit wurde das Parsivald-Modell und seine Implementierung um PVD- und CVD-Modi (Abbildung \ref{fig:parsivald-modes}), eventgebundene sterische Hinderung zum Zweck der CVD-Simu\-lation, ein allgemeines Konfigurationsformat\todo{Anhang?} und MD-Befehle mit Platzhaltern (sogenannte MD-Masken) erweitert.
Intern kam die Unterstützung verschiedener atomistischer Dateiformate, die Einbettung der LAMMPS-Umgebungs\-variablen zur Potentialsuche, globale und lokale Suchpfade für alle Eingabedateien und ein interner sowie beliebig viele externe Workerpools dazu.
Ein standardisiertes Buildsystem sorgt für schnelle und sichere Kompilierung der Software, und eine Vielzahl externer \todo{ref oder Anhang}Werkzeuge zur Prozess-Vorbereitung und -Analyse, wie sie in \todo{Referenz auf das Kapitel, in dem die Methoden vorgestellt werden} vorgestellt werden, ermöglicht schnelle Zyklen der Parameteroptimierung und Auswertung.

Diese Änderungen ermöglichen eine einfachere Vorbereitung, Simulation und Untersuchung verschiedener Prozesse und eine aussagekräftigere Analyse der Ergebnisse.
Auch eine automatisierte Optimierung der Prozessparameter wie Temperatur, Druck und Expositionszeit oder MD-spezifischer Parameter wie Thermostatdämpfung, Relaxationsdauer oder Länge des Zeitschrittes ist mittels des Konfigurationsmechanismus' denkbar.

Zusätzlicher Aufwand musste beim Einkapseln der MD-Bibliothek LAMMPS betrieben werden (Abschnitt \ref{lammpssucks}).

\subsection{Ausgabewerte}

Parsivald-Simulationsläufe geben neben der atomistischen Struktur verschiedene Statistiken und Werte in Form von Daten- und Logdateien aus.
Das beinhaltet die Zahl der versuchten, erfolgreichen und fehlgeschlagenen Ereignisse, laufende Worker, abgeschirmte und deshalb zurückgestellte Ereignisse und die Anzahl aller Atome.
Optional lässt sich eine Verteilung der Häufigkeit eines Zugriffes auf Positionen in der xy-Ebene angeben, um bei ALD- und CVD-Prozessen die Auswahlkriterien von Ereignisorten überprüfen zu können, was aber in dieser Arbeit nicht präsentiert wird\todo{Ergänzung im Anhang?}.
Anhand der atomistischen Struktur lassen sich Oberflächenrauheiten, Porenverteilungen, Dichten, Schichtdicken und Eigenschaften eventueller Kristalle bestimmen.\todo{Verweis auf Anhang -> Methoden}

Eine Auswertung der Abbruchrate ist besonders wichtig, während man nach validen Prozessparametern sucht, da sich über sie Fehler in der Prozesskonfiguration oder in der gebildeten Struktur abschätzen lassen (Siehe Abbildung \ref{fig:copperparsivald}).
Ereignisse, die einem Abbruch der LAMMPS-Bibliothek unterliegen, ihre Zeitbegrenzung erreichen oder unzulässige \todo{was ist unzulässig?}Strukturen bilden, werden einem Worker zur erneuten Berechnung übergeben.
Schlagen sie auch bei wiederholten Versuchen fehl, werden sie endgültig als fehlgeschlagen markiert und abgebrochen.

Da auf diese Weise bereits ausgewählte Ereignisse verworfen werden und somit die Abscheidungsraten leicht unterschätzt werden, müssen die Prozessraten entsprechend angepasst werden.
PVD-Prozesse weisen häufig vergleichsweise konstante Abbruchraten \SI{<1}{\percent} auf, weshalb es genügt, den Zielwert der Abscheidungsrate um die Abbruchrate zu erhöhen, was theoretisch auch automatisiert durchgeführt werden kann.
Bei ALD- und CVD-Prozessen sind solche Korrekturen auch möglich, allerdings aufgrund der Abhängigkeit von Reaktionsorten nicht notwendig:
Vor Ausführung jedes Ereignisses wird dessen lokale Nachbarschaft auf Übereinstimmung mit den Kriterien überprüft.
Wurde in dieser Zeit ein anderes Ereignis in direkter Nachbarschaft abgeschlossen, verbietet die sterische Hinderung seiner Oberflächenliganden das zweite Ereignis und es wird verworfen.
Schlägt das erste Ereignis jedoch fehl, werden dessen Abhängigkeiten aufgehoben, das zweite Ereignis wird überprüft und mangels Oberflächenliganden zur Ausführung freigegeben.

Problematisch sind Situationen, in denen die Vielzahl der Ereignisse fehlschlägt, wie es bei strukturellen Fehlern oder schlecht gewählten Parametern der Fall sein kann, da hier die Abscheidungsraten nicht verlässlich korrigiert werden können.
Für die Fälle, für die das Modell gestaltet wurde, wird die obere Schranke der Abbruchrate mit \SI{10}{\percent} abgeschätzt\todo{wie abgeschätzt?}.

\subsection{Probleme mit LAMMPS}
\label{lammpssucks}
\todo[inline]{Abschnitt überdenken}

Der hervorragende LAMMPS-Code wurde als MD-Bibliothek genutzt, da er gut optimiert und auf Funktionsebene in eigene Programme integrierbar ist sowie ReaxFF-Potentiale im Standardformat unterstützt.
Damit ist LAMMPS \todo{faktisch}faktisch konkurrenzlos, allerdings stößt man bei der Nutzung auf verschiedene Probleme.

Ein hauptsächliches Problem besteht in der Interaktion mit MPI-Bibliotheken.
Es liegt nahe, für hochparallele Systeme wie Parsivald auf MPI zu setzen, um Tasks vom Host an seine Worker zu vermitteln.
LAMMPS nutzt jedoch intern \todo{lammps.cpp, Zeile 247}globale MPI-Calls, auch wenn nur kleine Systeme untersucht werden oder explizit auf MPI verzichtet wird.
Zwar wird LAMMPS statisch verlinkt, nicht aber die MPI-Bibliothek\footnote{LAMMPS liegt auch ein MPI-Platzhalter(Stub) für serielle Anwendungsfälle bei, der aber separat verlinkt werden muss, statt in LAMMPS hinein kompiliert zu werden}, so dass ein Umschreiben der exportierten Symbole nicht zum gewünschten Erfolg führt.
Dieses Problem lässt sich durch fork-Prozesse oder proprietäre Patches umgehen, ist aber keine sichere Lösung für umfangreiche Software.
Zwar lässt sich LAMMPS auch mit einem MPI-Stub kompilieren, jedoch führt dieser durch kollidierende Namen globaler Funktionen der notwendigerweise dynamisch verlinkten MPI-Bibliotheken unweigerlich zu Problemen.

Weiterhin ruft LAMMPS nach jedem Berechnungsfehler \texttt{exit} auf, wodurch der aktuelle Prozess ohne Möglichkeit der Fehlerkorrektur beendet und die Kommunikation mit dem Hostprozess hart unterbrochen wird.
Geschieht dies in einem MPI-Prozess, werden alle beteiligten Prozesse auf allen Knoten unter Annahme eines kritischen Fehlers beendet.
Unerwartetes Übertreten von Systemgrenzen, hohen Teilchengeschwindigkeiten oder Ausnahmesituationen durch fehlgeformte Potentiale verursachen regelmäßig und unvorhergesehen \texttt{exit}-Calls und beenden ungeplant die gesamte Simulation.
Eine mögliche Lösung wäre die Nutzung von Prozess-Forks, die vom eigentlichen MPI-Prozess überwacht werden, wie es in Workerpools der Fall ist.

Darüber hinaus fallen einige Rechnungen unerwartet in eine nichtreproduzierbare Dauerschleife innerhalb der LAMMPS-Bibliothek, die sich aufgrund deren blockierender Architektur nicht ohne separaten Prozess diagnostizieren oder beenden lässt.
Hier besteht die Lösung schlicht in einem hostseitigen Timeout der Verbindung zu jedem verbundenen Worker.
Eine unbestimmte Zeit nach dem Verbindungsabbruch startet sich der Worker selbsttätig neu.
Im Worst-Case-Szenario endet ein Großteil der Worker in Dauerschleifen, wodurch die Simulation signifikant ausgebremst wird, jedoch weiterhin stabil läuft.


Als Lösung der vorgestellten Probleme wurde das in Parsivald vorhandene Host-Worker-System so umgestaltet, dass LAMMPS in gekapselten Prozessen lokal oder auf entfernten Knoten läuft, aber nur die Anfangsbedingungen und Ergebnisse kommuniziert.
Worker werden in Pools organisiert, sind ansonsten aber unabhängig, was sich in der direkten Nutzung des Netzwerkstacks ohne Umweg über den Management-Prozess zeigt.
Auf der einen Seite ist die Stabilität der Worker unabhängig von anderen Workern, andererseits bleiben die eigentlichen Fehlerursachen weiterhin verborgen.
Stattdessen werden fehlgeschlagene MD-Simulationen erneut gestartet und bei wiederholtem Berechnungsfehler als fehlgeschlagen markiert und verworfen.

Vom Hostprozess werden für statistische Auswertungen Zähler der erfolgreichen und fehlerhaften MD-Simulationen geführt, mit denen Fehlerraten mit Prozess- und Struktureigenschaften korreliert werden können.
Eine Anwendung davon befindet sich in Abbildung \ref{fig:copperparsivald}.
