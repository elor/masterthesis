%%======================================================================
%%      Inhaltsverzeichnis
%%======================================================================
%\clearpage

%\cleardoubleemptypage

%\pdfbookmark{Inhaltsverzeichnis}{Inhaltsverzeichnis}
\tableofcontents
%\setcounter{tocdepth}{2}

%%======================================================================
%%      Abbildungsverzeichnis
%%======================================================================
%\cleardoublepage
\markboth{Abbildungsverzeichnis}{Abbildungsverzeichnis}
\listoffigures

%%======================================================================
%%      Tabellenverzeichnis
%%======================================================================
%\cleardoublepage
\markboth{Tabellenverzeichnis}{Tabellenverzeichnis}
\listoftables

%======================================================================
%  Literaturverzeichnis
%======================================================================
%\cleardoublepage
%\manualmark
%\markboth{Literaturverzeichnis}{Literaturverzeichnis}
%\bibliographystyle{unsrt}
%\bibliographystyle{own}
%\bibliography{literatur}

%%======================================================================
%%      Algorithmenverzeichnis
%%======================================================================
%\renewcommand{\listalgorithmname}{Algorithmenverzeichnis}
\addcontentsline{toc}{chapter}{Algorithmenverzeichnis}
\listofalgorithms

%%======================================================================
%%      Abkuerzungsverzeichnis
%%======================================================================
%\cleardoublepage
\chapter*{Abkürzungsverzeichnis}
\addcontentsline{toc}{chapter}{Abkürzungsverzeichnis}
\markboth{Abkürzungsverzeichnis}{Abkürzungsverzeichnis}
\def\listacronymname{Abkürzungsverzeichnis}

\newcolumntype{Y}{>{\flushleft\arraybackslash}X}

%\begin{acronym}[SQL]
% \acro{CNTs}{Kohlenstoffnanoröhrchen (engl. Carbon Nanotubes, CNTs)}
% \acro{DFT}{Dichtefunktionaltheorie}
%\end{acronym}

\def\arraystretch{1.5}
\begin{tabularx}{\linewidth}{lll}
ALD	&	Atomic Layer Deposition	&	Atomlagenabscheidung	\\
CVD	&	Chemical Vapor Deposition	&	Chemische Gasphasenabscheidung	\\
PVD	&	Physical Vapor Deposition	&	Physikalische Gasphasenabscheidung	\\
RDF	&	Radial Distribution Function	&	Radiale Verteilungsfunktion	\\
Parsivald	&	\multicolumn{2}{X}{Parallel Atomistic Reaction Simulator for Vapor and Atomic Layer Depositions}	\\
\end{tabularx}

%\begin{longtable}{lll}
%
%MOSFET & Metall-Oxid-Halbleiter-Feldeffekttransistor & metal-oxid-semiconductor field-effect transistor \\
%
%\end{longtable}


\chapter*{Symbolverzeichnis}
\addcontentsline{toc}{chapter}{Symbolverzeichnis}
\markboth{Abkürzungsverzeichnis}{Symbolverzeichnis}
\def\listacronymname{Symbolverzeichnis}

\begin{tabularx}{\linewidth}{ll}
\end{tabularx}

%\printglosstex(acr)

%%======================================================================
%%      Ende
%%======================================================================
\cleardoublepage
\pagenumbering{arabic} % Fäng erneut bei 1 an.
