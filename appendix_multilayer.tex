\chapter{Multilagen-PVD}
\label{appendix:multilayer}

\section{Porenbildung bei Unterrelaxation}

Während der Präparation von Kupfer-Nickel-Multilagen-Simulationen haben sich verschiedene defektbehaftete Strukturen gebildet.
Typischerweise deuten Wachstumsdefekte auf geringe Relaxationszeiten, geringe Auftreffenergien oder geringe Temperaturen hin.
Das folgende System wurde bei \SI{500}{\kelvin} mit \SI{5.4}{\electronvolt} Auftreffenergie pro Teilchen für \SI{1.2}{\nano\second} relaxiert.
Die Auftreffenergie wird jedoch vom Thermostat vor dem eigentlichen Auftreffen reduziert, so dass sie mit \SI{5.4}{\electronvolt} nach Untersuchungen der Schichtqualität eigentlich zu niedrig liegt.

Als Resultat bilden sich Poren (Abbildung~\ref{fig:multilayer_surfacefail}), die sich vergleichbar zu den Kupferkratern in Abbildung~\ref{fig:coppercrater} entwickeln, sich aber erst spät wieder schließen.

\begin{figure}[!ht]
  \centering
  \includegraphics[height=10cm]{CuNi_surface8_noalpha}
  \caption{Oberflächenprofil einer CuNi-Oberfläche nach nur 4 Lagen (\SI{60}{\angstrom})}
  \label{fig:multilayer_surfacefail}
\end{figure}

\clearpage
Abbildung~\ref{fig:multilayer_columnfail} zeigt ein gleichartiges Resultat, das mit denselben Parametern erzeugt wurde.
Hier ist erkennbar, wie sich die gebildeten Poren wieder schließen.
Zur einfacheren Veranschaulichung wurde nur ein wenige Nanometer dünnes Profil abgebildet.

\begin{figure}[!ht]
  \captionsetup[subfigure]{singlelinecheck=false}
  \def\subfigwidth{7cm}
  \begin{subfigure}[t]{\subfigwidth}
    \includegraphics[width=\textwidth]{CuNi_columnfail}
    \subcaption{Dünnes Profil nach 24 Lagen (\SI{180}{\angstrom})}
    \label{fig:multilayer_columnfail}
  \end{subfigure}
  \hfill
  \begin{subfigure}[t]{\subfigwidth}
    \includegraphics[width=\textwidth]{CuNi_thicklayers}
    \caption{Profil einer Schicht mit 6 Lagen je \SI{6}{\nano\meter}}
    \label{fig:multilayer_thickfail}
  \end{subfigure}
  \caption{Fehlgeschlagene \ce{CuNi}-Abscheidungen während der Parameter-Optimierung}
\end{figure}

\section{Multilagen mit einer Dicke von 5 nm}

Ergänzend wurden auch Untersuchungen an Schichten mit Lagendicken begonnen, die sich mehr an experimentellen Werten von mehreren Nanometern orientieren.
Abbildung~\ref{fig:multilayer_thickfail} stellt ein solches System dar, das aber aus Mangel an Rechenzeit für die notwendige Optimierung der Simulationsparameter nicht weiter untersucht wurde.
Aus diesem Grund sind auch Krater- und Porenbildungen zu beobachten, die erwartungsgemäß mit Eingabe der optimalen Werte aus Kapitel~\ref{multilayer} eliminiert werden.
