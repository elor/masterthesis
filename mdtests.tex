\section{Tests für MD-Potentiale zur Schichtabscheidung}
\label{mdtests}

Vor der Nutzung von Parametrisierungen von Wechselwirkungspotentialen für die Molekulardynamik sollten diese durch eine Reihe von Tests auf ihre Anwendbarkeit geprüft werden.
Diese Parametrisierungen sind jeweils für einen speziellen Zweck erstellt worden und decken meist nur diesen zuverlässig ab\todo{und manchmal nicht mal das}.
Das bedeutet einerseits, dass man Potentiale, die auf thermodynamische Probleme optimiert sind, nur selten zur Simulation struktureller Eigenschaften nutzen kann, und umgekehrt.
Andererseits stehen hinter jeder Parametrisierung besondere Zielsetzungen, die oft vom eigenen Einsatzgebiet abweichen und eigentlich eine Anpassung der Parametrisierung, mindestens jedoch ihre Prüfung zur Voraussetzung haben.
So unterstützen Potentiale zur Darstellung von Molekülen häufig keine Bulks und Oberflächen, und umgekehrt.

Dem Thema der Arbeit entsprechend, sollen die zu untersuchenden Systeme strukturelle und chemische Eigenschaften in beschränkten Temperaturbereichen darstellen, was sich auch in den Tests zeigt.
So wird die kristalline und amorphe Struktur, ihre Dichte, radiale Verteilungsfunktion, Bindungslängen und Koordination einerseits untersucht, und die Darstellbarkeit der zugrunde liegenden Reaktionen per ReaxFF-Formulierung sowie die Diffusion eines Atomes auf der Oberfläche andererseits.
Diese Tests sind in Tabelle \ref{tab:mdtestvariants} dargestellt.

\begin{table}
  \oddrowcolors
  \begin{tabularx}{\textwidth}{|lXX|}
    \hline
    \textbf{Test} & \textbf{Vorgehensweise} & \textbf{Ergebnis} \\
    \hline
    Kristall-Strukturoptimierung & LAMMPS minimize & Dichte, RDF, Koordination, Bindungslänge \\
    Kristall-Relaxation & LAMMPS nvt/npt run & Dichte, RDF, Koordination, Bindungslänge \\
    Oberflächen-Relaxierung & Relaxierung einer nichtperiodischen Struktur in unbegrenztem Raum  & Oberflächenstabilität bei geringem Druck \\
    Präparation amorpher Struktur & zufällige Atompositionen, Relaxation bei hohen Temperaturen, Abkühlung & Dichte, Koordination, Hauptbindungslänge \\
    Precursor-Simulation & LAMMPS nvt run jedes Precursormoleküles & Stabilität \\
    Precursor-Reaktionen & LAMMPS nvt run beider Precursormoleküle & Reaktivität, Stabilität \\
    Oberflächen-Reaktionen & Reaktion eines Moleküles mit einer präparierten Oberfläche & Reaktivität, Diffusionsgrad \\
    Parsivald-Simulation & Komplette Abscheidungs-Simulation mit Parsivald & Anwendbarkeit \\
    \hline
  \end{tabularx}
  \caption[asd]{dsa mit LAMMPS, Präparation mit Materials Studio}
  \label{tab:mdtestvariants}
\end{table}
