\section{Parsivald-Modell}
\label{sec:parsivald}

\subsection{Bisheriger Stand}

Parsivald entstand als Resultat meiner Bachelorarbeit\cite{lorenz_entwicklung_2012} am Fraunhofer ENAS in Chemnitz, wurde darin aber noch als namenlose Software entwickelt.
Das Modell wurde mit dem Ziel der Atom\-lagen\-abscheidungs-Simulationen entwickelt und beschränkte sich deshalb auf atomistische Simulation von Metall\-oxid-ALD mittels MEAM-Potentialen.
Im Hintergrund werden LAMMPS neben einer eigene KMC-Bibliothek und Importer und Exporter verschiedener Dateiformate, Kommandozeilenargumente und Konfigurationsdateien zur einfacheren Prozesserforschung zu einer mehrteiligen Software verbunden, die per Host-Worker-Mechanismus aus dem Simulationsraum wiederholt Simulationseinheiten extrahiert, mit MD-Methoden simuliert und zurückführt (Abbildung \ref{fig:parsivald-schema}).

\begin{figure}[tbh]
  \centering
  \def\svgwidth{\textwidth}
  \input{img/parsivald-schema-flat.pdf_tex}
  \caption[Parsivald-Schema]{
    Auswahl und Durchführung eines Oberflächenereignisses, verteilt auf KMC und MD.
  }
  \label{fig:parsivald-schema}
\end{figure}

Daraus ergibt sich eine linear skalierende, queue-basierte Parallelisierung mit bis zu tausenden parallellen MD-Simulationen zur atomistischen Betrachtung von Schichtabscheidungen auf der Mikrometer-Skala.

\begin{figure}
  \centering
  \def\svgwidth{\textwidth}
  \input{img/parsivald-stephierarchy.pdf_tex}
  \caption[Parsivald-Funktionsweise asd]{
    Datenabhängigkeit und Programmablaufplan des Parsivald-Modelles
  }
  \label{fig:parsivald-stephierarchy}
\end{figure}


Dazu wird der gesamte Wachstumsprozess in das Auftreffen eines Precursormoleküles oder Atomes und seine Reaktion beziehungsweise physikalische Anlagerung an der Oberfläche getrennt.
Für alle weiteren Prozesse wurden folgende Annahmen getroffen:

\begin{enumerate}
\item Alle Reaktionen finden auf der Oberfläche statt
\item Reaktionen sind zeitlich und räumlich getrennt
\item Oberflächendiffusion ist vernachlässigbar
\item Bulkdiffusion ist vernachlässigbar
\end{enumerate}

Reaktionen in der Gasphase sind also ebenso wenig darstellbar wie diffusionsgestützte Prozesse.
Ohne diese notwendigen Annahmen käme das Parsival-Modell einer reinen Molekular\-dynamik-Simu\-lation gleich.
Stattdessen werden Reaktionen auf folgende Weise modelliert:

Auf der Oberfläche des gesamten Simulationsraumes wird zuerst nach einem Ort für eine Precursorreaktion gesucht.
Dessen Auswahl kann rein zufällig sein, oder aber von der lokalen Nachbarschaft abhängen, beispielsweise über molekulare Gruppen oder Atomdichten.
Nach dessen Auswahl extrahiert man alle Atome in der unmittelbaren Nachbarschaft und erstellt aus ihnen eine Reaktionszelle für MD-Simulationen.
Diese Zelle enthält einen Rand fester Atome, dessen Größe wie auch die Größe der Zelle selbst von der Reichweite der genutzten MD-Potentiale abhängt.
In diese Zelle fügt man je nach Simulationsart, gewünschter Genauigkeit und Fähigkeit der Potentiale das Precursormolekül oder einzelne Atome daraus ein.
Nach einer erfolgreichen Reaktions- oder Relaxationssimulation fügt man die Atome der Reaktionszelle wieder in die Gesamtstruktur ein und wählt das nächste Oberflächenereignis aus.

Ergänzend lässt sich durch verschiedene Optimierungen die Laufzeit minimieren.
Beispielsweise kann man durch Parallelisierung des KMC-Teiles linearen Speedup \todo{Linearer Speedup?} erreichen, der nur durch die Größe des Raumes begrenzt ist.
Eine Parallelisierung der MD-Simulationen ist hingegen nicht sinnvoll, da bloß kleine Mengen von maximal einigen Tausend Atomen gleichzeitig simuliert werden.
Andererseits kann man durch oben diskutierte Datenstrukturen die Suche neuer KMC-Ereignisse sowie die Extraktion einer atomaren Nachbarschaft verkürzen.

\todo{fehlt noch was?}

\subsection{Einschränkungen}

Neben den durch die oben genannten Annahmen \todo{Referenz} eingeführten gibt es noch eine weitere Einschränkung.
So lassen sich nur Prozesse und Strukturen betrachten, die auch durch Molekulardynamik darstellbar sind, wobei sich sowohl das Bulk-Material als auch seine Oberfläche simulieren lassen müssen.
Im optimalen Fall stellen die genutzten MD-Potentiale auch noch Precursor-Oberflächen-Reaktionen verlässlich dar, so dass man auf weitere Annäherungen verzichten kann.
Darunter fallen die Reduktion des Precursors auf sein Zentralatom, explizite sterische Hinderung sowie eine ausführliche Vorauswahl der Reaktionsorte.

Falls sich der komplette Precursor oder seine Reaktion mit der Oberfläche nicht durch passende MD-Potentiale darstellen lässt, kann man auf die Abscheidung seines Zentralatomes zurück greifen.
Obwohl es sich um eine stark vereinfachte Methode einer chemischen Abscheidung handelt, lässt sich so wenigstens das Wachstum der Struktur beobachten.
Da sich so keine Hinweise auf zurück gebliebene Precursorliganden auf der Oberfläche finden, muss man zusätzlich die sterische Hinderung explizit modellieren, um freies Wachstum zu verhindern.

\missingfigure{steric-hindrance}

Sterische Hinderung verhindert normalerweise durch die lokale Oberflächendichte von Precursorliganden eine weitere Anlagerung von Precursormolekülen bei chemischen Abscheidungen, bis diese Liganden wieder von der Oberfläche entfernt wurden.
Als Ersatz wird eine kugelförmige Zone definiert, innerhalb derer weitere Precursorreaktionen auf der Oberfläche ausgeschlossen sind, bis der nächste Prozessschritt gestartet wurde.
Als entsprechender Parameter steht nun jedem möglichen Abscheidungsereignis der Radius dieser Kugel zur Verfügung.
Diese Methode ignoriert Feinheiten wie die genaue Anzahl an verbliebenen Precursorliganden, stellt jedoch eine gute\todo{wirklich gut?} Näherung dar.

\subsection{Implementierung}

Das Parsivald-Modell wurde ursprünglich im Rahmen meiner Bachelorarbeit \todo{Referenz} in Software umgesetzt.
Seither wurden verschiedene Funktionen ergänzt, wie beispielsweise Erleichterungen im Prozessdesign sowie der separate PVD-Modus.
Auch einfachere Möglichkeiten, eine bestimmte Zahl an Worker-Prozessen zu starten, vereinfacht die Nutzung der Software auf größeren Clustersystemen.

Grundlage der Implementierung bieten hauptsächlich die beiden Bibliotheken libenskmc sowie LAMMPS für KMC und MD, wobei libenskmc aus eigener Entwicklung stammt.
Man hätte an dieser Stelle auch die SPPARKS-Bibliothek für KMC-Simulationen einbinden können, die von den selben Autoren der LAMMPS-Bibliothek stammt.
Es hat sich jedoch heraus gestellt, dass SPPARKS nur unter großem Aufwand das Parsivald-Modell handhaben könnte, was vor allem in Hinblick auf häufige kritische Abbrüche der LAMMPS-Bibliothek nicht vertretbar war.

Auf eigenen Code wurde bei folgenden Systemen gesetzt:
Das Host-Worker-System mit angebundenen Netzwerk-Code musste geschrieben werden, da LAMMPS die Nutzung einer MPI-Bibliothek unterbindet.
Die Octrees zum räumlichen Binning wurden eng mit libenskmc verzahnt, weshalb auf eine externe Geometrie-Bibliothek verzichtet wurde.
DUMP-Substrate werden aufgrund der klaren Format-Definition durch eigenen Code eingelesen.
Ebenso erfolgt die Ausgabe der .xyz-Dateien über eigenen Code.
Im Gegensatz dazu werden .lmp-Substrate jedoch über lmpio eingelesen, einer IO-Bibliothek für .lmp-Dateien, die im Hintergrund LAMMPS nutzt, um Kompatibilität mit der Vielzahl an uneindeutigen Atomeigenschaften zu garantieren.
Um Konformität zwischen Kommandozeilenargumenten und Konfigurationsdateien zu garantieren, wurde auf die ArgumentParser-Bibliothek zurück gegriffen, die aus eigener Feder entstammt und in weiteren externen Projekten Verwendung findet.
Die Abhängigkeitsauflösung bei KMC-Ereignissen entstammt ebenfalls der eigenen Feder, ebenso wie der sonstige Binding Code zwischen LAMMPS und libenskmc, inklusive beider Parallelisierung-Codes (Threads und losgelöste Unterprozesse).

