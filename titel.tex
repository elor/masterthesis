%======================================================================
%  Titelseite
%======================================================================

% Wir setzten hier die Seitennummerierung auf Großrömisch, auch wenn diese
% nicht auf dem "Papier" erscheint. Es dient nur der internen Zählung der
% Seiten im hyperref-Paket, welches für die Seitennummerierung in den pdfs
% zuständig ist. Ohne diesen Trick kommen ein paar sinnlose Warnungen und
% die Seiten-Navigation im Adobe Reader kann durcheinander geraten.
\pagenumbering{Roman}


%%======================================================================
%% Schmutztitel
%%======================================================================
%\extratitle{
%  \usekomafont{disposition}\mdseries
%  \begin{center}
%    \Huge \dcsubject\\[1.5ex]
%    \hrule
%    \vspace*{\fill}
%    \includegraphics{TUC_deutsch_einzeile_CMYK}
%  \end{center}
%}

%%======================================================================
%% Titelkopf
%%======================================================================
\titlehead{
  \vspace*{-1.5cm}
  \begin{center}
    \raisebox{-1ex}{\includegraphics[scale=1.1]{img/logo_tu_tailored}}\\
    \hrulefill \\[1em]
    {\Large\dcdepart}\\[0.2em]
    \dcinstitute\\[0.5em]
%    {\Large\dcdepartsecond}\\[0.5em]
%    \dcprofsecond
  \end{center}
  \vspace*{0.1cm}
  \begin{center}
    \raisebox{-1ex}{\includegraphics[scale=1.2]{img/logo_fhg_tailored}}\\
    \hrulefill \\[1em]
    {\Large\dcinstituteext}\\[0.2em]
    \dcdepartext\\[0.5em]
  \end{center}
  \vspace*{0.5cm}
}



%%======================================================================
%% Subjekt
%%======================================================================
\subject{\huge\textnormal{\textsc{\dcsubject}}}


%%======================================================================
%% Titel
%%======================================================================
\title{\huge
  \dctitle
  \\
  \dcsubtitle
  \vspace*{-0.4em}
}

%%======================================================================
%% Autor des Dokumentes
%%======================================================================
\author{\dcauthortitle~\dcauthorfirstname~\dcauthorlastname}

%%======================================================================
%% Ort, Datum
%%======================================================================
\date{\dcplace, den \dcdate
}


%%======================================================================
%% Publishers
%%======================================================================
\publishers{
    \begin{tabbing}
      \hspace*{10.1em}\=\kill
      \hspace*{4.8em}{Prüfer:}\>\dcexaminerfirst\\[-0.2em]
                  \>{\small\dcproffirst}\\[0.2em]
                  \>\dcexaminersecond\\[-0.2em]
                  \>{\small\dcprofsecond}\\
\\
      \hspace*{4.8em}{Betreuer:}\>\dcadvisor\\[-0.2em]
                  \>{\small\dcinstituteadvisor}
    \end{tabbing}
  \vspace*{-7em}
}

%%======================================================================
%% bibliografische Angaben
%%======================================================================
\lowertitleback{
\textbf{\dcauthorlastname, \dcauthorfirstname}\\
\textit{\dctitle}\\
%\dcsubject,~\dcdepart\\
\dcsubject\\
\dcuni,~\ifcase\month\or
  Januar\or Februar\or März\or April\or Mai\or Juni\or
    Juli\or August\or September\or Oktober\or November\or Dezember\fi
    ~\number\year\\
Stichworte: \dckeywords
}

%%======================================================================
%% maketitle
%%======================================================================

\maketitle

%%======================================================================
%% Danksagung
%%======================================================================
%\thispagestyle{empty}
%\null\vfil
%\begin{center}
%\usekomafont{disposition}\textbf{Danksagung}
%\vspace{-.5em}\vspace{\parsep}
%%
%% Hier steht der Text für die Danksagung
%%
%\end{center}
%\par\vfil\null
%\cleardoubleemptypage

%%======================================================================
%%      Kurzfassung / Abstract
%%======================================================================
%\def\abstractname{Abstract} % Wenn der Text "Zusammenfassung" erscheinen
                             % soll, dann muß dies auskommentiert werden
