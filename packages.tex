\usepackage[ngerman]{babel}
\usepackage[utf8]{inputenc}
\usepackage[]{geometry}	% Seitenraender
\usepackage{
	acronym, % Verwaltung von Abkuerzungen
	bibgerm, % Deutsche Bibliographie
	calc, % Erweiterung der arithmetischen Funktionen in
	color, % epigraph,
	scrpage2, % Kopf- und Fußzeilen flexibel gestalten
	tabularx, % Blocksatzspalten
	ngerman,
	capt-of,
	longtable,
	amsmath,
	pdfpages,
	graphicx,
	todonotes,
	listings, % Quellcode
%	url,
}
\usepackage[rightcaption]{sidecap}

\setkomafont{disposition}{\normalcolor\bfseries}	
\lstloadlanguages{[LaTeX]TeX}

% Für schöne Darstellung von Algorithmen
\usepackage[german, algoruled, algochapter]{algorithm2e}
\usepackage{algorithmicx}

\graphicspath{{img/},{img-results/}}

\usepackage[plainpages=false,pdfpagelabels]{hyperref}
\usepackage{amssymb}
\usepackage{subfigure}
\usepackage{scrhack}

\usepackage{ext/user}

\usepackage{graphicx}
\usepackage{color}
\usepackage{transparent}
