\section{Silizium-PVD}

Mit dem Ziel der Simulation des \ce{SiH4}-\ce{H2O}-CVD-Prozesses habe ich zunächst Potentiale gesucht, die Siliziumkristalle darstellen können.
Diese Untersuchungen ergaben sich aus den Eigenschaften einiger ReaxFF-Parametrisierungen, entweder Moleküle oder Bulkmaterialien zu unterstützen.

\subsection{Potentialdateien}

\begin{table}[h]
  \caption[Silizium-Potentialdateien]{Potentialdateien für Silizium- und Siliziumoxidsysteme}
  \label{tab:siliconpotentials}
  \rowcolors{0}{white}{lightgray}
  \begin{tabularx}{1\textwidth}{|lXc|}
    \hline
    \textbf{Bezeichnung} & \textbf{Anwendung \& Kommentare} & \textbf{Ref.} \\
    \hline
    Al\_AlO\_AlN & \ce{Al}, \ce{Al2O3}, \ce{AlN}. Basiert auf einer Si-Parametrisierung                                         & \cite{plimpton_lammps_2014} \\
    chenoweth       & Zersetzung von Polydimethylsiloxane bei hohen Drücken und Temperaturen. Ergänzung von \ce{C-Si}-Bindungen & \cite{chenoweth_simulations_2005} \\
    kulkarni        & Reaktion von Sauerstoff mit \ce{OH}-terminierten Siliziumoxid-Oberflächen                                 & \cite{kulkarni_oxygen_2013} \\
    lg              & ``low gradients''. Siehe liu\_nitramines. Fehlerhafte Version aus LAMMPS                                  & \cite{liu_reaxff-lg:_2011} \\
    liu\_ettringite & Verspannung von Ettringit (\ce{Ca6[Al(OH)6]2(SO4)3 26H2O}). Basiert auf Si-Parametrisierung               & \cite{liu_development_2012} \\
    liu\_nitramines & Dichtebestimmung von Nitramin-Molekülen bei hohen Drücken. Dichte erhöht durch Van-der-Waals-Korrekturen  & \cite{liu_reaxff-lg:_2011} \\
    narayanan       & Präparation mit \ce{Li-Al}-Silikaten. Für Phasenübergänge von Eukryptit-Kristallen (\ce{LiAl[SiO4]})      & \cite{narayanan_reactive_2012} \\
    newsome         & Oxidation von \ce{SiC}-Oberflächen mit \ce{O2} und \ce{H2O} bei \SIrange{500}{5000}{\kelvin}              & \cite{newsome_oxidation_2012} \\
    nielson         & Reaktionskinetik an Metallkatalysatoren bei hohen Temperaturen                                            & \cite{nielson_development_2005} \\
    zhang           & Zersetzung energetischer Moleküle (Nitramin-Explosionen)                                                  & \cite{zhang_carbon_2009} \\
    \hline
  \end{tabularx}
\end{table}

