\chapter{Zusammenfassung und Ausblick}
\label{summary}

\section{Zusammenfassung}
%% {Was hab ich getan? - Parsivald}

Im Rahmen dieser Arbeit wurde ein bestehendes Hybrid-Modell zur atomistischen Simulation von Atomlagenabscheidungen mit Methoden der Molekulardynamik und der Kinetischen Monte Carlo-Simulationen um die Beschreibung allgemeiner Gasphasenabscheidungen sowie um die Möglichkeit der Nutzung reaktiver Kraftfelder erweitert.
Als Resultat entstand eine Software namens \textit{Parsivald}, mit der die atomistische Simulation von Gasphasenabscheidungen mit Substratgrößen bis zu \SI{200x200}{\nano\meter} ermöglicht wird, was bis zu \num{1e8} Atomen entspricht.
Dies wird durch Nutzung effizienter Datenstrukturen und einem Host-Worker-Schema der Parallelisierung ermöglicht.

%% {Ergebnisse Skalierung}

Anhand der Simulation eines Gold-PVD-Prozesses wurde das Skalierungsverhalten von Parsivald untersucht, wobei gezeigt werden konnte, dass Oberflächen bis zu \SI{0.1x0.1}{\micro\meter} effizient parallelisiert werden können.
Für sie ergibt sich ein linearer Speedup bis zu einer substrat- und potentialabhängigen kritischen Ereignisdichte, bei der auf bis zu \SI{40}{\percent} der Oberfläche gleichzeitig Ereignisse berechnet werden.
Bei Abscheidungssimulationen mit Schichtdicken von \SI{92}{\angstrom} ergibt das ohne spezielle Optimierungen eine Laufzeit von vier Tagen unter Nutzung von durchschnittlich \num{76} parallelen Prozessen.
Für größere Simulationsräume begrenzt der maximale Ereignisdurchsatz des seriellen Hauptprozesses die Zahl der gleichzeitigen Ereignisse, so dass die Parallelisierbarkeit zwar bei \SI{99.5}{\percent} liegt, allerdings nur in \SI{0.06}{\percent} des Simulationsraumes gleichzeitig Ereignisse berechnet werden.
Derart große Simulationsräume stellen ohnehin einen Ausnahmefall dar, für den mehrere tausend Prozessorkerne für eine effiziente Simulation notwendig sind.
Als längster realistischer Anwendungsfall konnte bisher eine Simulation von Silizium-PVD mit reaktiven Kraftfeldern, die ohne weitere Optimierungen drei Wochen Rechenzeit für eine Schicht der Größe \SI{200x200x80}{\angstrom} beanspruchte, demonstriert werden.
Somit konnte gezeigt werden, dass auch unter Nutzung rechenaufwendiger Parsivald Kraftfelder ein wertvolles Werkzeug zur effizienten Simulation von großflächigen Gasphasenabscheidungen ist.

%% {Was hab ich getan? - PVD}

Die Funktion des Parsivald-Modells wurde weiterhin anhand von Simulationen physikalischer Gasphasenabscheidungen von Gold, Kupfer, Silizium und einem Kupfer-Nickel-Multi\-lagen\-system demonstriert.
Die Ergebnisse dieser Simulationsrechnungen wurden mit denen anderer Simulationsmethoden sowie experimentellen Daten verglichen.

Für Gold-PVD zeigt sich dabei epitaktisches Wachstum auf monokristallinen Substraten, das auch bei Simulationen von Gold-PVD auf strukturierten Substraten beobachtet wurde.
Für Strukturen mit extremen Auftreffwinkel bis zu \SI{90}{\degree} formten sich zudem Nanoporen an groben Unebenheiten der Struktur, welche die Oberflächenrauheit dominierten, sich jedoch im Laufe der Simulation langsam schlossen und so Hohlräume innerhalb der Schicht bildeten.
Darin zeigt sich eine Schwäche des Parsivald-Programmes in der Bestimmung der Oberfläche und der damit verbundenen Ereignisorte, die für eine künstliche Abschirmung von Bereichen an steilen Hängen sorgen kann.
Somit sind weitere Anpassungen für die Darstellung beliebiger Auftreffwinkel notwendig, wie sie etwa bei CVD- und ALD-Prozessen vorkommen.

Anhand der Simulation von Kupfer-PVD, welche ebenfalls epitaktisches Wachstum zeigte, konnte demonstriert werden, wie sich nanoskopische Unebenheiten im Laufe des Schichtwachstums zu Hohlräumen abschließen, wodurch die Bildung größerer Unebenheiten verhindert wird.

In Abscheidungssimulationen von mehrlagigen Schichten per Kupfer-Nickel-PVD sind perfekte Übereinstimmungen mit molekulardynamischen Simulationen in Rauheit und Struktur bei stark verringertem Rechenaufwand zu beobachten.
Im direkten Vergleich konnte dabei gezeigt werden, dass der Parsivald-Ansatz bei gleichen Bedingungen und gleicher Laufzeit weniger als \SI{10}{\percent} der Rechenleistung benötigt.
Auch für dünne Lagen mit einer Dicke von nur \SI{1}{\nano\meter} sind dabei klare Lagengrenzen vorhanden, deren Ausprägung mit der Relaxationszeit, was ähnlichen Untersuchungen entspricht\cite{zhou_atomistic_1998}.

Schließlich wurde die Abscheidung amorpher Schichten anhand von Silizium-PVD mit reaktiven Kraftfeldern simuliert, wobei sich dichte, aber vergleichsweise rauhe Schichten ergaben.
Reaktive Kraftfelder wurden genutzt, da EAM-Formulierungen und N-Teilchen-Potentiale eher die Bildung von Kristallen beschreiben, die zu untersuchenden amorphen Strukturen hingegen auf Valenzbindungen und damit verbundenen Koordinationen basieren, welche für Feststoffe und Oberflächen von reaktiven Kraftfeldern modelliert werden.
Die Oberfläche der Schicht zeigt eine RMS-Rauheit von \SI{11.5}{\angstrom}, welche sich mit experimentellen Werten deckt, allerdings im untersuchten Zeitraum linear ansteigt.
Anhand der Oberflächenprofile ist Porenbildung als Ursache der Rauheit erkennbar, die sich mit erhöhter Relaxationszeit und -temperatur jedoch reduzieren lässt.

%% {Was hab ich getan? - CVD und ALD}

%% Die betrachteten Silizium-Kraftfelder wurden weiterhin hinsichtlich der Beschreibung der an Siliziumdioxid-CVD beteiligten Precursor-Reaktionen untersucht.
%% Dabei zeigt sich, dass chemische Reaktionen zwar beschrieben werden, aber mit den untersuchten Kraftfeldern und den genutzten Methoden nicht zuverlässig für Parsivald-Simulationen genutzt werden können.
%% So sind Reaktionen zwischen Silan und Sauerstoff in der Gasphase nur mit einer begrenzten Menge von Startbedingungen erfolgreich, während sich bei großen Mengen dieser Moleküle Clusterbildung mit unrealistischen Bindungsarten zeigt.

Bei der Simulation von Oberflächen-Reaktionen von Wasser mit Aluminiumoxid mit reaktiven Kraftfeldern kann jedoch eine Hydroxylierung beobachtet werden, die Übereinstimmung mit der Hydroxyl-Bedeckung aus ab initio-Methoden zeigt.
Allerdings können die benutzten ReaxFF-Parametersätze den zweite Precursor der \ce{Al2O3}-ALD, Trimethylaluminium, mangels passender Parameter für \ce{Al-C}-Bindungen nicht beschreiben, weshalb eine reaktive Abscheidungssimulation dieses ALD-Prozesses mit den vorgestellten Methoden bisher nicht erfolgreich war.

\section{Ausblick}
%% {Was kann ich tun? - konkrete Untersuchungen}
Anknüpfungspunkte an die Arbeit bestehen in der Erweiterung von Parsivald um effizientere Algorithmen und Methoden zur weiteren Senkung der Laufzeiten von Abscheidungssimulationen sowie in der Anwendung von Parsivald zur Simulation verschiedener Gasphasenabscheidungen.

%% Parsivald-Erweiterungen

Der Ereignisdurchsatz des Hauptprozesses ließe sich verringern, indem das Host-Worker-Protokoll optimiert wird, wodurch die obere Grenze der Zahl paralleler Worker und somit die maximale Größe des Simulationsraumes für effiziente Simulationen weiter gesteigert wird.
Zusätzlich ermöglicht die Nutzung einer Delaunay-Triangulation des gesamten Raumes neben einer einfacheren Parallelisierbarkeit des Hauptprozesses auch eine implizite Beschreibung der Oberflächenatome per Alpha-Form, womit die Suche bei der Erstellung von Ereignissen massiv beschleunigt würde.
Durch diesen Graph-basierten Ansatz sind außerdem die Vorausberechnung ähnlicher Ereignisse möglich, die zu einer weiteren Beschleunigung des Prozesses und zur Verringerung der Reichweiten von Adsorptionsereignissen führt.
Ebenso würde die Ereignisauswahl vereinfacht, da die unmittelbare Nachbarschaft in der Delaunay-Triangulation direkt referenziert ist, sodass umfangreichere Ereignis-Bedingungen, wie sie bei reaktiven CVD-Simulationen notwendig werden, beschrieben werden können.
Die anschließende Einbindung von Parsivald in das ACCELERATE-Projekt führt zu einer Sammlung von numerischen Werkzeugen für die Multiskalen-Simulation von Gasphasenabscheidungen, wobei Parsivald von effizienteren Methoden zur Bestimmung und Durchführung von Ereignissen profitieren kann.
Seine Off-Lattice-Datenstrukturen können hingegen anderer Software wie Zacros\cite{stamatakis_zacros_2014} zur effizienten Beschreibung amorpher Strukturen verhelfen.

%% Weitere Optimierungen und Prozesse
Wie in der Arbeit demonstriert wurde, lässt sich Parsivald bereits jetzt zur effizienten Simulation von physikalischen Gasphasenabscheidungen nutzen, wobei die Simulation der PVD auf polykristallinen und amorphen Substraten bessere Vergleichbarkeit mit Experimenten ermöglicht\cite{adamov_electrical_1974}.
Es lassen sich auch gemischte Systeme wie beispielsweise Legierungen und verschiedene Multilagen-Systeme in Simulationen physikalischer Gasphasenabscheidungen untersuchen.

Die reaktive Simulation von chemischen Gasphasenabscheidungen ist mit dem Parsivald-Modell theoretisch möglich, konnte aber noch nicht demonstriert werden, weshalb die Untersuchung passender ReaxFF-Parametersätze notwendig ist.
Damit wäre auch die Simulation von mehrstufigen Abscheidungen, wie sie etwa in der Mikroelektronik genutzt werden\cite{grannemann_thin_1993}, denkbar, bei denen viele verschiedene Materialien auf atomarer Ebene per MD beschrieben werden müssten.

Unter Nutzung verschiedener Schichtarten ist auch die Simulation des anfänglichen Schichtwachstums auf verschiedenartigen Substraten möglich.
Zwar wurde mit Kupfer-Nickel-Multi\-lagen-PVD bereits eine solche Untersuchung durchgeführt, doch zeigen diese Materialien gute Hafteigenschaften.
Interessant wäre besonders das Wachstum von High-$\kappa$-Dielektrika auf Silizium und Siliziumdioxid, wie es bei der Produktion moderner MOSFETs auftritt.
Hierbei ist die Diffusion der Atome auf der Oberfläche und in sie hinein zu beachten, die mit Parsivald nur bedingt darstellbar ist.
