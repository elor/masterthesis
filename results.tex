\chapter{Ergebnisse}
\label{results}

Unter Nutzung der in Kapitel~\ref{theory} und Kapitel~\ref{models} vorgestellten Methoden werden im folgenden Kapitel Abscheidungssimulationen mit dem Ziel durchgeführt, die Präparation, Durchführung, Möglichkeiten und Einschränkungen des Parsivald-Modelles für PVD und CVD zu prüfen.

Dazu werden in eigenständigen Abschnitten verschiedene Systeme auf unterschiedliche Aspekte untersucht:
Abschnitt~\ref{goldpvd} zeigt anhand von Gold-PVD allgemeine Voruntersuchungen für einen Parametersatz, strukturelle Unterschiede zwischen realen und simulierten Schichten, Schichtabscheidung auf strukturierten Substraten sowie die praktische Skalierbarkeit der Simulation.
Der darauf folgende Abschnitt~\ref{copperpvd} stellt Voruntersuchungen unterschiedlicher Parametersätze, Unebenheiten an der Oberfläche und deren automatische Schließung anhand von Kupfer-PVD dar.
Anschließend werden Kupfer-Nickel-Multilagensysteme in Abschnitt~\ref{multilayer} zur Parsivald-Simulation vorbereitet und mit LAMMPS-simulierten Systemen verglichen.
In Abschnitt~\ref{siliconpvd} werden Silizium-PVD-Systeme mit dem Ziel der Abscheidung amorpher Schichten untersucht.
Zusätzlich werden dort Voruntersuchungen für Siliziumoxid-CVD durchgeführt.
Abschnitt~\ref{aluminaald} kehrt kurz zum Thema der Bachelorarbeit zurück und untersucht TMA und Wasser als Precursoren der \ce{Al2O3}-ALD.
%% Zuletzt werden in Abschnitt~\ref{silicacvd} Precursor-Reaktionen und die darauf basierende Siliziumoxid-CVD behandelt.

%% \section{Gold-PVD}

Als Testsystem für PVD-Prozesse bietet sich Gold als ein System an, dessen Abscheidungsprozess zwar durch Oberflächendiffusion dominiert wird, jedoch kristalline Strukturen bildet und für das ausgiebig erforschte MD-Parametrisierungen existieren.

\subsection{Voruntersuchungen}

Als kristallines, unäres Abscheidungssystem wurden zur Vorbetrachtung die Bindungslängen, Dichten sowie Koordinationen der kristallinen Phase untersucht\todo{Oberfläche?}.
Eine Stabilitätsanalyse des Precursormoleküles entfällt, da bei PVD-Prozessen nur einzelne Atome auf die Oberfläche aufgebracht werden.
Die Ergebnisse dieser Untersuchungen, die erwartungsgemäß gute Übereinstimmung zwischen Simulation und Experiment zeigen, sind in Tabelle \ref{tab:goldpreresults} zusammengefasst.

\begin{table}[hbtp]
  %% \rowcolors{0}{white}{lightgray} 
  \caption[Eigenschaften von Gold]{Eigenschaften von Gold als Voruntersuchung des PVD-Prozesses. Die Abweichung der Koordinationszahl wird in der Nichtperiodizität der untersuchten Struktur vermutet}
  \label{tab:goldpreresults}
  \begin{tabularx}{\textwidth}{|XXXX|}
    \hline
    \textbf{unters. Größe} & \textbf{Simulation} & \textbf{Experiment} & \textbf{Abweichung}\\
    \hline
    Bindungslänge & 2.879 \AA & 2.884 \AA & 0.2\% \\
    Koordination & 11.81 & 12.00 & 1.6\% \\
    Dichte (300K) & 18.99 g/cm$^3$ & 19.30 g/cm$^3$ & 1.6\%\\
    Dichte (500K) & 18.89 g/cm$^3$ & 19.13 g/cm$^3$ & 1.2\%\\
    \hline
  \end{tabularx}
\end{table}


\subsection{Prozess-Simulation}

Mangels chemischer Reaktionen genügt es, die Auftrefforte der Goldatome zufällig gleichverteilt auf der xy-Ebene auszuwählen.

\subsubsection{Strukturierte Substrate}

Als nächste Stufe wurden strukturierte Substrate untersucht, von denen eine Auswahl in Abbildung \ref{fig:goldsubstrate} dargestellt sind.
Diese wurden per Materials Studio präpariert, per Atomsk in ein kompatibles Format überführt und anschließend von Parsivald eingelesen und mit einzelnen Goldatomen beschichtet.

\begin{figure}[bt]
  \captionsetup[subfigure]{singlelinecheck=false}
  \def\subfigwidth{0.31\textwidth}
  \begin{subfigure}[t]{\subfigwidth}
    \includegraphics[width=\textwidth]{Au_substrate_flat}
    \subcaption{Flaches Gold-Substrat}
    \label{fig:goldsubstrate-a}
  \end{subfigure}
  \hfill
  \begin{subfigure}[t]{\subfigwidth}
    \includegraphics[width=\textwidth]{Au_substrate_step30}
    \subcaption{Gold-Stufe, 30°}
    \label{fig:goldsubstrate-a}
  \end{subfigure}
  \hfill
  \begin{subfigure}[t]{\subfigwidth}
    \includegraphics[width=\textwidth]{Au_substrate_tip60}
    \subcaption{Gold-Spitze, 60°}
    \label{fig:goldsubstrate-a}
  \end{subfigure}
  \caption[Strukturierte Goldsubstrate]{Goldsubstrate mit unterschiedlicher Struktur und Breite und Tiefe von 100 \AA.
    Abscheidungen wurden auf flachen Substraten sowie Stufen und Spitzen mit jeweils 15°, 20°, 30°, 45°, 60° und 90° Neigung durchgeführt.}
  \label{fig:goldsubstrate}
\end{figure}

\begin{figure}[bt]
  \captionsetup[subfigure]{singlelinecheck=false}
  \def\subfigwidth{0.31\textwidth}
  \begin{subfigure}[t]{\subfigwidth}
    \includegraphics[width=\textwidth]{Au_deposition_flat}
    \subcaption{Abscheidung auf flachem Gold-Substrat}
    \label{fig:goldsubstrate-a}
  \end{subfigure}
  \hfill
  \begin{subfigure}[t]{\subfigwidth}
    \includegraphics[width=\textwidth]{Au_deposition_step30}
    \subcaption{Abscheidung auf Gold-Stufe, 30°}
    \label{fig:goldsubstrate-a}
  \end{subfigure}
  \hfill
  \begin{subfigure}[t]{\subfigwidth}
    \includegraphics[width=\textwidth]{Au_deposition_tip60}
    \subcaption{Abscheidung auf Gold-Spitze, 60°}
    \label{fig:goldsubstrate-a}
  \end{subfigure}
  \caption[Abscheidung auf strukturierten Substraten]{
    Ergebnis der Abscheidung.
    Die Substratstruktur bleibt erkennbar, wird aber nach oben verstärkt, ansonsten aber kristallin und flach fortgesetzt.
  }
  \label{fig:golddepositions}
\end{figure}

Die Ergebnisse der Gold-Abscheidungen mit Parsivald (Abbildung \ref{fig:golddepositions}) zeigen perfekt fortgesetzte Kristallstrukturen, wobei die Schicht auf dem flachen Substrat nach 10 Kristall-Lagen eine Rauheit von einem Atomdurchmesser zeigt, die weiter beibehalten wird.
Die strukturierten Substrate hingegen zeigen den Trend, die Neigungswinkel an Stufen und Spitzen zu verstärken.
Nach längeren Laufzeiten entstehen somit Überhänge, die durch Abschluss zu Hohlräumen führen, die sich in der Realität durch thermische Relaxation schließen.
Dahinter steht einerseits die Notwendigkeit, Gold-Atome bei Ankunft auf der Oberfläche diffundieren zu lassen, was beim aktuellen Modell nur in Grenzen angewandt wird.

Andererseits steckt dahinter ein methodischer Fehler bei Nutzung von Binning-Methoden:
Die Oberfläche wird aufgrund von Laufzeitbegrenzungen nur entlang der z-Achse bestimmt, woraufhin das neue Atom über einem Atom auf der Oberfläche platziert wird.
Das führt bei Stufen in der Struktur zu Atomen, die immer am oberen Ende einer Kante oder Neigung aufgetragen werden und dort mit statistischer Wahrscheinlichkeit verbleiben.

Eine mögliche Lösung stellt die ausführliche Parametrisierung der Oberfläche dar, beispielsweise per Alpha-Form (Abschnitt \ref{data}, über die man die Ereigniswahrscheinlichkeit entsprechend der Einbettungsenergie, angenähert über die Oberflächenkrümmung, variierte.

%% \clearpage
%% \section{Kupfer-PVD}
\label{copperpvd}

Ein zweites PVD-System stellt Kupfer dar, für das eine Vielzahl an unterschiedlichen Parametrisierungen vorliegt (Tabelle \ref{tab:copperpots}).
Es stellt sich die Aufgabe, darunter eine passende Parametrisierung zu suchen und zu entscheiden, inwiefern eine Vorauswahl anhand weniger Parameter \todo{chword} möglich ist.

\begin{table}[hbtp]
  \caption[EAM-Parametrisierungen für Kupfersysteme]{EAM-Parametrisierungen für Kupfersysteme.}
  \label{tab:copperpots}
  \rowcolors{0}{white}{lightgray}
  \begin{tabularx}{\textwidth}{|lXc|}
    \hline
    \textbf{Bezeichnung} & \textbf{Anwendung \& Kommentare} & \textbf{Ref.} \\
    \hline
    CuAg.eam.alloy & Strukturelle und thermische Eigenschaften von \ce{Cu-Ag} & \cite{williams_embedded-atom_2006} \\
    cu\_ag\_ymwu.eam.alloy & Mono-, Di-, Trimere und Inseln von \ce{Cu} auf \ce{Ag} & \cite{wu_cu/ag_2009} \\
    Cu\_smf7.eam & Oberflächen von \ce{Ni-Cu}-Legierungen bei \SI{800}{\kelvin} & \cite{foiles_calculation_1985} \\
    Cu\_u3.eam & Oberflächen und Bulks verschiedener Legierungen & \cite{foiles_embedded-atom-method_1986} \\
    Cu\_u6.eam & Aktivierungsenergie für Eigendiffusionen & \cite{adams_self-diffusion_1989} \\
    Cu-Zr\_2.eam.fs & Flüssige und amorphe \ce{Cu-Zr}-Legierungen & \cite{mendelev_development_2009} \\
    Cu-Zr.eam.fs & Flüssige und amorphe \ce{Cu-Zr}-Legierungen & \cite{mendelev_using_2007} \\
    Mendelev\_Cu2\_2012.eam.fs & Unterkühlte \ce{Al-Cu}-Schmelzen. Basiert auf \cite{mendelev_analysis_2008} & \cite{_interatomic_2014} \\
    \hline
  \end{tabularx}
  
\end{table}

Viele der Parametersätze wurden für Legierungen angepasst, die Cu-Zr-Potentiale sind sogar mit der Warnung versehen, man könne ein reines Metall damit nicht mehr verlässlich simulieren.
Ein Hinweis auf die Kompatibilität mit LAMMPS oder den untersuchten Systemen ließ sich häufig nicht finden, weshalb Testrechnungen an Kupfer-Bulks und -Oberflächen durchgeführt wurden.

\subsection{Voruntersuchungen}

Mit Ausnahme der drei Parametersätze Cu\_u3.eam, Cu\_u6.eam und Cu\_smf7.eam konnten die Potentiale nicht von LAMMPS zur Simulation genutzt werden.
Es zeigten sich dabei Probleme beim Laden der Dateien, kryptische Fehlerausgaben nach einigen Schritten oder ein Aufhängen der Simulation ohne Vorankündigung.
Die Ergebnisse der verbliebenen Potentiale sind jedoch in guter Übereinstimmung mit Literaturwerten (Tabelle \ref{tab:copperpreresults}).
\todo[inline]{Dichte}
Im Gegensatz zu den ebenfalls durch EAM-Potentiale simulierten Goldsystemen wurde der Schmelzpunkt nicht zuverlässig simuliert (Abbildung \ref{fig:copperthermo}), was durch die sonst geringen Simulationstemperaturen vernachlässigbar ist.

\begin{table}[tbh]
  \rowcolors{0}{white}{lightgray} 
  \caption[Eigenschaften von Kupfer]{Vergleich der Eigenschaften von Kupfer mit experimentellen und Literaturdaten als Voruntersuchung des PVD-Prozesses\todo[inline]{ref}}
  \label{tab:copperpreresults}
  \begin{tabularx}{\textwidth}{|lXXXX|}
    \hline
    \textbf{unters. Größe} & \textbf{Experiment} & \textbf{Cu\_smf7.eam} & \textbf{Cu\_u3.eam} & \textbf{Cu\_u6.eam} \\
    \hline
    Koordination   &  \SI{12.00}{} & \SI{12.00}{} & \SI{12.00}{} & \SI{12.00}{} \\
    Bindungslänge  &  \SI{2.556}{\angstrom} & \SI{2.558}{\angstrom} (\SI{0.08}{\percent}) & \SI{2.558}{\angstrom} (\SI{0.08}{\percent}) & \SI{2.558}{\angstrom} (\SI{0.08}{\percent}) \\
    Dichte         & \SI{8.92}{\gram\per\cubic\centi\meter} & \SI{8.908}{\gram\per\cubic\centi\meter} (\SI{-0.13}{\percent}) & \SI{8.915}{\gram\per\cubic\centi\meter} (\SI{-0.06}{\percent}) & \SI{8.910}{\gram\per\cubic\centi\meter}  (\SI{-0.11}{\percent}) \\
    \hline
  \end{tabularx}
\end{table}

\todo[inline]{Oberflächenvalidierung?}

\begin{figure}[bht]
  \captionsetup[subfigure]{singlelinecheck=false}
  \def\subfigwidth{7cm}
  \begin{subfigure}[t]{\subfigwidth}
    \includegraphics[width=\textwidth]{Cu_u6_meltingpoint}
    \subcaption{Phasenübergang mit Cu\_u6.eam bei unterschiedlichen $t_\text{relax}$}
  \end{subfigure}
  \hfill
  \begin{subfigure}[t]{\subfigwidth}
    \includegraphics[width=\textwidth]{Cu_smf7_meltingpoint}
    \subcaption{Phasenübergang mit Cu\_smf7.eam bei unterschiedlichen $t_\text{relax}$}
  \end{subfigure}
  \caption[Abweichung der Schmelztemperaturen bei Kupfer-MD]{
    Abweichung der Schmelztemperatur mit verschiedenen Parametrisierungen.
    Experimentelle Werte von Brillo et al.\cite{brillo_density_2006}.
  }
  \label{fig:copperthermo}
\end{figure}

\subsection{Prozess-Simulation}

Aufgrund der Ähnlichkeit des Gold-PVD-Prozesses wurden dessen Parameter für die Kupfer-PVD übernommen und auf dessen Eigenschaften leicht angewandt.
So liegen kleinere Bindungslängen und geringere Massen vor, die beispielsweise zu erhöhten \todo{wirklich?}Auftreffgeschwindigkeiten führen.

Zu Beginn der Simulation ergeben sich hohe Abbruchquoten von \SI{25}{\percent}, die im Laufe der Simulation nachlassen.
Genauere Untersuchungen zeigen, dass bei Ankunft eines neuen Kupferatomes auf der glatten Gitteroberfläche ein vorhandenes Atom herausgeschlagen wird.
Sobald die Oberfläche mit genügend Off-Lattice-Atomen versehen ist, verschwindet dieser Effekt.
Die kritische Bedeckung liegt zwischen \SI{0.034}{\per\nano\meter\squared} und \SI{0.074}{\per\nano\meter\squared}, was sich mit der maximalen MD-Ereignisdichte von \SI{0.073}{\per\nano\meter\squared} deckt.
Es ist also zu vermuten, dass perfekte Gitterkonfigurationen nicht robust gegenüber gerichteten Energieeinträgen ist, kleine Perturbationen der Atompositionen aber zur gleichmäßigeren Verteilung der eingebrachten Energien führen.
\todo[inline]{Entropie?}

\begin{figure}[bth]
  \captionsetup[subfigure]{singlelinecheck=false}
  \def\subfigwidth{0.49\textwidth}
  \begin{subfigure}[t]{\subfigwidth}
    \includegraphics[width=\textwidth]{Cu_abortstatplot}
    \subcaption{Verlauf der Abbruchraten}
    \label{fig:copperparsivald-a}
  \end{subfigure}
  \hfill
  \begin{subfigure}[t]{\subfigwidth}
    \includegraphics[width=\textwidth]{missing}
    \subcaption{Zeitliche Entwicklung der Schichtdicke}
    \label{fig:copperparsivald-b}
  \end{subfigure}
  \begin{subfigure}[t]{\subfigwidth}
    \includegraphics[width=\textwidth]{missing}
    \subcaption{Zeitliche Entwicklung der Rauheit}
    \label{fig:copperparsivald-c}
  \end{subfigure}
  \hfill
  \begin{subfigure}[t]{\subfigwidth}
    \includegraphics[width=\textwidth]{Cu_eventhistogram}
    \subcaption{Häufigkeit gleichzeitiger Ereignisse}
    \label{fig:copperparsivald-d}
  \end{subfigure}
  \caption{Vergleich der Kupfer-Potentiale im Parsivald-Programm}
  \label{fig:copperparsivald}
\end{figure}

Im weiteren Verlauf der Simulation konvergiert die Abbruchquote gegen einen niedrigen Wert unterhalb von \SI{2}{\percent} (Abbildung \ref{fig:copperparsivald-a}).
Strukturelle Untersuchungen zeigen keine Einschlüsse, was auch bei diesem Prozess durch in der fcc-kristallinen Struktur der abgeschiedenen Schicht begründet ist.
Rauheit und Schichtdicke (Abbildungen \ref{fig:copperparsivald-b} und \ref{fig:copperparsivald-c}) \todo{SCHÖNER!}{sind schön}.

\subsubsection{Maximale Ereignisdichte}
Ergänzend ist in Abbildung \ref{fig:copperparsivald-d} ein Histogramm der Zahl paralleler Ereignisse dargestellt.
Die maximale Oberflächendichte der Ereignisse liegt bei der gewählten MD-Box-Größe von \SI{37x37}{\angstrom} bei \SI{0.073}{\per\nano\meter\squared}.
Dem stehen beobachtete Werte von 12 aktiven Ereignissen gegenüber, die einer Dichte von \SI{0.03}{\per\nano\meter\squared} und somit \SI{30}{\percent} maximaler Bedeckung entsprechen.
Aufgrund der zufälligen Positionierung der MD-Box innerhalb der KMC-Simulation und der Blocking von Ereignissen bei Überlappung der MD-Kästen scheint dieser Wert plausibel.
Ähnliche Werte werden bei Simulationsläufen mit unterschiedlichen Substratgrößen und Materialien beobachtet werden.

%% \clearpage
%% \section{Multilagen-PVD}
\label{multilayer}

\todoline{LAMMPS und Parsivald von einander abheben: Parsivald nutzt auch LAMMPS, LAMMPS steht nachfolgend für reine MD-Simulationen}

Mit PVD-Methoden können auch mehrlagige Schichten abgeschieden werden, wie sie für röntgenoptische oder magnetische Systeme (Riesenmagnetowiderstand GMR, Tunnelmagnetowiderstand) interessant sind.
Im folgenden Abschnitt soll am Beispiel von dünnen \ce{Cu-Ni}-Multilagen ein System näher untersucht werden, welches zwar einen GMR-Effekt zeigt, aber in der Praxis von \ce{Cu-Co}-Systemen aufgrund des stärkeren GMR-Effektes abgelöst wurde\cite{bird_giant_1995}.
Das Kupfer-Nickel-System wurde aufgrund ähnlicher Gitterkonstanten gewählt (\ce{Ni}:~\SI{3.52}{\angstrom}, \ce{Cu}:~\SI{3.61}{\angstrom}), die epitaktisches Wachstum ermöglichen und somit Fehlstellen unterbinden.
Durch die Ähnlichkeit zur Kupfer-PVD lässt sich der Prozess zudem auf den dort entwickelten Prozessparametern und den untersuchten Potentialparametersätzen aufbauen.

Im Experiment werden mehrlagige Kupfer-Nickel-Schichten per Elektro\-deposition\cite{yang_pulsed_1995} oder durch Sputtern\cite{cammarata_nanoindentation_1990} hergestellt, wobei üblicherweise Lagendicken im Bereich mehrerer Nanometer erzielt werden.
Dieses Vorgehen lässt sich direkt in Parsivald-Simulationen übertragen, in denen zugunsten der Rechenzeit in den folgenden Untersuchungen vergleichsweise dünne Lagen mit einer Dicke von \SI{1}{\nano\meter} abgeschieden wurden.
Anschließend werden diese auf Ähnlichkeit mit LAMMPS-präparierten Multilagen hinsichtlich ihrer Lagendicke und -rauheit untersucht.
Eine Auswertung der relativen Verteilung der Spezies entlang der Abscheidungsrichtung wird ergänzend für verschiedene Relaxationszeiten als Maß der Lagenqualität durchgeführt.
Abscheidungen von Lagen mit einer Dicke von \SI{6}{\nano\meter} wurden ebenfalls mit Parsivald simuliert, doch mangels verfügbarer Rechenzeit für vergleichbare reine LAMMPS-Simulationen nicht eingehender untersucht.

Wie bei den Gold-PVD-Simulation zuvor müssen für erfolgreiche Simulationen einige Simulationsparameter wie Relaxationszeit, Thermostatdämpfung und Substrattemperatur optimiert werden.
Als Zielgrößen für die Optimierung wurden zur Vermeidung von Fehlstellen die Rauheit der Oberfläche und die Qualität der einzelnen Lagen im Vergleich mit ähnlichen Untersuchungen\cite{zhou_atomistic_1998} gewählt.
In diesen Untersuchungen wurde bereits gezeigt, dass die kinetische Energie der einfallenden Atome einen erheblichen Einfluss auf die Qualität der einzelnen Lagen hat, weshalb gleichartige Untersuchungen nur hinsichtlich der Substrattemperatur durchgeführt wurden.

\subsection{Ergebnisse}

Parsivald-Simulationen erzeugen nach korrekter Parametereinstellung klar abgegrenzte epitaktische Atomlagen geringer Rauheit, die sich gut mit den Ergebnissen gleichartiger LAMMPS-Simulationen decken (Abbildung~\ref{fig:multilayerresults}).
RMS-Rauheiten um \SI{1.2}{\angstrom} stellen sich mit beiden Simulationsmethoden bis zur zehnten Lage ein und stimmen somit untereinander und mit den bisherigen Ergebnissen überein (Abbildung~\ref{fig:multilayerplots-a}).
Anhand der Schichten ist eine schwach korrelierte Rauheit der einzelnen Lagen erkennbar.

Zuvor war eine Anpassung der Temperaturen und Relaxationszeiten notwendig, die jedoch für LAMMPS und Parsivald gleichermaßen gelten.
Als Richtwert wurde die Qualität der einzelnen Lagen in Form des Anteils der Spezies in Abhängigkeit der Höhe über dem Substrat genutzt (Abbildung~\ref{fig:multilayerplots-b}).
Lagen schlechterer Qualität zeigen eine höhere Durchmischung der Schichten\todo{Joerg: was aber auch korrekt sein kann, je nach Mischbarkeit}, was wiederum zu einer Senkung der relativen Häufigkeit einer Spezies innerhalb ihrer Schicht führt, wie für die beiden Verteilungen bei einer Relaxationszeit von \SI{0.2}{\femto\second} pro Ereignis beobachtet werden kann.
Erst bei Verdopplung der Relaxationszeit bilden sich \todo{Joerg: hier sollte man vielleicht irgendwo mal erwähnen, dass man das für dieses System erwartet? Eigentlich müsste man da was zu Mischbarkeit und Phasendiagramm der Legierungsbildung sagen ...}klar abgegrenzte Lagen aus, wie sie in Abbildung~\ref{fig:multilayerresults} dargestellt sind.

In Anhang~\ref{appendix:multilayer} ist eine Auswahl von mehrlagigen Kupfer-Nickel-Schichten dargestellt, die durch Unterrelaxation verursachte strukturelle Fehler aufweisen.
Bei größeren Systemen ist zudem mit dem Auftreten von Verspannungen aufgrund der leicht unterschiedlichen Bindungslängen sowie mit der Entstehung von Gitterversetzungen und Fehlstellen zu rechnen, die allerdings durch Finite-Size-Effekte unterdrückt sein können.

\begin{figure}
  \captionsetup[subfigure]{singlelinecheck=false}
  \def\subfigwidth{7cm}
  \begin{subfigure}[t]{\subfigwidth}
    \includegraphics[width=\textwidth]{CuNi_layerroughness_comparison}
    \subcaption{
      Vergleich der Lagen-Rauheit (Abb.~\ref{fig:multilayerresults})
    }
    \label{fig:multilayerplots-a}
  \end{subfigure}
  \hfill
  \begin{subfigure}[t]{\subfigwidth}
    \includegraphics[width=\textwidth]{CuNi_atomdistribution_relax}
    \subcaption{Einfluss von $t_\text{relax}$ auf die Lagen-Qualität}
    \label{fig:multilayerplots-b}
  \end{subfigure}
  \caption[Rauheit und Qualität von Kupfer-Nickel-Multilagen]{
    Rauheit und Qualität von Kupfer-Nickel-Multilagen
  }
  \label{fig:multilayerplots}
\todoline{Joerg: etwas mehr Luft zwischen Kurven und Legende wäre hübsch}
\end{figure}

\begin{figure}
  \captionsetup[subfigure]{singlelinecheck=false}
  \def\subfigwidth{7cm}
  \begin{subfigure}[t]{\subfigwidth}
    \includegraphics[width=\textwidth]{CuNi_profile_LAMMPS_nice}
    \subcaption{Profil von \ce{Cu-Ni}-Multilagen, reine MD-Simulation mit LAMMPS}
  \end{subfigure}
  \hfill
  \begin{subfigure}[t]{\subfigwidth}
    \includegraphics[width=\textwidth]{CuNi_profile_Parsivald}
    \subcaption{Profil von \ce{Cu-Ni}-Multilagen mit Parsivald}
  \end{subfigure}
  \caption{Vergleich von Multilagen-Profilen mit LAMMPS und Parsivald}
  \label{fig:multilayerresults}
\todoline{Joerg: Es wäre sinnvoll gewesen, identische Systeme zu rechnen ...}
\end{figure}

\todoline{Joerg: Fazit der schönen Ergebnisse (Übereinstimmung LAMMPS und Parsivald)}
\todoline{Laufzeiten usw. schnell vergleichen}

%% \clearpage
\section{Silizium-PVD}
\label{siliconpvd}

Mit Silizium soll ein weiteres PVD-Material untersucht werden, das im Gegensatz zu den bisher untersuchten Metallen, welche ungerichtete Bindungen aufbauen und damit die kristalline Strukturen bevorzugen, auf gerichteten Bindungen beruht.
Deshalb werden reaktive Kraftfelder genutzt, welche auf einer expliziten Beschreibung gerichteter Bindungen über Bindungsordnungen benachbarter Atome basieren.

\subsection{Verfügbare ReaxFF-Parametrisierungen}

Da die ReaxFF-Formulierung erst innerhalb des letzten Jahrzehntes an Popularität gewonnen hat, sind die verfügbaren Potentiale auf sehr spezielle Probleme angepasst und unterstützen meist entweder Bulkmaterialien oder Reaktionen zwischen Molekülen.
Tabelle~\ref{tab:siliconpotentials} listet die im Rahmen der Arbeit untersuchten ReaxFF-Parametrisierungen auf, die in der Literatur gefunden werden konnten.

\begin{table}[hb]
  \caption{Untersuchte ReaxFF-Parametrisierungen für Silizium- und Siliziumoxidsysteme}
  \label{tab:siliconpotentials}
  \oddrowcolors
  \begin{tabularx}{1\textwidth}{|lXc|}
    \hline
    \textbf{Bezeichnung}  & \textbf{Anwendung \& Kommentare}                                                                          & \textbf{Ref.}                     \\
    \hline
    \pot{Al\_Al0\_AlN}    & \ce{Al}, \ce{Al2O3}, \ce{AlN}. Basiert auf einer Si-Parametrisierung                                      & \cite{plimpton_lammps_2014}       \\
    \pot{chenoweth}       & Zersetzung von Polydimethylsiloxane bei hohen Drücken und Temperaturen. Ergänzung von \ce{C-Si}-Bindungen & \cite{chenoweth_simulations_2005} \\
    \pot{kulkarni}        & Reaktion von Sauerstoff mit \ce{OH}-terminierten Siliziumoxid-Oberflächen                                 & \cite{kulkarni_oxygen_2013}       \\
    \pot{lg}              & ``low gradients''. Siehe liu\_nitramines. Fehlerhafte Version aus LAMMPS                                  & \cite{liu_reaxff-lg:_2011}        \\
    \pot{liu\_ettringite} & Verspannung von Ettringit (\ce{Ca6[Al(OH)6]2(SO4)3 26H2O}). Basiert auf Si-Parametrisierung               & \cite{liu_development_2012}       \\
    \pot{liu\_nitramines} & Dichtebestimmung von Nitramin-Molekülen bei hohen Drücken. Dichte erhöht durch Van-der-Waals-Korrekturen  & \cite{liu_reaxff-lg:_2011}        \\
    \pot{narayanan}       & Präparation mit \ce{Li-Al}-Silikaten. Für Phasenübergänge von Eukryptit-Kristallen (\ce{LiAl[SiO4]})      & \cite{narayanan_reactive_2012}    \\
    \pot{newsome}         & Oxidation von \ce{SiC}-Oberflächen mit \ce{O2} und \ce{H2O} bei \SIrange{500}{5000}{\kelvin}              & \cite{newsome_oxidation_2012}     \\
    \pot{nielson}         & Reaktionskinetik an Metallkatalysatoren bei hohen Temperaturen                                            & \cite{nielson_development_2005}   \\
    \pot{zhang}           & Zersetzung energetischer Moleküle (Nitramin-Explosionen)                                                  & \cite{zhang_carbon_2009}          \\
    \hline
  \end{tabularx}
\end{table}

\subsection{Voruntersuchungen}

In Ergänzung zu den bisherigen Voruntersuchungen, welche sich entsprechend der erwarteten Strukturen der abgeschiedenen Schichten auf die Beschreibung kristalliner Strukturen beschränkten, werden für die Silizium-Parametrisierungen auch die Eigenschaften amorpher Strukturen untersucht.
Die verwendeten Methoden wurden bereits in Abschnitt~\ref{mdmethods} vorgestellt.
Diese zusätzlichen Untersuchungen haben umfassendere Aussagen über die Anwendbarkeit der Parametrisierungen für vollständige Abscheidungssimulationen zum Ziel.
Die Ergebnisse dieser Betrachtungen sind in Tabelle~\ref{tab:siliconpreresults} zusammen gefasst und werden im Weiteren kurz diskutiert.

\begin{table}[th]
  \begin{threeparttable}
    \caption[Zusammenfassung der Voruntersuchungen für Silizium-Systeme]{
      Zusammenfassung der Voruntersuchungen für Silizium-Systeme.
      Siehe Anhang~\ref{appendix:silicon}
    }
    \label{tab:siliconpreresults}

    \oddrowcolors
    \begin{tabularx}{\textwidth}{|lCCCCCCC|}
      \hline
      \textbf{Bezeichnung}    & LMP\tnote{a} & c-\ce{Si} & c-\ce{SiO2} & a-\ce{Si} & \ce{SiH4} & \ce{+O2} & PVD\tnote{b} \\
      \hline                % & LAMMPS       & c-Si      & c-SiO2      & a-Si      & Silane    & +O2      & PVD          \\
      \pot{Al\_Al0\_AlN}      & \cmark       & ~         & (\cmark)    & \cmark    & \cmark    & ~        & \cmark       \\
      \pot{chenoweth}         & ~            & ~         & ~           & ~         & ~         & ~        & ~            \\
      \pot{kulkarni}          & \cmark       & \cmark    & \cmark      & \cmark    & \cmark    & (\cmark) & \cmark       \\
      \pot{lg}                & ~            & ~         & ~           & ~         & ~         & ~        & ~            \\
      \pot{liu\_ettringite}   & \cmark       & ~         & \cmark      & \cmark    & ~         & ~        & \cmark       \\
      \pot{liu\_nitramines}   & ~            & ~         & ~           & ~         & ~         & ~        & ~            \\
      \pot{narayanan}         & \cmark       & ~         & \cmark      & \cmark    & ~         & ~        & \cmark       \\
      \pot{newsome}           & \cmark       & ~         & (\cmark)    & ~         & ~         & (\cmark) & \cmark       \\
      \pot{nielson}           & \cmark       & \cmark    & \cmark      & \cmark    & \cmark    & ~        & \cmark       \\
      \pot{zhang}             & \cmark       & ~         & ~           & ~         & \cmark    & \cmark   & ~            \\
      \hline
    \end{tabularx}

 %% & CVD\tnote{b}
 %% & CVD
 %% & \cmark?
 %% & ~
 %% & \cmark?
 %% & ~
 %% & \cmark?
 %% & ~
 %% & \cmark?
 %% & \cmark?
 %% & \cmark?
 %% & ~

    \begin{tablenotes}[para]
      \item[a] LMP: Kompatibilität mit LAMMPS
      \item[b] PVD: a-\ce{Si}-PVD mit Parsivald
      %% \item[b] CVD: a-\ce{SiO2}-CVD mit Parsivald
    \end{tablenotes}
  \end{threeparttable}
\end{table}

\subsubsection{Kompatibilität mit der Molekulardynamiksoftware LAMMPS (LMP)}

Einige Potentialdateien sind aus unerfindlichen Gründen nicht mit der aktuellen Version der LAMMPS-Bibliothek kompatibel, was sich in harten Abbrüchen des Programmes äußert und sie von weiteren Untersuchungen ausschließt.
Andere Dateien lassen sich zwar laden und benutzen, äußern jedoch Warnungen über fehlerhafte van-der-Waals-Parameter, die aber nicht zu sonstigen Fehlern führen und meist nur Stickstoff- oder Platzhalteratome\footnote{ReaxFF-Parametrisierungen enthalten ein wechselwirkungsfreies Platzhalter-Element \ce{X} zum Zweck des Ausschlusses einzelner Atome aus der Simulation. Einige seiner Parameter werden von LAMMPS als fehlerhaft markiert.} betreffen.
Die Parametersätze \pot{chenoweth}, \pot{lg} und \pot{liu\_nitramines} können nicht mit LAMMPS genutzt werden.

\subsubsection{Kristalleigenschaften (c-\ce{Si}, c-\ce{SiO2})}

Diese Untersuchungen sind identisch zu den Untersuchungen der Kristallstrukturen aus den vorherigen Abschnitten.
Eine Parameterdatei gilt in dieser Hinsicht als erfolgreich, wenn eine Relaxierung der Kristallstruktur unterhalb der Schmelztemperatur von \SI{1687}{\kelvin}\cite{haynes_crc_2011} die Gittereigenschaften bewahrt, wofür die \todo{Diagramm mit den Dichten}Dichten und radialen Verteilungsfunktionen sowie die daraus gewonnenen \todo{Diagramm mit Koordinationszahlen und Bindungslängen}Koordinationszahlen und Bindungslängen verglichen werden.
Dabei überwiegen die Formen der radialen Verteilungsfunktionen, die nach dem langsamen Herunterkühlen der erhitzten Struktur wieder kristalline Eigenschaften zeigen sollten.
Dies geschieht allerdings nur bei \pot{kulkarni} und \pot{nielson}, wohingegen die anderen Parametrisierungen auch bei niedrigen \todo{wie niedrig waren die Experimente?}Temperaturen zu einer Verformung des Gitters hin zu amorphen Systemen neigen.
\todo{nicht auf Anhang verweisen?}\todo{vor allem: Informationen in den Anhang schreiben!}Detaillierte Informationen zu den Tests sind in Anhang~\ref{appendix_silicon} zu finden.

\subsubsection{Amorphes Silizium (a-\ce{Si})}

Durch langsame Relaxation zufällig positionierter Siliziumatome wurde amorphes Silizium generiert, das wie die Kristalle zuvor auf Dichte und Bindungslängen untersucht wurde.
Deren Werte variieren für amorphes Silizium stärker als für kristallines, liegen jedoch mit maximal \SI{4}{\percent} nah am experimentell bestimmten Wert für dünne Schichten von \SI{2.3}{\gpcc}\cite{remes_optical_1998}.
Die meisten Parametrisierungen erzeugen plausible Werte, wobei \pot{Al\_Al0\_AlN} und \pot{newsome} sehr starke Abweichungen zeigen.
Detaillierte Daten zu diesem Test sind in Anhang~\ref{appendix_silicon} zu finden.

\subsubsection{Abscheidungssimulationen (PVD)}

Die Simulationen von Silizium-PVD selbst verlaufen wie in Kapitel~\ref{parsivald} vorgestellt und unterscheiden sich kaum von den Parsivald-Simulationen der vorherigen Abschnitte.
Durch den Aufbau von gerichteten Bindungen zwischen den Silizium-Atomen wird die Bildung amorpher Schichten erwartet, die sich auch durch verringerte Mobilität der Atome auf der Oberfläche ergibt.
Eine Simulation gilt als erfolgreich, wenn die Parsivald-Simulation terminiert und einen dichten Silizium-Film gebildet hat, was nur bei \pot{newsome} nicht der Fall war\todo{was war bei newsome?}.

\subsection{Silizium-PVD}

\continuehere
Silizium-PVD dient in der Produktion elektronischer Bauelemente der Erzeugung einer dünnen, amorphen Siliziumschicht für unterschiedliche Anwendungsszenarien\todo{welche Anwendungen für a-Si? Solarzellen?}, für die konforme Schichten gleichbleibender Qualität gewünscht sind.
Durch den amorphen Charakter des Materials sind nanoskopische Leerstellen und höhere Rauheiten als bei monokristallinen Schichten zu erwarten, die im Folgenden kurz untersucht werden sollen.
\todo{Felix schreibt so was auch, wenn er keinen besseren Ausdruck findet.}Rechenaufwendigere Rechenvorschriften des ReaxFF-Potentiales legen eine längere Simulationsdauer als bei EAM-Potentialen nahe, weshalb nur eine kleine Menge an Simulationen durchgeführt wurde.
Typische Laufzeiten von mehreren Wochen wurden für vollständige ReaxFF-Abscheidungssimulationen beobachtet, jedoch sind im Gegensatz zu rein molekulardynamischen Untersuchungen größere Simulationsräume mit isolierten Ereignissen möglich, die eine Reduktion einiger Finite Size-Effekte zur Folge hat.

Als Substrat für die Parsivald-Simulationen wurden unrelaxierte Silizium-Monokristalle mit Oberflächen entlang der drei Kristallebenen (001), (011) und (111) präpariert und durch periodische Erweiterung auf \SI{106.416x103.68}{\angstrom} vergrößert.
Sonstige Parameter umfassen eine Temperatur von \SI{1300}{\kelvin} (der Schmelzpunkt liegt bei \SI{1687}{\kelvin}), Relaxationszeiten von \SI{350}{\femto\second} und MD-Box-Größen von \SI{37x37}{\angstrom}.
Die Auftreffenergie der Silizium-Atome liegt mit \SI{11.2}{\electronvolt} erneut vergleichsweise hoch, wird aber auch hier durch das Thermostat auf einen unbestimmten Wert verringert.
Damit werden im Schnitt \num{1.68} parallele Ereignisse mit durchschnittlich \num{1510.65} Atomen und einer mittleren Laufzeit von \SI{60.71}{\second} berechnet.
Die Laufzeit lässt sich beispielsweise über die Zeitschrittweite noch minimieren, zeigt allerdings den Unterschied in der Laufzeit bei der Nutzung von EAM- und ReaxFF-Potentialen, der einem Faktor von etwa \num{12} für vergleichbare Simulationen entspricht.
Die hohe Temperatur wurden zur Beschleunigung der Relaxationen gewählt und übersteigt die Temperaturen realer Abscheidungen.
Durch Optimierung der Simulation durch Relaxationszeit, Zeitschrittweite, Thermostatdämpfung und Teilchenenergie ließe sich die Temperatur auf einen realistischeren Wert bei gleicher Verlässlichkeit der Simulation senken.

Das Ergebnis der Abscheidungssimulation ist eine vergleichsweise glatte, amorphe Siliziumschicht, die mit konstanter Rate wächst, jedoch eine Zunahme der Rauheit aufgrund von sich langsam verstärkenden Oberflächenunebenheiten aufweist.

Zur Charakterisierung der Kristalleigenschaften der abgeschiedenen Schicht wurden ihre radiale Verteilungsfunktionen berechnet, aus denen ersichtlich ist, dass bereits nach \SI{4}{\angstrom}, also kurz vor der zweiten Korrelationslänge bei \SI{4.4}{\angstrom}, keine langreichweitige Ordnung mehr vorhanden ist.
Die engen Spitzen an den charakteristischen Abständen des reinen Silizium-Kristalles werden durch das Substrat erzeugt\todo{Schicht ohne Substrat RDF-untersuchen}, welches bei \SI{100}{\angstrom} Schichtdicke immerhin noch \SI{20}{\percent} der Struktur ausmacht, jedoch durch anfängliche Relaxierungen zum Teil seine Kristalleigenschaften verloren hat.
Anders als bei Gold oder Kupfer, bei denen metallische Bindungen dominieren, überwiegen in reinem Silizium kovalente Bindungen, so dass die mittleren Koordinationszahl von \num{3.99} anzeigt, dass alle 4 möglichen Bindungen der Siliziumatome tatsächlich ausgeprägt sind.
Somit zeigt sich die ReaxFF-Formulierung erfolgreich in der Darstellung der strukturellen Eigenschaften von Silizium.

\todoline{Anhang-Referenzen vermindern}
Die Unebenheiten der Schicht, welche die Form von Nanoporen annehmen, wachsen im Gegensatz zu den Kupfer-Kratern aus Abschnitt~\ref{copperpvd} mit der Oberfläche entlang der Wachstumsrichtung, schließen sich aber ebenfalls selbsttätig, wenngleich über einen größeren Zeitraum.
Abbildung~\ref{fig:siliconresults-a} stellt dazu über der Simulationszeit neben der Schichtdicke die Rauheit dar, welche im Verlauf der Simulation linear steigt und zuletzt einen RMS-Wert von \SI{1.15}{\nano\meter} annimmt, der experimentellen Erwartungen von \SIrange{1}{10}{\nano\meter} entspricht\cite{gago_nanopatterning_2002}.
An Abbildung~\ref{fig:siliconroughness} lässt sich erkennen, dass die Rauheit \todo{weiter schreiben}asd
\todo{Leider?}Leider ermöglicht die begrenzte Laufzeit der Simulation keine Aussage über den weiteren Verlauf der Rauheit, von der sublinearer Verlauf durch Schließung der Unebenheiten erwartet wird, wie er sich bei Sputterprozessen zeigt\cite{gago_nanopatterning_2002}\todo{Hinweis auf nicht-sublinearen Verlauf!}.

\begin{figure}
  \captionsetup[subfigure]{singlelinecheck=false}
  \def\subfigwidth{0.48\textwidth}
  \begin{subfigure}[t]{\subfigwidth}
    \includegraphics[width=\textwidth]{Si111_combined}
    \subcaption{Dicke und Rauheit der Schicht}
    \label{fig:siliconresults-a}
  \end{subfigure}
  \hfill
  \begin{subfigure}[t]{\subfigwidth}
    \includegraphics[width=\textwidth]{si111_rdf}
    \subcaption{Radiale Verteilungsfunktion bei $t=80$}
    \label{fig:siliconresults-b}
  \end{subfigure}
  \caption[Struktur einer Silizium-PVD-Schicht aus Parsivald]{
    Struktur einer Silizium-PVD-Schicht aus Parsivald (\SI{10x10}{\nano\meter})
  }
  \label{fig:siliconresults}
\end{figure}

Abbildung~\ref{fig:siliconprofile} stellt die räumliche Verteilung der Unebenheiten dar, die sich lokal in Kratern und Poren von bis zu \SI{32}{\angstrom} Tiefe konzentrieren, aufgrund ihrer geringen Breite aber nur zu einer RMS-Rauheit von \SI{11.5}{\angstrom} führen.
Breitere Krater haben sich durch die geringe Größe des Simulationsraumes nicht entwickelt, jedoch wäre eine Untersuchung einer ca. \SI{500x500}{\angstrom} breiten Struktur auf deren Bildung interessant.
Anhand des Profiles lässt sich auch erkennen, dass mitunter längere Relaxationszeiten oder höhere Teilchenenergien notwendig wären, um Porenbildung weiter zu verringern und langreichweitigere Unebenheiten zu befördern, wie sie etwa bei der Bildung nanoskopischer Silizium-Partikel auftreten würden.

Zum Vergleich beinhaltet Abbildung~\ref{fig:siliconunderrelaxedprofile} das Oberflächenprofil einer unterrelaxierten Oberfläche, wie sie während der Anpassung der Parsivald-Parameter entstanden sind.
Es zeigen sich stärkere Unterschiede und steilere Hänge, die sich aus der hohen Porösität des Materiales ergeben.
Die Porentiefen betragen \SI{6}{\nano\meter} und wachsen linear mit der Schichtdicke, wobei sich aus Zählung der Atome und des Volumens eine Dichte von \SI{2.634}{\gpcc} ergibt, welche durch die Unterschätzung der mittleren Höhe der Oberfläche durch die Nanoporen etwas überschätzt wird und somit oberhalb der kristallinen Dichte von \SI{2.32}{\gpcc} liegt.
Somit ist zu erwarten, dass die Rauheit der Schicht mit stärkerer Relaxierung während der Abscheidung weiter abnimmt.

\begin{figure}[H]
  \centering
  \captionsetup[subfigure]{singlelinecheck=false}
  \begin{subfigure}[t]{7.1cm}
    \includegraphics[width=\textwidth]{si111_surface_profile}
  \end{subfigure}
  \begin{subfigure}[t]{1.7cm}
    \def\svgwidth{\textwidth}
    \begin{overpic}[width=0.7cm]{greyhalfscale}
      \put(0,0){\input{img/si111_surface_profile_halfscale.pdf_tex}}
    \end{overpic}
  \end{subfigure}
  \caption[Oberflächenprofil einer Silizium-PVD-Schicht]{
    Oberflächenprofil einer auf Si-(111) per PVD abgeschiedenen Schicht
  }
  \label{fig:siliconprofile}
\todoline{Längenskala}
\end{figure}

\begin{figure}[H]
  \centering
  \captionsetup[subfigure]{singlelinecheck=false}
  \begin{subfigure}[t]{7.1cm}
    \includegraphics[width=\textwidth]{si111_underrelaxed_profile}
  \end{subfigure}
  \begin{subfigure}[t]{1.7cm}
    \def\svgwidth{\textwidth}
    \begin{overpic}[width=0.66cm]{greyscale}
      \put(0,0){\input{img/si111_underrelaxed_profile_scale.pdf_tex}}
    \end{overpic}
  \end{subfigure}
  \caption[Oberflächenprofil einer unterrelaxierten Siliziumschicht]{
    Oberflächenprofil einer unterrelaxierten, porösen Silizium-PVD-Schicht
  }
  \label{fig:siliconunderrelaxedprofile}
\end{figure}

\subsection{Voruntersuchungen für Siliziumdioxid-CVD}

ReaxFF-Potentiale versprechen die Simulation von Molekülen und deren Reaktion miteinander, die mit den folgenden Tests für Silan und molekularen Sauerstoff überprüft werden sollen.

\subsubsection{Stabilität der Precursormoleküle (\ce{SiH4}, \ce{O2})}

Simulationen einzelner und mehrerer Precursormoleküle (\ce{SiH4} und \ce{O2}) hinsichtlich ihrer Stabilität wurden im mikrokanonischen beziehungsweise kanonischen Ensemble bei verschiedenen Temperaturen durchgeführt.
\todo{Abbildung hier her kopieren}Abbildung~\ref{fig:silanestability} zeigt eine Auswahl der Ergebnisse der Silan-Simulationen, an denen sich erkennen lässt, wie instabile Simulationen zur Ablösung der Wasserstoffatome vom Silanmolekül führen.

\subsubsection{Reaktion der Precursormoleküle (\ce{SiH4 + O2})}

Reaktionen von einzelnen Precursormolekülen wurden stichprobenartig in verschiedenen Orientierungen, Energien und Temperaturen vorgenommen, um einen Überblick über die Verlässlichkeit zu bekommen.
Zusätzlich wurden durchmischte Precursorgase mit dem Ziel eventueller Reaktionen simuliert, was jedoch mit keiner der Parametrisierungen zum gewünschten Erfolg bei hoher Zuverlässigkeit führte.
Einige Parametrisierungen zeigen jedoch vielversprechende Teilreaktionen, die korrekte Doppelbindungen und Bildung von Wasserstoffmolekülen beinhalten (\todo{Abbildung hier her kopieren}Abbildung~\ref{fig:precursorreactions}).
Vor allem bei größeren Reaktionsräumen bilden sich Cluster aus Precursormolekülen, die von attraktiven Termen in den Kraftfeldern dominiert werden, aber nicht durch chemische Wechselwirkungen zu erklären sind (\todo{Abbildung hier her kopieren}Abbildung~\ref{fig:precursorclusters}).


%% \clearpage
%% \section{Aluminiumoxid-ALD}
\label{aluminaald}

Einen beispielhaften Prozess für Atomlagenabscheidungen bildet die Abscheidung von Aluminiumoxid \ce{Al2O3}\cite{puurunen_surface_2005}, für die häufig das Precursorpaar Trimethylaluminium (TMA, \ce{Al(CH4)3}) und Wasser genutzt wird.
Mit einer Permittivität von $k=\approx 8$ hat Aluminiumoxid neben anderen Materialien das zuvor geläufige Siliziumdioxid ($k=3.9$) in der Halbleiterindustrie etwa in modernen Prozessoren ersetzt, obwohl die Reaktionsmechanismen nicht vollends verstanden sind\todo{dringend ref!}.
Deshalb soll der TMA-\ce{H2O}-Prozess im Folgenden untersucht werden, allerdings ist seine vollständige Simulation per Parsivald mit vollständigen Precursormolekülen bisher nicht gelungen.

\subsection{Precursor-Simulationen}

Die Zahl der verfügbaren Parametrisierungen reduziert sich für \ce{Al2O3} auf drei Kandidaten, die bereits aus den Untersuchungen der Silizium-Potentiale bekannt sind:\\
\todo{Stil ok?}\texttt{Al\_AlO\_AlN}, \texttt{liu\_ettringite}, das nachfolgend nur unter der Bezeichnung \texttt{liu} geführt werden soll, und \texttt{narayanan}.

Die Al\_AlO\_AlN-Datei stammt direkt aus der offiziellen LAMMPS-Distribution, wurde jedoch spätestens in der Version vom 17. Dezember 2014 aus dem Paket genommen, was vermutlich durch die schlechte Darstellung der Materialeigenschaften von Bulk-Materialien liegt, wie nachfolgend untersucht wird.

Bei der Liu-Potentialdatei rührt die Unterstützung für sowohl Aluminium als auch Silizium von der Abwandlung einer bestehender Parametrisierungen, die bereits Silizium unterstützt hat.
Da jedoch bei der Erweiterung auf Ettringit (\ce{Ca6[Al(OH)6]2(SO4)3 26H2O}) die Konsistenz der Silizium-Parameter nicht bewahrt wurde, können Silizium-Materialien damit nicht verlässlich simuliert werden, wie in Abschnitt \ref{siliconpvd} gezeigt wurde.
Ettringit hingegen enthält \ce{Al}-\ce{O}-Bindungen und \ce{OH}-Gruppen, so dass zumindest die Simulation des Bulkmateriales und einer hydroxylierten Oberfläche aussichtsreich erscheint.

Die Narayanan-Parametrisierung lässt aufgrund ihrer Herkunft als Potential für \ce{Li-Al}-Silikate für die Simulation von Eukryptit (\ce{LiAl[SiO4]}) endgültige Aussage über die Qualität der \ce{Al}-\ce{O}-Bindungen und \ce{OH}-Gruppen zu.
Zwar besteht der Trainingssatz aus Lithium-Aluminium-Kristallen, aber nur $\gamma$-\ce{LiAlO2} beinhaltet direkte \ce{Al}-\ce{O}-Bindungen, während keine der für das Fitting genutzten Strukturen Wasserstoff beinhaltet hat.
Es ist daher unwahrscheinlich, dass die Narayanan-Potentialdatei komplizierte Systeme verlässlich darstellt.

\subsection{Precursor-Oberflächen-Reaktionen}

Der erste Schritt in der Simulation von Aluminiumoxid ist \todo{continuehere}

\begin{figure}
  \captionsetup[subfigure]{singlelinecheck=false}
  \def\subfigwidth{0.32\textwidth}
  \begin{subfigure}[t]{\subfigwidth}
    \includegraphics[width=\textwidth]{alumina_h2o_before}
    \subcaption{Seitenansicht, vorher}
  \end{subfigure}
  \hfill
  \begin{subfigure}[t]{\subfigwidth}
    \includegraphics[width=\textwidth]{alumina_h2o_after}
    \subcaption{Seitenansicht, nachher}
  \end{subfigure}
  \hfill
  \begin{subfigure}[t]{\subfigwidth}
    \includegraphics[width=\textwidth]{alumina_h2o_topview}
    \subcaption{Draufsicht, nachher.
      Hydroxyl ist grün hervorgehoben.
    }
  \end{subfigure}
  \caption[Oberflächenreaktion von Wasser mit $\alpha$-\ce{Al2O3}]{Ergebnisse einer Oberflächenreaktion von Wasser mit $\alpha$-\ce{Al2O3}.
    Das Wasser reagiert mit Sauerstoffatomen an der Oberfläche zu Hydroxylgruppen.
  }
  \label{fig:wateraluminasurface}
\end{figure}

\subsection{Vollständige ALD-Simulationen}

\todo[inline]{Erklären, warum es vermutlich noch nicht geht}

