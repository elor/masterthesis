\graphicspath{{img/},{img-results/}}

\pagestyle{scrheadings} % Standart Kopf- und Fußzeile
\setkomafont{pageheadfoot}{\small\scshape} % for new Koma Script

% ---------------------------------------------------------------------
%\setcounter{secnumdepth}{2}
%\setcounter{chapter}{-1}
\setcounter{tocdepth}{2}

% ---------------------------------------------------------------------
%\sloppy % weniger Worttrennungen, größere Wortabstände
\fussy % viele Worttrennungen, "schönere" Wortabstände

% ---------------------------------------------------------------------
%\flushbottom % Ausrichtung der Seitenenden jeweils auf
% gleicher Höhe

% ---------------------------------------------------------------------
%\sloppypar % Das hier relaxt die Einstellungen zum Wortabstand
% extrem. Damit ragen keine Worte über den rechten
% Zeilenabstand hinaus. Dafür muß stärker auf
% Wortabstand geachtet werden, der kann dann ziemlich
% groß werden. Man erhält aber keine Meldung mehr
% über underfull boxes.

%Hiermit kann man das gleiche mit weniger Holzhammer erreichen:
%\setlength{\tolerance}{2000} % Strafpunkt für Zeilenumbruch
%\setlength{\emergencystretch}{3pt} % Soweit dürfen einzelne Worte mehr
% auseinandergezogen werden
%\setlength{\hfuzz}{1pt} % Macht den rechten Rand um bis zu 1pt
% flatterig.

% ---------------------------------------------------------------------
%Vermeiden einzelner Zeilen am Ende einer Seite oder oben auf einer neuen Seite
\clubpenalty10000
\widowpenalty10000

% eigene Definitionen
\newcommand{\BigO}[1]{$\mathcal{O}(#1)$}
\newcommand{\mergedcell}[2][l]{\begin{tabular}[#1]{@{}l@{}}#2\end{tabular}}
\newcommand{\cmark}{\ding{51}}
\newcommand{\ccross}{\ding{55}}

\newcolumntype{C}{>{\centering\arraybackslash}X}
\newcolumntype{R}{>{\hsize=0.15\textwidth}X}
\newcolumntype{Y}{>{\flushleft\arraybackslash}X}

\newcommand{\cG}[1]{\cellcolor{Green}#1}
\newcommand{\cY}[1]{\cellcolor{Yellow}#1}
\newcommand{\cR}[1]{\cellcolor{Red}#1}

\newcommand{\setdefaultcolors}{
  \definecolor{Green}{rgb}{0.7 1.0 0.7}
  \definecolor{Yellow}{rgb}{1.0 1.0 0.7}
  \definecolor{Red}{rgb}{1.0 0.7 0.7}
}

\ifrowcolors
  \ifcolortblfix
    \definecolor{Green}{rgb}{0.0 1.0 0.0}
    \definecolor{Yellow}{rgb}{1.0 1.0 0.0}
    \definecolor{Red}{rgb}{1.0 0.0 0.0}

    \newcommand{\oddrowcolors}{\def\eltransparency{0.1}\rowcolors{1}{}{black}}
    \newcommand{\evenrowcolors}{\def\eltransparency{0.1}\rowcolors{1}{black}{}}

    %% rowcolors fix: This makes the rowcolors boxes transparent, so they won't overwrite hline and vline commands (i.e. table borders)
    \def\eltransparency{1.0}
    \makeatletter
    \def\CT@@do@color{%
      \global\let\CT@do@color\relax
            \@tempdima\wd\z@
            \advance\@tempdima\@tempdimb
            \advance\@tempdima\@tempdimc
            \kern-\@tempdimb
    \transparent{\eltransparency}%
            \leaders\vrule
                    \hskip\@tempdima\@plus 1fill
            \kern-\@tempdimc
            \hskip-\wd\z@ \@plus -1fill }
    \makeatother
    %\setlength{\arrayrulewidth}{0.6pt}

  \else

    \setdefaultcolors

    \definecolor{lightgray}{gray}{0.9}
    \newcommand{\oddrowcolors}{\rowcolors{1}{}{lightgray}}
    \newcommand{\evenrowcolors}{\rowcolors{1}{lightgray}{}}

  \fi

\else

  \setdefaultcolors
  \definecolor{Green}{rgb}{0.7 1.0 0.7}
  \definecolor{Yellow}{rgb}{1.0 1.0 0.7}
  \definecolor{Red}{rgb}{1.0 0.7 0.7}

  \newcommand{\oddrowcolors}{}
  \newcommand{\evenrowcolors}{}

\fi

\def\arraystretch{1.1}

%threeparttable textwidth
\renewcommand{\TPTminimum}{\textwidth}

%% avoid the [inline] option of todo
\newcommand{\todoline}[1]{\todo[inline]{#1}{}}
\newcommand{\continuehere}{\todoline{continuehere}{}}

\newcommand{\ignore}[1]{}

\DeclareSIUnit\gpcc{\gram\per\cubic\centi\meter}

%% disable todo
\ifdraft{
  \newcommand{\Si}[2]{\todo{Tippfehler: \\Si statt \\SI}\SI{#1}{#2}}
}{
  \renewcommand{\todoline}[1]{}
  \renewcommand{\todo}[1]{}
}

\newcommand{\qqquad}[0]{\qquad\quad}
\newcommand{\pot}[1]{\texttt{#1}}

\newcommand{\chemlegend}[1]{\begin{minipage}[c]{0.7cm}\includegraphics[width=\textwidth]{#1}\end{minipage}}
\newcommand{\chemlegendce}[1]{\chemlegend{#1}\ce{#1}}
