\section{Kinetic Monte Carlo-Methoden}
\label{kmc}

Kinetic Monte Carlo-Methoden bezeichnen zufallsbestimmte numerische Prozesse, bei denen das System eine Reihe von diskreten Zuständen durchläuft, die auf Basis einer Übergangswahrscheinlichkeit aus dem vorhergehenden Zustand zufällig gewählt werden.
Ursprünglich für die Simulation von  Diffusionsprozessen entwickelt, finden sie inzwischen in vielen Gebieten der Natur- und Ingenieurswissenschaften Anwendung, insbesondere auch für Nichtgleichgewichtssysteme wie  Oberflächenreaktionen und Schichtabscheidungen.

Die mathematische Formulierung folgt Markov-Ketten, die eine Abfolge von Zuständen $X_t$ aus einem abzählbar großen Zustandsraum $S$ darstellen.

ASD

$$
X_t = S \forall t \geq 0
$$
$$
X_{t+\Delta t} = E(X_t, t)
$$
$$
\Delta t = - \frac{\ln(u)}{r_i}
$$

