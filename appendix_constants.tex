\chapter{Physikalische Konstanten und Stoffeigenschaften}
\label{appendix_constants}

\footnotetext[1]{Die Kristall-Bindungslängen wurden über die Kristallstrukturen aus den Gitterkonstanten berechnet}
\footnotetext[2]{Die Dichte bei höheren Temperaturen weit unterhalb des Schmelzpunktes wurde über die Dichte bei Raumtemperatur und den linearen Ausdehnungskoeffizienten berechnet}

\begin{table}[!h]
  \centering
  \caption{Physikalische Konstanten}
  \oddrowcolors
  \begin{tabularx}{\textwidth}{|XXR|}
    \hline
    \textbf{Größe}                  & \textbf{Wert}                               & \textbf{Referenz}            \\
    \hline
    Avogadro-Konstante $N_\text{A}$ & \SI{6.02214179e23}{\per\mole}               & \cite{haynes_crc_2011} S.1-1 \\
    Boltzmann-Konstante $k_B$       & \SI{1.3806504e-23}{\joule\per\kelvin}       & \cite{haynes_crc_2011} S.1-2 \\
    Molare Gaskonstante $R$         & \SI{8.314472}{\joule\per\mole\per\kelvin} & \cite{haynes_crc_2011} S.1-2 \\
    Atomare Masseneinheit $u$       & \SI{1.660538782e-27}{\kilo\gram}            & \cite{haynes_crc_2011} S.1-2 \\
    \hline
  \end{tabularx}
\end{table}

\begin{table}[!h]
  \centering
  \caption{Eigenschaften von Gold}
  \evenrowcolors
  \begin{tabularx}{\textwidth}{|XXR|}
    \hline
    \textbf{Größe}                           & \textbf{Wert}                                  & \textbf{Referenz}               \\
    \hline
    Dichte $\rho$, \SI{300}{\kelvin}         & \SI{19.3}{\gpcc}                               & \cite{haynes_crc_2011} S.4-65   \\
    Dichte $\rho$, \SI{500}{\kelvin}         & \SI{19.13}{\gpcc}                              & berechnet\footnotemark[2]       \\
    Dichte $\rho_m$, flüssig                 & \SI{17.31}{\gpcc}                              & \cite{haynes_crc_2011} S.4-128  \\
    Linearer Ausdehnungskoeffizient $\alpha$ & \SI{14.2e-6}{\per\kelvin}                      & \cite{haynes_crc_2011} S.12-206 \\
    Schmelztemperatur $T_m$                  & \SI{1064.18}{\celsius} (\SI{1337.33}{\kelvin}) & \cite{haynes_crc_2011} S.4-65   \\
    Atomgewicht $u$                          & \SI{196.967}{\gram\per\mole}                   & \cite{haynes_crc_2011} S.1-12   \\
    Kristallstruktur                         & fcc                                            & \cite{haynes_crc_2011} S.4-147  \\
    Gitterkonstante $a$                      & \SI{4.0786}{\angstrom}                         & \cite{haynes_crc_2011} S.4-147  \\
    Bindungslänge $r_\text{bond}$            & \SI{2.8840}{\angstrom}                         & berechnet\footnotemark[1]       \\
    \hline
  \end{tabularx}
\end{table}

\clearpage

\footnotetext[1]{Die Kristall-Bindungslängen wurden über die Kristallstrukturen aus den Gitterkonstanten berechnet}

\begin{table}[!h]
  \centering
  \caption{Eigenschaften von Kupfer}
  \oddrowcolors
  \begin{tabularx}{\textwidth}{|XXR|}
    \hline
    \textbf{Größe}                           & \textbf{Wert}                                  & \textbf{Referenz}               \\
    \hline
    Dichte $\rho$, fest                      & \SI{8.96}{\gpcc}                               & \cite{haynes_crc_2011} S.4-61   \\
    Dichte $\rho_m$, flüssig                 & \SI{7.997}{\gpcc}                              & \cite{haynes_crc_2011} S.4-128  \\
    Schmelztemperatur $T_m$                  & \SI{1084.62}{\celsius} (\SI{1357.77}{\kelvin}) & \cite{haynes_crc_2011} S.4-61   \\
    Atomgewicht $u$                          & \SI{63.546}{\gram\per\mole}                    & \cite{haynes_crc_2011} S.1-12   \\
    Kristallstruktur                         & fcc                                            & \cite{haynes_crc_2011} S.4-146  \\
    Gitterkonstante $a$                      & \SI{3.6150}{\angstrom}                         & \cite{haynes_crc_2011} S.4-146  \\
    Bindungslänge $r_\text{bond}$            & \SI{2.5562}{\angstrom}                         & berechnet\footnotemark[1]       \\
    Linearer Ausdehnungskoeffizient $\alpha$ & \SI{16.5e-6}{\per\kelvin}                      & \cite{haynes_crc_2011} S.12-206 \\
    \hline
  \end{tabularx}
\end{table}

\begin{table}[!h]
  \centering
  \caption{Eigenschaften von Nickel}
  \evenrowcolors
  \begin{tabularx}{\textwidth}{|XXR|}
    \hline
    \textbf{Größe}                & \textbf{Wert}          & \textbf{Referenz}              \\
    \hline
    Kristallstruktur              & fcc                    & \cite{haynes_crc_2011} S.4-150 \\
    Gitterkonstante $a$           & \SI{3.5238}{\angstrom} & \cite{haynes_crc_2011} S.4-150 \\
    Bindungslänge $r_\text{bond}$ & \SI{2.4917}{\angstrom} & berechnet\footnotemark[1]      \\
    \hline
  \end{tabularx}
\end{table}

\begin{table}[!h]
  \centering
  \caption{Eigenschaften von Silizium}
  \oddrowcolors
  \begin{tabularx}{\textwidth}{|XXR|}
    \hline
    \textbf{Größe}                & \textbf{Wert}                            & \textbf{Referenz}              \\
    \hline
    Dichte $\rho$, kristallin     & \SI{2.3296}{\gpcc}                       & \cite{haynes_crc_2011} S.4-87  \\
    Dichte $\rho$, amorpher Film  & \SI{2.29}{\gpcc}                         & \cite{remes_optical_1998}      \\
                                  & (\SI{2.2}{\gpcc} - \SI{2.24}{\gpcc})     & \cite{renner_density_1973}     \\
    Schmelztemperatur $T_m$       & \SI{1414}{\celsius} (\SI{1687}{\kelvin}) & \cite{haynes_crc_2011} S.4-87  \\
    Atomgewicht $u$               & \SI{28.086}{\gram\per\mole}              & \cite{haynes_crc_2011} S.4-13  \\
    Kristallstruktur              & diamant                                  & \cite{haynes_crc_2011} S.4-151 \\
    Gitterkonstante $a$           & \SI{5.4305}{\angstrom}                   & \cite{haynes_crc_2011} S.4-151 \\
    Bindungslänge $r_\text{bond}$ & \SI{2.3515}{\angstrom}                   & berechnet\footnotemark[1]      \\
    \hline
  \end{tabularx}

\end{table}

\clearpage

\footnotetext[1]{Die Kristall-Bindungslängen wurden über die Kristallstrukturen aus den Gitterkonstanten berechnet}

\begin{table}[!h]
  \centering
  \caption{Struktur der Silizium-CVD-Precursormoleküle}
  \evenrowcolors
  \begin{tabularx}{\textwidth}{|XXR|}
    \hline
    \textbf{Größe}                                      & \textbf{Wert}          & \textbf{Referenz}             \\
    \hline
    Struktur von Silan $\left(\ce{SiH4}\right)$         & tetraedrisch           & \cite{haynes_crc_2011} S.9-29 \\
    Bindungslänge in \ce{SiH4}                          & \SI{1.4798}{\angstrom} & \cite{haynes_crc_2011} S.9-29 \\
    Bindungslänge von Sauerstoff $\left(\ce{O2}\right)$ & \SI{1.2074}{\angstrom} & \cite{haynes_crc_2011} S.9-26 \\
    \hline
  \end{tabularx}

\end{table}

\begin{table}[!h]
  \centering
  \caption{Eigenschaften von Aluminiumoxid}
  \oddrowcolors
  \begin{tabularx}{\textwidth}{|XXR|}
    \hline
    \textbf{Größe}                                         & \textbf{Wert}                            & \textbf{Referenz}                   \\
    \hline
    Dichte $\rho$, $\alpha$-kristallin, \SI{300}{\kelvin}  & \SI{3.99}{\gpcc}                         & \cite{haynes_crc_2011} S.4-45       \\
                                                           & \SI{3.98}{\gpcc}                         & \cite{fiquet_high-temperature_1999} \\
    Dichte $\rho$, $\alpha$-kristallin, \SI{1500}{\kelvin} & \SI{3.80}{\gpcc}                         & \cite{fiquet_high-temperature_1999} \\
    Dichte $\rho$, $\gamma$-kristallin, \SI{300}{\kelvin}  & \SI{3.67}{\gpcc}                         & \cite{dynys_alpha_1982}             \\
    Dichte $\rho$, amorph, \SI{300}{\kelvin}               & \SI{3.2}{\gpcc} - \SI{3.9}{\gpcc}          & \cite{wang_dependence_1997}         \\
    Schmelztemperatur $T_m$, $\alpha$-kristallin           & \SI{2054}{\celsius} (\SI{2327}{\kelvin}) & \cite{haynes_crc_2011} S.4-87       \\
    Atomgewicht $u$, \ce{Al}                               & \SI{26.982}{\gram\per\mole}              & \cite{haynes_crc_2011} S.1-12       \\
    Atomgewicht $u$, \ce{O}                                & \SI{15.999}{\gram\per\mole}              & \cite{haynes_crc_2011} S.1-13       \\
    Kristallstruktur                                       & corundum                                 & \cite{haynes_crc_2011} S.4-146      \\
    Gitterkonstante $a$                                    & \SI{4.7591}{\angstrom}                   & \cite{haynes_crc_2011} S.4-146      \\
    Gitterkonstante $c$                                    & \SI{12.9894}{\angstrom}                  & \cite{haynes_crc_2011} S.4-146      \\
    Bindungslänge $r_\text{bond}$ & \SI{1.90}{\angstrom}                   & berechnet\footnotemark[1]      \\
    \hline
  \end{tabularx}

\end{table}

\begin{table}[!h]
  \centering
  \caption{Struktur der \ce{Al2O3}-ALD-Precursormoleküle}
  \evenrowcolors
  \begin{tabularx}{\textwidth}{|XXR|}
    \hline
    \textbf{Größe}                                & \textbf{Wert}          & \textbf{Referenz}             \\
    \hline
    Bindungswinkel von Wasser                     & \SI{104.51}{\degree}   & \cite{haynes_crc_2011} S.9-24 \\
    \ce{O-H}-Bindungslänge in Wasser              & \SI{0.9575}{\angstrom} & \cite{haynes_crc_2011} S.9-24 \\
    Struktur von TMA $\left(\ce{Al(CH3)3}\right)$ & trigonal-planar        & \cite{haynes_crc_2011} S.9-46 \\
    \ce{Al-C}-Bindungslänge in TMA                & \SI{1.957}{\angstrom}  & \cite{haynes_crc_2011} S.9-46 \\
    \ce{C-H}-Bindungslänge in TMA                 & \SI{1.113}{\angstrom}  & \cite{haynes_crc_2011} S.9-46 \\
    \hline
  \end{tabularx}
\end{table}
